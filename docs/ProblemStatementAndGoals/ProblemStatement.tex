\documentclass{article}

\usepackage{tabularx}
\usepackage{booktabs}

\title{Problem Statement and Goals\\\progname}

\author{\authname}

\date{}

%% Comments

\usepackage{color}

\newif\ifcomments\commentstrue %displays comments
%\newif\ifcomments\commentsfalse %so that comments do not display

\ifcomments
\newcommand{\authornote}[3]{\textcolor{#1}{[#3 ---#2]}}
\newcommand{\todo}[1]{\textcolor{red}{[TODO: #1]}}
\else
\newcommand{\authornote}[3]{}
\newcommand{\todo}[1]{}
\fi

\newcommand{\wss}[1]{\authornote{magenta}{SS}{#1}} 
\newcommand{\plt}[1]{\authornote{cyan}{TPLT}{#1}} %For explanation of the template
\newcommand{\an}[1]{\authornote{cyan}{Author}{#1}}

%% Common Parts

\newcommand{\progname}{SFWRENG 4G06} % PUT YOUR PROGRAM NAME HERE
\newcommand{\authname}{Team 9, Flow
\\ Hussain Muhammed
\\ Chengze Zhao
\\ Jeffrey Doan
\\ Kevin Zhu
\\ Ethan Patterson } % AUTHOR NAMES                   

\usepackage{hyperref}
    \hypersetup{colorlinks=true, linkcolor=blue, citecolor=blue, filecolor=blue,
                urlcolor=blue, unicode=false}
    \urlstyle{same}
                                


\begin{document}

\maketitle

\begin{table}[hp]
\caption{Revision History} \label{TblRevisionHistory}
\begin{tabularx}{\textwidth}{llX}
\toprule
\textbf{Date} & \textbf{Developer(s)} & \textbf{Change}\\
\midrule
Date1 & Name(s) & Description of changes\\
Date2 & Name(s) & Description of changes\\
... & ... & ...\\
\bottomrule
\end{tabularx}
\end{table}

\section{Problem Statement}

\wss{You should check your problem statement with the
\href{https://github.com/smiths/capTemplate/blob/main/docs/Checklists/ProbState-Checklist.pdf}
{problem statement checklist}.} 

\wss{You can change the section headings, as long as you include the required
information.}

\subsection{Problem}

\subsection{Inputs and Outputs}

\wss{Characterize the problem in terms of ``high level'' inputs and outputs.  
Use abstraction so that you can avoid details.}


\subsubsection{Inputs}
\begin{itemize}
	\item Combinations of keystrokes that:
	      \begin{itemize}
		      \item Spell out defined commands/shortcuts
		      \item Spell out note text for display
	      \end{itemize}
	\item User defined keybinds
\end{itemize}

\subsubsection{Outputs}
\begin{itemize}
	\item Displayed text on the screen.
	\item Displayed diagrams/geometry on the screen.
	\item Content (i.e. text and diagrams) on screen that can be saved in a file and re-opened to display the same content
\end{itemize}

\subsection{Stakeholders}
Broadly, the stakeholders for this project will all fall into a group of more
technically savvy individuals who are likely comfortable using keyboard
shortcuts in other familiar software programs. The target users should be able
to memorize a combination of shortcuts and commands to align with the keyboard
based usage.

\begin{itemize}
	\item People who use keyboard based workflows and prefer not to use their mouse to take notes that involve simple diagrams.
	\item People who do not have access to a drawing tablet, but would like to take digital notes involving diagrams in an efficient way.
	\item People who often take notes on a computer and see value in learning a new system to improve their note taking efficiency.
\end{itemize}

\subsection{Environment}

\wss{Hardware and Software Environment}

\section{Goals}

\section{Stretch Goals}

\section{Extras}

\wss{For CAS 741: State whether the project is a research project. This
designation, with the approval (or request) of the instructor, can be modified
over the course of the term.}

\wss{For SE Capstone: List your extras.  Potential extras include usability
testing, code walkthroughs, user documentation, formal proof, GenderMag
personas, Design Thinking, etc.  (The full list is on the course outline and in
Lecture 02.) Normally the number of extras will be two.  Approval of the extras
will be part of the discussion with the instructor for approving the project.
The extras, with the approval (or request) of the instructor, can be modified
over the course of the term.}

\newpage{}

\section*{Appendix --- Reflection}

\wss{Not required for CAS 741}

The purpose of reflection questions is to give you a chance to assess your own
learning and that of your group as a whole, and to find ways to improve in the
future. Reflection is an important part of the learning process.  Reflection is
also an essential component of a successful software development process.  

Reflections are most interesting and useful when they're honest, even if the
stories they tell are imperfect. You will be marked based on your depth of
thought and analysis, and not based on the content of the reflections
themselves. Thus, for full marks we encourage you to answer openly and honestly
and to avoid simply writing ``what you think the evaluator wants to hear.''

Please answer the following questions.  Some questions can be answered on the
team level, but where appropriate, each team member should write their own
response:


\begin{enumerate}
    \item What went well while writing this deliverable? 
    \item What pain points did you experience during this deliverable, and how
    did you resolve them?
    \item How did you and your team adjust the scope of your goals to ensure
    they are suitable for a Capstone project (not overly ambitious but also of
    appropriate complexity for a senior design project)?
\end{enumerate}  

\end{document}