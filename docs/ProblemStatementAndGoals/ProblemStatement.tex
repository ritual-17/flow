\documentclass{article}

\usepackage{tabularx}
\usepackage{booktabs}

\title{Problem Statement and Goals\\\progname}

\author{\authname}

\date{}

%% Comments

\usepackage{color}

\newif\ifcomments\commentstrue %displays comments
%\newif\ifcomments\commentsfalse %so that comments do not display

\ifcomments
\newcommand{\authornote}[3]{\textcolor{#1}{[#3 ---#2]}}
\newcommand{\todo}[1]{\textcolor{red}{[TODO: #1]}}
\else
\newcommand{\authornote}[3]{}
\newcommand{\todo}[1]{}
\fi

\newcommand{\wss}[1]{\authornote{magenta}{SS}{#1}} 
\newcommand{\plt}[1]{\authornote{cyan}{TPLT}{#1}} %For explanation of the template
\newcommand{\an}[1]{\authornote{cyan}{Author}{#1}}

%% Common Parts

\newcommand{\progname}{SFWRENG 4G06} % PUT YOUR PROGRAM NAME HERE
\newcommand{\authname}{Team 9, Flow
\\ Hussain Muhammed
\\ Chengze Zhao
\\ Jeffrey Doan
\\ Kevin Zhu
\\ Ethan Patterson } % AUTHOR NAMES                   

\usepackage{hyperref}
    \hypersetup{colorlinks=true, linkcolor=blue, citecolor=blue, filecolor=blue,
                urlcolor=blue, unicode=false}
    \urlstyle{same}
                                


\begin{document}

\maketitle

\begin{table}[hp]
\caption{Revision History} \label{TblRevisionHistory}
\begin{tabularx}{\textwidth}{llX}
\toprule
\textbf{Date} & \textbf{Developer(s)} & \textbf{Change}\\
\midrule
Date1 & Name(s) & Description of changes\\
Date2 & Name(s) & Description of changes\\
... & ... & ...\\
\bottomrule
\end{tabularx}
\end{table}

\section{Problem Statement}

\subsection{Problem}

The speed of thought should be the limiting factor in efficiency when typing or
taking notes. Editors like Vim and Emacs follow this philosophy for text
through keyboard driven shortcuts and commands. However, users can be limited
in applying a similar approach to creating simple diagrams and drawings
alongside text. This would be particularly useful when creating documents that
need to be completed in a timely manner, such as taking notes in a lecture. For
example, disciplines like engineering require taking notes with a combination
of text and diagrams, whether it be simple tables, state machines, or circuits.
\vspace{1em}

\noindent Familiar solutions exist for taking notes involving both text and
diagrams, but each have some limitations:
\begin{itemize}
  \item \textbf{Taking notes with pen and paper/tablet.} While this gives the
    freedom of drawing diagrams the way you want, you are forced to print text
    rather than type it. There is also no option for copy/pasting repetitive
    parts of a diagram. Writing out longer thoughts or sentences can take a
    while, and potentially lead to falling behind. Correcting mistakes or
    repositioning diagrams/text can also be timely.
  \item \textbf{Taking notes with your computer and keyboard.} While this can
    allow for faster speeds with typing, options for inserting diagrams are
    limited and/or slow. Existing diagram description domain-specific languages
    exist such as (PlantUML, Mermaid.js). However, they focus
    specifically and only on diagram creation, which requires definition,
    rendering, and then insertion into your note document. Beyond this,
    manipulating text rather than the visual objects themselves can lead to
    inefficiencies when forgetting syntax and lack immediate feedback from
    changes. Other options embedded in OneNote or Draw.io follow a mouse-based
    workflow, which can be cumbersome for various reasons (e.g. navigating to
    the menu for a given shape you would like to select).
  \item \textbf{Hybrid approach.} It is possible to use both of the above
    approaches together, switching between drawing and typing. This carries its
    own cognitive load for swapping context between two media. Swapping between
    a computer and pencil and paper means the notes will be stored in different
    places. Swapping between a computer and drawing tablet also means
    purchasing another device.
\end{itemize}

\noindent The overarching problem is that no unified digital platform exists for
efficiently taking notes involving text and diagrams using a keyboard-driven
approach.

\subsection{Inputs and Outputs}

\subsubsection{Inputs}
\begin{itemize}
	\item Combinations of keystrokes that:
	      \begin{itemize}
		      \item Spell out defined commands/shortcuts
		      \item Spell out note text for display
	      \end{itemize}
	\item User defined keybinds
\end{itemize}

\subsubsection{Outputs}
\begin{itemize}
	\item Displayed text on the screen.
  \item Displayed diagrams/geometry on the screen.
  \item Content (i.e. text and diagrams) on screen that can be saved in a file
    and re-opened to display the same content
\end{itemize}

\subsection{Stakeholders}
Broadly, the stakeholders for this project will all fall into a group of more
technically savvy individuals who are likely comfortable using keyboard
shortcuts in other familiar software programs. The target users should be able
to memorize a combination of shortcuts and commands to align with the keyboard
based usage.

\begin{itemize}
  \item People who use keyboard based workflows and prefer not to use their
    mouse to take notes that involve simple diagrams.
  \item People who do not have access to a drawing tablet, but would like to
    take digital notes involving diagrams in an efficient way.
  \item People who often take notes on a computer and see value in learning a
    new system to improve their note taking efficiency.
\end{itemize}

\subsection{Environment}

\subsubsection{Hardware Environment}
The hardware environment will be a personal computer running a modern operating
system (i.e. Windows, MacOS, or Linux). The computer will have a keyboard and
mouse for user input.

\subsubsection{Software Environment}
The software environment will be a desktop application.

\section{Goals}
\begin{itemize}
  \item Allow users to create, edit, and save documents involving text and
    diagrams using only their keyboard.
  \item Allow users to generate simple custom geometry involving lines and
    predefined shapes (e.g. circle, rectangle, triangle).
  \item Allow users to directly interact with displayed geometry (i.e. not the
    underlying text representation of the geometry and no required text to
    visual render step).
  \item Allow users to save custom geometry as reusable commands/shortcuts for
    future use.
  \item Provides the user the potential to take notes involving diagrams more
    efficiently (time-wise) than other individual/combined note taking
    applications.
\end{itemize}

\section{Stretch Goals}
\begin{itemize}
  \item Provides the user the potential to take notes involving diagrams more
    efficiently than by hand on paper/drawing tablet.
  \item Ability to export documents as images and/or PDFs.
  \item Support for collaborative note taking.
  \item Support for LaTeX math rendering.
  \item Support for Vim style keybindings when editing text.
\end{itemize}

\section{Extras}
\begin{itemize}
  \item Usability testing
  \item User manual
\end{itemize}

\wss{For CAS 741: State whether the project is a research project. This
designation, with the approval (or request) of the instructor, can be modified
over the course of the term.}

\wss{For SE Capstone: List your extras.  Potential extras include usability
testing, code walkthroughs, user documentation, formal proof, GenderMag
personas, Design Thinking, etc.  (The full list is on the course outline and in
Lecture 02.) Normally the number of extras will be two.  Approval of the extras
will be part of the discussion with the instructor for approving the project.
The extras, with the approval (or request) of the instructor, can be modified
over the course of the term.}

\newpage{}

\section*{Appendix --- Reflection}

\wss{Not required for CAS 741}

The purpose of reflection questions is to give you a chance to assess your own
learning and that of your group as a whole, and to find ways to improve in the
future. Reflection is an important part of the learning process.  Reflection is
also an essential component of a successful software development process.  

Reflections are most interesting and useful when they're honest, even if the
stories they tell are imperfect. You will be marked based on your depth of
thought and analysis, and not based on the content of the reflections
themselves. Thus, for full marks we encourage you to answer openly and honestly
and to avoid simply writing ``what you think the evaluator wants to hear.''

Please answer the following questions.  Some questions can be answered on the
team level, but where appropriate, each team member should write their own
response:


\begin{enumerate}
    \item What went well while writing this deliverable? 
    \begin{itemize}
        \item Chengze - I came up with many ideas for on stakeholders 
        and goals based on our initial problem statement. And I was 
        able to clearly articulate these ideas in the document.
        \item Ethan - Through filling out the problem statement and goals
          sections, I gained a more concrete idea of the requirements for our
          project. I think it helped flesh out the converging goals of improved
          efficiency and support for keyboard driven workflows. This was
          particularly helpful in the stakeholders section, which is a crucial
          section for creating goals and usability tests going forward.
          Originally we had broad ideas for stakeholders like "Students" and
          "Professionals", but from the brainstorming done in other sections we
          were able to narrow them down to more precise descriptions that we
          will design the product for.
        \item Hussain - While writing the Proof of Concept Demonstration Plan, I 
        gained a key understanding of the risks involved in the project.
        This not only helped me understand the challenges we may face, but also
        allowed me to bring up the risks to my team members and the TA. This
        further lead to in depth discussions on how we can mitigate these risks. 
        Confirming an effective proof of concept demonstration plan was a key
        milestone for our team, giving us confidence in the feasibility of
        our project.
        \item Jeff - Our first meeting discussing how we wanted to assign roles 
        and tasks for this deliverable went very well. 
        We were able to easily distribute tasks to members without any disagreements.
        \item Kevin - Our group was able to easily assign everyone their work 
        and get the work done when needed. Additionaly, the team communicated 
        effectively meaning that everyone was participating.
    \end{itemize}

    \item What pain points did you experience during this deliverable, and how
    did you resolve them?
    \begin{itemize}
        \item Chengze - I found it difficult to limit the number 
        of goals of the project to be within the appropriate number 
        of main goals suggested. I resolved this by discussing with 
        my TA and my teammates, we agreed to focus on the core 
        functionalities of the project and leave some features as 
        stretch goals. 
        \item Ethan - It was difficult to come up with "selling point" goals
          and describe them in a concise way. We felt that a lot of our goals
          sound similar at first glance, but they subtly describe different
          goals for what the system would provide/look like. In general it was
          also a challenge to remind myself to focus on "what" rather than
          "how" for our problem statement and goals.
        \item Hussain - One pain point I experienced was writing the 
        expected technologies section. I was initially confused as to how 
        specific I should be when listing the technologies. A simple
        confirmation from the TA helped me understand that I should
        list the specific technologies I expect to use, but also note that
        these may change as the project progresses.
        \item Jeff - During this deliverable, I found it difficult to 
        navigate and install all the tools required for this capstone. 
        Additionally, I had issues running the make file and 
        couldn't troubleshoot it. In order to resolve this issue,
        I asked my team members who had already finished configuring their
        environment for assistance.
        \item Kevin - One pain point of this deliverable was understanding the 
        GitHub merge checks. As I had never used this type of code managment 
        before, we had issues merging when needed. 
    \end{itemize}
    
    \item How did you and your team adjust the scope of your goals to ensure
    they are suitable for a Capstone project (not overly ambitious but also of
    appropriate complexity for a senior design project)?
    \begin{itemize}
        \item We discussed the goals and stretch goals as a team
        and prioritized the core functionalities of the project.
        Instead of focusing on the features/goals that satisfy all 
        the stakeholders, we decided to go with the functionalities that
        we desire to have in the project. And for those goals that are 
        challenging to implement, we put them as stretch goals so that 
        we can focus on the main goals first, and then work on the 
        stretch goals if we have time.
    \end{itemize}
\end{enumerate}  

\end{document}
