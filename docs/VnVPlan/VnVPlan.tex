\documentclass[12pt, titlepage]{article}

\usepackage{booktabs}
\usepackage{tabularx}
\usepackage{hyperref}
\hypersetup{
    colorlinks,
    citecolor=blue,
    filecolor=black,
    linkcolor=red,
    urlcolor=blue
}
\usepackage[round]{natbib}

%% Comments

\usepackage{color}

\newif\ifcomments\commentstrue %displays comments
%\newif\ifcomments\commentsfalse %so that comments do not display

\ifcomments
\newcommand{\authornote}[3]{\textcolor{#1}{[#3 ---#2]}}
\newcommand{\todo}[1]{\textcolor{red}{[TODO: #1]}}
\else
\newcommand{\authornote}[3]{}
\newcommand{\todo}[1]{}
\fi

\newcommand{\wss}[1]{\authornote{magenta}{SS}{#1}} 
\newcommand{\plt}[1]{\authornote{cyan}{TPLT}{#1}} %For explanation of the template
\newcommand{\an}[1]{\authornote{cyan}{Author}{#1}}

%% Common Parts

\newcommand{\progname}{SFWRENG 4G06} % PUT YOUR PROGRAM NAME HERE
\newcommand{\authname}{Team 9, Flow
\\ Hussain Muhammed
\\ Chengze Zhao
\\ Jeffrey Doan
\\ Kevin Zhu
\\ Ethan Patterson } % AUTHOR NAMES                   

\usepackage{hyperref}
    \hypersetup{colorlinks=true, linkcolor=blue, citecolor=blue, filecolor=blue,
                urlcolor=blue, unicode=false}
    \urlstyle{same}
                                


\setlength{\parindent}{0pt}
\setlength{\parskip}{8pt}

\begin{document}

\title{System Verification and Validation Plan for \progname{}}
\author{\authname}
\date{\today}

\maketitle

\pagenumbering{roman}

\section*{Revision History}

\begin{tabularx}{\textwidth}{p{3cm}p{2cm}X}
\toprule {\bf Date} & {\bf Version} & {\bf Notes}\\
\midrule
Date 1 & 1.0 & Notes\\
Date 2 & 1.1 & Notes\\
\bottomrule
\end{tabularx}

~\\
\wss{The intention of the VnV plan is to increase confidence in the software.
However, this does not mean listing every verification and validation technique
that has ever been devised.  The VnV plan should also be a \textbf{feasible}
plan. Execution of the plan should be possible with the time and team available.
If the full plan cannot be completed during the time available, it can either be
modified to ``fake it'', or a better solution is to add a section describing
what work has been completed and what work is still planned for the future.}

\wss{The VnV plan is typically started after the requirements stage, but before
the design stage.  This means that the sections related to unit testing cannot
initially be completed.  The sections will be filled in after the design stage
is complete.  the final version of the VnV plan should have all sections filled
in.}

\newpage

\tableofcontents

\listoftables
\wss{Remove this section if it isn't needed}

\listoffigures
\wss{Remove this section if it isn't needed}

\newpage

\section{Symbols, Abbreviations, and Acronyms}

\renewcommand{\arraystretch}{1.2}
\begin{tabular}{l l}
  \toprule
  \textbf{symbol} & \textbf{description}\\
  \midrule
  T & Test\\
  \bottomrule
\end{tabular}\\

\wss{symbols, abbreviations, or acronyms --- you can simply reference the SRS
  \citep{SRS} tables, if appropriate}

\wss{Remove this section if it isn't needed}

\newpage

\pagenumbering{arabic}

This document ... \wss{provide an introductory blurb and roadmap of the
  Verification and Validation plan}

\section{General Information}

\subsection{Summary}

\wss{Say what software is being tested.  Give its name and a brief overview of
  its general functions.}

\subsection{Objectives}

\wss{State what is intended to be accomplished.  The objective will be around
  the qualities that are most important for your project.  You might have
  something like: ``build confidence in the software correctness,''
  ``demonstrate adequate usability.'' etc.  You won't list all of the qualities,
  just those that are most important.}

\wss{You should also list the objectives that are out of scope.  You don't have
the resources to do everything, so what will you be leaving out.  For instance,
if you are not going to verify the quality of usability, state this.  It is also
worthwhile to justify why the objectives are left out.}

\wss{The objectives are important because they highlight that you are aware of
limitations in your resources for verification and validation.  You can't do everything,
so what are you going to prioritize?  As an example, if your system depends on an
external library, you can explicitly state that you will assume that external library
has already been verified by its implementation team.}

\subsection{Extras}

\wss{Summarize the extras (if any) that were tackled by this project.  Extras
can include usability testing, code walkthroughs, user documentation, formal
proof, GenderMag personas, Design Thinking, etc.  Extras should have already
been approved by the course instructor as included in your problem statement.
You can use a pull request to update your extras (in TeamComposition.csv or
Repos.csv) if your plan changes as a result of the VnV planning exercise.}

\subsection{Relevant Documentation}

\wss{Reference relevant documentation.  This will definitely include your SRS
  and your other project documents (design documents, like MG, MIS, etc).  You
  can include these even before they are written, since by the time the project
  is done, they will be written.  You can create BibTeX entries for your
  documents and within those entries include a hyperlink to the documents.}

\citet{SRS}

\wss{Don't just list the other documents.  You should explain why they are relevant and
how they relate to your VnV efforts.}

\section{Plan}

This section will go over the Verification and Validation Team members and the techniques
and methodology required for a VnV Plan's success.

\subsection{Verification and Validation Team}

\begin{enumerate}
    \item \textbf{Ethan} --- Team Lead / Test Manager
    \begin{itemize}
        \item Develop test plan, testing strategies, and test timelines
    \end{itemize}

    \item \textbf{Hussain} --- Test Analyst
    \begin{itemize}
        \item Create and review detailed tests for functional, non-functional, and edge cases required for the project
    \end{itemize}

    \item \textbf{Oliver} --- Quality Assurance Tester
    \begin{itemize}
        \item Conduct tests to verify that all features work as expected according to the SRS
    \end{itemize}

    \item \textbf{Jeffrey} --- Automation Tester
    \begin{itemize}
        \item Design, develop, and maintain most of the automated test scripts used for repetitive and regression testing
    \end{itemize}

    \item \textbf{Kevin} --- Document Specialist
    \begin{itemize}
        \item Write and maintain most of the documentation, referring to any test plans, test cases, reports, or defect logs
    \end{itemize}
\end{enumerate}


\subsection{SRS Verification}

\begin{enumerate}
\item Internal Review and Cross-Check
\begin{enumerate}
\item This is the first level of verification for the SRS. The team will conduct a review as a whole, reviewing each member’s contribution and giving feedback. Additionally, this will help members confirm consistency between each individually worked on part.
\item Moreover, the team will also be given the opportunity to cross-check with the requirements listed in any previous documents to ensure there is consistency between documentation.
\end{enumerate}

\item Peer Review
\begin{enumerate}
\item This verification process will allow our team to seek assistance from a peer capstone team. This portion will allow another team to verify that the SRS is filled to completion and is error-free, ready for validation and development.
\item Additionally, this gives our team a chance to review another capstone’s SRS, which will allow us to cross-reference with our SRS, verifying that both documents adhere to the Myers SRS convention and include every necessary portion.
\end{enumerate}

\item Check List Review
\begin{enumerate}
\item This level of verification will have the team review and cross-reference the SRS checklist once again after the deliverable has been completed.
\item This is an extensive SRS checklist provided by our course instructor and is a professional way to ensure that our SRS contains everything required of us.
\item Going over this as a team once the document has been completed allows us to verify that all members adhered to the checklist when creating their portion of the SRS document.
\end{enumerate}
\end{enumerate}

\subsection{Design Verification}

\begin{enumerate}
\item Purpose:
\begin{enumerate}
\item We will use this to ensure that the design of our application meets the specifications and user needs outlined in the SRS.
\end{enumerate}

\item Design Verification Scope:
\begin{enumerate}
\item This plan will cover the verification of the following components:
\begin{enumerate}
\item User Interface (UI): Verify that the UI is intuitive and is minimized to accommodate a more keyboard-friendly environment, which would not require any mouse involvement.
\item Diagram Creation: Verify that the diagram creation feature works using only a keyboard and no mouse/trackpad. Ensure that diagrams are created with the same freedom and range as writing on a piece of paper or using a mouse.
\item Accessibility and Usability: Verify that users can efficiently use the application without the need for other external tools aside from a keyboard.
\item Performance: Verify that the system performs well even when generating multiple objects or diagrams.
\end{enumerate}
\end{enumerate}

\item Verification Objectives:
\begin{enumerate}
\item Comply with Functional Requirements: Verify that the design supports all key features and complies with the functional requirements listed in all past documents, such as the SRS.
\item Verify Performance: Verify that the application can handle multiple diagrams and object elements.
\end{enumerate}

\item Design Verification Methodology:
\begin{enumerate}
\item Verification through Design Reviews:
\begin{enumerate}
\item Internal Design Review:
\begin{enumerate}
\item The development team will conduct an internal review of the application to ensure that we are complying with all the functional requirements specified in the SRS and other related documents.
\item The team will also verify that the application is user-friendly by using past knowledge on product usability and referencing existing note-taking applications.
\end{enumerate}
\item Design Walkthrough
\begin{enumerate}
\item The development team will conduct walkthroughs with other team members who may not have been involved in specific feature development. Additionally, peers and friends may be used to conduct a proper walkthrough session.
\item During the walkthrough, ensure that users are able to navigate the application and create diagrams using the keyboard as intended.
\end{enumerate}
\end{enumerate}

\item Functional Testing:
\begin{enumerate}
\item Test Case Development:
\begin{enumerate}
\item Develop test cases that focus on verifying key design features, such as keyboard navigation or the creation of diagrams or objects using shortcuts.
\end{enumerate}
\end{enumerate}

\item Performance Testing:
\begin{enumerate}
\item Load Testing:
\begin{enumerate}
\item Perform tests that will create strain on a system through the creation of multiple objects or diagrams that are being manipulated. Verify that the application is able to perform well under this load with the system requirements specified in the SRS.
\end{enumerate}
\item Speed Test:
\begin{enumerate}
\item Verify that all interactions (i.e., diagram/object creation and manipulation) are quick and responsive with minimal lag or delay.
\end{enumerate}
\end{enumerate}
\end{enumerate}

\item Design Verification Criteria:
\begin{enumerate}
\item Usability Criteria:
\begin{enumerate}
\item Users can create diagrams using only keyboard shortcuts.
\item Users can navigate the user interface without relying on a mouse.
\item No critical keyboard shortcut conflict, and functionality across all supported platforms.
\end{enumerate}
\item Functional Criteria:
\begin{enumerate}
\item All diagram creation features are functional. This includes adding shapes, manipulating elements, connecting existing objects, and deleting components, without the use of a mouse.
\end{enumerate}
\item Compliance with Requirements:
\begin{enumerate}
\item The design must meet all requirements outlined in the Software Requirements Specification (SRS).
\end{enumerate}
\end{enumerate}
\end{enumerate}

\subsection{Verification and Validation Plan Verification}

\begin{enumerate}
\item Purpose:
\begin{enumerate}
\item To verify that the VnV Plan is complete, correct, and meets the standards for verifying and validating our application.
\end{enumerate}

\item VnV Plan Verification Scope:
\begin{enumerate}
\item Completeness: Verify that all necessary sections and objectives are included.
\item Clarity and Precision: Verify that the plan is clearly written and unambiguous.
\item Feasibility: Verify that all techniques and processes described to be used in the verification and validation process are practical and achievable.
\end{enumerate}

\item VnV Plan Verification Objectives:
\begin{enumerate}
\item Verify that the VnV Plan is comprehensive and covers all necessary portions required of a complete verification and validation plan.
\item Verify that the techniques outlined for verification are appropriate for validating the functional requirements listed in the SRS.
\item Verify that the plan can realistically be executed and will assist and result in the resolution of any defects found in our application.
\end{enumerate}

\item Verification Techniques:
\begin{enumerate}
\item Internal Verification Plan Review:
\begin{enumerate}
\item The entire capstone team will collectively review the document once finished. This is to verify that all members have completed each task to a standard that the team can agree on.
\item The team will also be utilizing the checklist provided by the course instructor. This will help in verifying that the VnV Plan is completed correctly and incorporates all the main features and sections required of a proper VnV Plan.
\end{enumerate}

\item Peer Review:
\begin{enumerate}
\item The VnV Plan will also be reviewed by another project team that has also been tasked to work on a similar VnV Plan for their capstone. This will help verify that our team has completed the document properly by comparing sections and content with another completed VnV document in the same cohort.
\item Moreover, while our team also reviews the partnered team’s VnV Plan, we can cross-reference ours with theirs to verify that both teams have incorporated all the necessary sections and techniques required for this deliverable.
\end{enumerate}

\item Mutation Testing:
\begin{enumerate}
\item Our team will incorporate mutation testing to verify that our VnV Plan is working as intended. In order to do this, we will be incorporating small changes or “mutants” and testing to see if the verification and validation process noted would catch them. If the plan catches these changes, it will help confirm that the VnV plan is comprehensive and effective.
\item Example of possible mutations:
\begin{enumerate}
\item Mutation 1: Remove a key verification step (e.g., Usability Verification) and see if the VnV process identifies this omission.
\item Mutation 2: Add an unnecessary or incorrect verification technique (e.g., testing mouse object creation) and see if the VnV process can identify that the new technique is unsuitable or conflicts with other requirements.
\end{enumerate}
\end{enumerate}
\end{enumerate}
\end{enumerate}

\subsection{Implementation Verification}

\wss{You should at least point to the tests listed in this document and the unit
  testing plan.}

\wss{In this section you would also give any details of any plans for static
  verification of the implementation.  Potential techniques include code
  walkthroughs, code inspection, static analyzers, etc.}

\wss{The final class presentation in CAS 741 could be used as a code
walkthrough.  There is also a possibility of using the final presentation (in
CAS741) for a partial usability survey.}

Firstly, the tests outlined in the System Tests (and eventually the Unit Test
Description) section will be used for verification. The purpose of these tests
are to ensure the system behaves as expected given the specification, which
naturally assists in verification of the implemented system. It is expected
that not every individual unit test will be specified in the unit testing plan
and devs should not withhold from writing a useful unit test just because it is
not specified in this document. An example is writing a new unit test to cover
the expected functionality of a newly discovered bug. Devs are expected to
write tests that cover their code at the specified coverage rate of
\textit{TEST\_COVERAGE\_RATE}.

Beyond the specified test cases, the product will be test driven by members
of the team for taking notes during lectures. All members of the team are
expected to know the general expected functionality of the system, so having
the team test the product as a user would greatly benefits verifying the
product. Additionally, this is a feasible plan given the time frame and
resources of the project. All team members attend classes on a daily basis and
have the opportunity to use Flow during their lecture time.

While manual and automated dynamic tests are essential for verification of the
product, non-dynamic testing approaches can also be extremely effective and be
used throughout the development process. Typical static testing approaches
include static code analyzers, code walkthroughs, and code review. These three
approaches will be used by the team as part of the verification process.

A linting tool will be selected and each team member will be required to use it
when developing locally. To enforce its use, a new workflow step will be added
to the GitHub actions that will run the linting tool on an opened pull request.

Code review is expected to conducted on every pull request before it can be
approved and merged. To keep the plan realistic, only one review will be
required per PR, but more are encouraged if possible. In the case of looming
hard deadlines that must be met in a timely manner, the team may come to a
collective decision to skip a rigorous review process on a PR if they deem the
change to be small enough. However, the GitHub actions pipeline is still
expected to be run and pass before merging to main.

Lastly, synchronous code walkthrough may be conducted per a dev's request. This
is expected to supplement a code review if a reviewer requests it. It may also
be proposed by the code owner if they feel knowledge sharing and familiarity of
the code/architecture will be beneficial for the team. Examples include a new
module that will be reused by other devs for other parts of the codebase or
some critical functionality that would benefit from group review/verification.
Since there is more overhead with this approach due to there needing to be a
scheduled time, this is more of an option that is open to the team rather than
a strictly enforced testing plan approach.

\subsection{Automated Testing and Verification Tools}

\wss{What tools are you using for automated testing.  Likely a unit testing
  framework and maybe a profiling tool, like ValGrind.  Other possible tools
  include a static analyzer, make, continuous integration tools, test coverage
  tools, etc.  Explain your plans for summarizing code coverage metrics.
  Linters are another important class of tools.  For the programming language
  you select, you should look at the available linters.  There may also be tools
  that verify that coding standards have been respected, like flake9 for
  Python.}

\wss{If you have already done this in the development plan, you can point to
that document.}

\wss{The details of this section will likely evolve as you get closer to the
  implementation.}

Firstly, the planned tech stack for the project is an Electron app with a
React frontend using TypeScript.

For automated unit testing, we will use the \href{https://jestjs.io/}{Jest}
framework, which is a JavaScript testing framework that can be used with
TypeScript and React. For more end-to-end feature testing, we will use
\href{https://playwright.dev/}{Playwright}, which is built for end-to-end
testing web apps, but also has support for Electron apps.

For profiling and debugging, since Electron is built on top of Chromium, we can
use Chrome DevTools to measure the performance and memory usage of our app.

For static analysis and linting we will use ESLint with the TypeScript and
React plugins, which will enforce coding rules and find anti-patterns.
Additionally, Prettier will be used for formatting. The combination of these
tools will lead to a higher quality codebase with consistent styling.

Pnpm will be our build system. We can define pnpm scripts for running,
building, linting, and testing our app.

GitHub Actions will be used for continuous integration. The pipeline will
compile the app, run the automated tests, linting step, and formatting step on
each PR and main merge.

Jest has a built in feature for measuring and summarizing code test coverage,
simply by passing the `--coverage` flag. This report will then be integrated
into the CI pipeline through Coveralls, which is free for public GitHub repos
and has built in GitHub integration.

\subsection{Software Validation}

\wss{If there is any external data that can be used for validation, you should
  point to it here.  If there are no plans for validation, you should state that
  here.}

\wss{You might want to use review sessions with the stakeholder to check that
the requirements document captures the right requirements.  Maybe task based
inspection?}

\wss{For those capstone teams with an external supervisor, the Rev 0 demo should
be used as an opportunity to validate the requirements.  You should plan on
demonstrating your project to your supervisor shortly after the scheduled Rev 0 demo.
The feedback from your supervisor will be very useful for improving your project.}

\wss{For teams without an external supervisor, user testing can serve the same purpose
as a Rev 0 demo for the supervisor.}

\wss{This section might reference back to the SRS verification section.}

There are no plans for formal validation of the project as there is no formal
individual stakeholder/supervisor for the project.

As a team we are eliciting requirements amongst ourselves and we are among the
projected users of Flow, so we expect to work as a group throughout the
development process to tweak existing requirements and add missing requirements
as they are discovered.

\section{System Tests}

\wss{There should be text between all headings, even if it is just a roadmap of
the contents of the subsections.}

\subsection{Tests for Functional Requirements}

\wss{Subsets of the tests may be in related, so this section is divided into
  different areas.  If there are no identifiable subsets for the tests, this
  level of document structure can be removed.}

\wss{Include a blurb here to explain why the subsections below
  cover the requirements.  References to the SRS would be good here.}

\subsubsection{Area of Testing1}

\wss{It would be nice to have a blurb here to explain why the subsections below
  cover the requirements.  References to the SRS would be good here.  If a section
  covers tests for input constraints, you should reference the data constraints
  table in the SRS.}

\paragraph{Title for Test}

\begin{enumerate}

\item{test-id1\\}

Control: Manual versus Automatic

Initial State:

Input:

Output: \wss{The expected result for the given inputs.  Output is not how you
are going to return the results of the test.  The output is the expected
result.}

Test Case Derivation: \wss{Justify the expected value given in the Output field}

How test will be performed:

\item{test-id2\\}

Control: Manual versus Automatic

Initial State:

Input:

Output: \wss{The expected result for the given inputs}

Test Case Derivation: \wss{Justify the expected value given in the Output field}

How test will be performed:

\end{enumerate}

\subsubsection{Area of Testing2}

...

\subsection{Tests for Nonfunctional Requirements}

\wss{The nonfunctional requirements for accuracy will likely just reference the
  appropriate functional tests from above.  The test cases should mention
  reporting the relative error for these tests.  Not all projects will
  necessarily have nonfunctional requirements related to accuracy.}

\wss{For some nonfunctional tests, you won't be setting a target threshold for
passing the test, but rather describing the experiment you will do to measure
the quality for different inputs.  For instance, you could measure speed versus
the problem size.  The output of the test isn't pass/fail, but rather a summary
table or graph.}

\wss{Tests related to usability could include conducting a usability test and
  survey.  The survey will be in the Appendix.}

\wss{Static tests, review, inspections, and walkthroughs, will not follow the
format for the tests given below.}

\wss{If you introduce static tests in your plan, you need to provide details.
How will they be done?  In cases like code (or document) walkthroughs, who will
be involved? Be specific.}

\subsubsection{Area of Testing1}

\paragraph{Title for Test}

\begin{enumerate}

\item{test-id1\\}

Type: Functional, Dynamic, Manual, Static etc.

Initial State:

Input/Condition:

Output/Result:

How test will be performed:

\item{test-id2\\}

Type: Functional, Dynamic, Manual, Static etc.

Initial State:

Input:

Output:

How test will be performed:

\end{enumerate}

\subsubsection{Area of Testing2}

...

\subsection{Traceability Between Test Cases and Requirements}

\wss{Provide a table that shows which test cases are supporting which
  requirements.}

\section{Unit Test Description}

\wss{This section should not be filled in until after the MIS (detailed design
  document) has been completed.}

\wss{Reference your MIS (detailed design document) and explain your overall
philosophy for test case selection.}

\wss{To save space and time, it may be an option to provide less detail in this section.
For the unit tests you can potentially layout your testing strategy here.  That is, you
can explain how tests will be selected for each module.  For instance, your test building
approach could be test cases for each access program, including one test for normal behaviour
and as many tests as needed for edge cases.  Rather than create the details of the input
and output here, you could point to the unit testing code.  For this to work, you code
needs to be well-documented, with meaningful names for all of the tests.}

\subsection{Unit Testing Scope}

\wss{What modules are outside of the scope.  If there are modules that are
  developed by someone else, then you would say here if you aren't planning on
  verifying them.  There may also be modules that are part of your software, but
  have a lower priority for verification than others.  If this is the case,
  explain your rationale for the ranking of module importance.}

\subsection{Tests for Functional Requirements}

\wss{Most of the verification will be through automated unit testing.  If
  appropriate specific modules can be verified by a non-testing based
  technique.  That can also be documented in this section.}

\subsubsection{Module 1}

\wss{Include a blurb here to explain why the subsections below cover the module.
  References to the MIS would be good.  You will want tests from a black box
  perspective and from a white box perspective.  Explain to the reader how the
  tests were selected.}

\begin{enumerate}

\item{test-id1\\}

Type: \wss{Functional, Dynamic, Manual, Automatic, Static etc. Most will
  be automatic}

Initial State:

Input:

Output: \wss{The expected result for the given inputs}

Test Case Derivation: \wss{Justify the expected value given in the Output field}

How test will be performed:

\item{test-id2\\}

Type: \wss{Functional, Dynamic, Manual, Automatic, Static etc. Most will
  be automatic}

Initial State:

Input:

Output: \wss{The expected result for the given inputs}

Test Case Derivation: \wss{Justify the expected value given in the Output field}

How test will be performed:

\item{...\\}

\end{enumerate}

\subsubsection{Module 2}

...

\subsection{Tests for Nonfunctional Requirements}

\wss{If there is a module that needs to be independently assessed for
  performance, those test cases can go here.  In some projects, planning for
  nonfunctional tests of units will not be that relevant.}

\wss{These tests may involve collecting performance data from previously
  mentioned functional tests.}

\subsubsection{Module ?}

\begin{enumerate}

\item{test-id1\\}

Type: \wss{Functional, Dynamic, Manual, Automatic, Static etc. Most will
  be automatic}

Initial State:

Input/Condition:

Output/Result:

How test will be performed:

\item{test-id2\\}

Type: Functional, Dynamic, Manual, Static etc.

Initial State:

Input:

Output:

How test will be performed:

\end{enumerate}

\subsubsection{Module ?}

...

\subsection{Traceability Between Test Cases and Modules}

\wss{Provide evidence that all of the modules have been considered.}

\bibliographystyle{plainnat}

\bibliography{../../refs/References}

\newpage

\section{Appendix}

This is where you can place additional information.

\subsection{Symbolic Parameters}

The definition of the test cases will call for SYMBOLIC\_CONSTANTS.
Their values are defined in this section for easy maintenance.\\

\begin{description}
  \item[TEST\_COVERAGE\_RATE] - 70\%. Ideally this value would be 100\%, but 70\% was chosen as a realistic balance between time constraints and verification quality.
\end{description}

\subsection{Usability Survey Questions?}

\wss{This is a section that would be appropriate for some projects.}

\newpage{}
\section*{Appendix --- Reflection}

\wss{This section is not required for CAS 741}

The information in this section will be used to evaluate the team members on the
graduate attribute of Lifelong Learning.

The purpose of reflection questions is to give you a chance to assess your own
learning and that of your group as a whole, and to find ways to improve in the
future. Reflection is an important part of the learning process.  Reflection is
also an essential component of a successful software development process.  

Reflections are most interesting and useful when they're honest, even if the
stories they tell are imperfect. You will be marked based on your depth of
thought and analysis, and not based on the content of the reflections
themselves. Thus, for full marks we encourage you to answer openly and honestly
and to avoid simply writing ``what you think the evaluator wants to hear.''

Please answer the following questions.  Some questions can be answered on the
team level, but where appropriate, each team member should write their own
response:


\begin{enumerate}
  \item What went well while writing this deliverable?
  \item What pain points did you experience during this deliverable, and how
    did you resolve them?
  \item What knowledge and skills will the team collectively need to acquire to
  successfully complete the verification and validation of your project?
  Examples of possible knowledge and skills include dynamic testing knowledge,
  static testing knowledge, specific tool usage, Valgrind etc.  You should look to
  identify at least one item for each team member.
  \item For each of the knowledge areas and skills identified in the previous
  question, what are at least two approaches to acquiring the knowledge or
  mastering the skill?  Of the identified approaches, which will each team
  member pursue, and why did they make this choice?
\end{enumerate}

\end{document}
