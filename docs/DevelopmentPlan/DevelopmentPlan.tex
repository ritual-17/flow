\documentclass{article}

\usepackage{booktabs}
\usepackage{tabularx}

\title{Development Plan\\\progname}

\author{\authname}

\date{}

%% Comments

\usepackage{color}

\newif\ifcomments\commentstrue %displays comments
%\newif\ifcomments\commentsfalse %so that comments do not display

\ifcomments
\newcommand{\authornote}[3]{\textcolor{#1}{[#3 ---#2]}}
\newcommand{\todo}[1]{\textcolor{red}{[TODO: #1]}}
\else
\newcommand{\authornote}[3]{}
\newcommand{\todo}[1]{}
\fi

\newcommand{\wss}[1]{\authornote{magenta}{SS}{#1}} 
\newcommand{\plt}[1]{\authornote{cyan}{TPLT}{#1}} %For explanation of the template
\newcommand{\an}[1]{\authornote{cyan}{Author}{#1}}

%% Common Parts

\newcommand{\progname}{SFWRENG 4G06} % PUT YOUR PROGRAM NAME HERE
\newcommand{\authname}{Team 9, Flow
\\ Hussain Muhammed
\\ Chengze Zhao
\\ Jeffrey Doan
\\ Kevin Zhu
\\ Ethan Patterson } % AUTHOR NAMES                   

\usepackage{hyperref}
    \hypersetup{colorlinks=true, linkcolor=blue, citecolor=blue, filecolor=blue,
                urlcolor=blue, unicode=false}
    \urlstyle{same}
                                


\begin{document}

\maketitle

\begin{table}[hp]
\caption{Revision History} \label{TblRevisionHistory}
\begin{tabularx}{\textwidth}{llX}
\toprule
\textbf{Date} & \textbf{Developer(s)} & \textbf{Change}\\
\midrule
Date1 & Name(s) & Description of changes\\
Date2 & Name(s) & Description of changes\\
... & ... & ...\\
\bottomrule
\end{tabularx}
\end{table}

\newpage{}

\wss{Put your introductory blurb here.  Often the blurb is a brief roadmap of
what is contained in the report.}

\wss{Additional information on the development plan can be found in the
\href{https://gitlab.cas.mcmaster.ca/courses/capstone/-/blob/main/Lectures/L02b_POCAndDevPlan/POCAndDevPlan.pdf?ref_type=heads}
{lecture slides}.}

\section{Confidential Information?}

\wss{State whether your project has confidential information from industry, or
not.  If there is confidential information, point to the agreement you have in
place.}

\wss{For most teams this section will just state that there is no confidential
information to protect.}
\section{IP to Protect}

\wss{State whether there is IP to protect.  If there is, point to the agreement.
All students who are working on a project that requires an IP agreement are also
required to sign the ``Intellectual Property Guide Acknowledgement.''}

\section{Copyright License}

\wss{What copyright license is your team adopting.  Point to the license in your
repo.}

\section{Team Meeting Plan}

\wss{How often will you meet? where?}

\wss{If the meeting is a physical location (not virtual), out of an abundance of
caution for safety reasons you shouldn't put the location online}

\wss{How often will you meet with your industry advisor?  when?  where?}

\wss{Will meetings be virtual?  At least some meetings should likely be
in-person.}

\wss{How will the meetings be structured?  There should be a chair for all meetings.  There should be an agenda for all meetings.}

\section{Team Communication Plan}

\wss{Issues on GitHub should be part of your communication plan.}

\section{Team Member Roles}

\wss{You should identify the types of roles you anticipate, like notetaker,
leader, meeting chair, reviewer.  Assigning specific people to those roles is
not necessary at this stage.  In a student team the role of the individuals will
likely change throughout the year.}

\section{Workflow Plan}

\begin{itemize}
	\item How will you be using git, including branches, pull request, etc.?
	\item How will you be managing issues, including template issues, issue
	classification, etc.?
  \item Use of CI/CD
\end{itemize}

\section{Project Decomposition and Scheduling}

\begin{itemize}
  \item How will you be using GitHub projects?
  \item Include a link to your GitHub project
\end{itemize}

\wss{How will the project be scheduled?  This is the big picture schedule, not
details. You will need to reproduce information that is in the course outline
for deadlines.}

\section{Proof of Concept Demonstration Plan}

What is the main risk, or risks, for the success of your project?  What will you
demonstrate during your proof of concept demonstration to convince yourself that
you will be able to overcome this risk?

\section{Expected Technology}

\wss{What programming language or languages do you expect to use?  What external
libraries?  What frameworks?  What technologies.  Are there major components of
the implementation that you expect you will implement, despite the existence of
libraries that provide the required functionality.  For projects with machine
learning, will you use pre-trained models, or be training your own model?  }

\wss{The implementation decisions can, and likely will, change over the course
of the project.  The initial documentation should be written in an abstract way;
it should be agnostic of the implementation choices, unless the implementation
choices are project constraints.  However, recording our initial thoughts on
implementation helps understand the challenge level and feasibility of a
project.  It may also help with early identification of areas where project
members will need to augment their training.}

Topics to discuss include the following:

\begin{itemize}
\item Specific programming language
\item Specific libraries
\item Pre-trained models
\item Specific linter tool (if appropriate)
\item Specific unit testing framework
\item Investigation of code coverage measuring tools
\item Specific plans for Continuous Integration (CI), or an explanation that CI
  is not being done
\item Specific performance measuring tools (like Valgrind), if
  appropriate
\item Tools you will likely be using?
\end{itemize}

\wss{git, GitHub and GitHub projects should be part of your technology.}

\section{Coding Standard}

\wss{What coding standard will you adopt?}

\newpage{}

\section*{Appendix --- Reflection}

\wss{Not required for CAS 741}

The purpose of reflection questions is to give you a chance to assess your own
learning and that of your group as a whole, and to find ways to improve in the
future. Reflection is an important part of the learning process.  Reflection is
also an essential component of a successful software development process.  

Reflections are most interesting and useful when they're honest, even if the
stories they tell are imperfect. You will be marked based on your depth of
thought and analysis, and not based on the content of the reflections
themselves. Thus, for full marks we encourage you to answer openly and honestly
and to avoid simply writing ``what you think the evaluator wants to hear.''

Please answer the following questions.  Some questions can be answered on the
team level, but where appropriate, each team member should write their own
response:


\begin{enumerate}
    \item Why is it important to create a development plan prior to starting the
    project?
    \item In your opinion, what are the advantages and disadvantages of using
    CI/CD?
    \item What disagreements did your group have in this deliverable, if any,
    and how did you resolve them?
\end{enumerate}

\newpage{}

\section*{Appendix --- Team Charter}

\wss{borrows from
\href{https://engineering.up.edu/industry_partnerships/files/team-charter.pdf}
{University of Portland Team Charter}}

\subsection*{External Goals}

\wss{What are your team's external goals for this project? These are not the
goals related to the functionality or quality fo the project.  These are the
goals on what the team wishes to achieve with the project.  Potential goals are
to win a prize at the Capstone EXPO, or to have something to talk about in
interviews, or to get an A+, etc.}

Our team has many external goals for this capstone. It would go without saying that we would like to achieve a high grade for our work in this course; however, there are many other factors as to why we are trying to succeed with this project. First off, as engineering students, we understand the hardships of having to take notes in class and not being able to smoothly incorporate diagrams. As we continue our careers and education, we would like to have an application that allows us to execute exactly what we wished we had in our previous undergraduate years.
Additionally, as some of us are graduating and moving on to new career opportunities, it would be a great benefit to us if we had a successful project to give us leverage when applying for new graduate positions. 


\subsection*{Attendance}

\subsubsection*{Expectations}

\wss{What are your team's expectations regarding meeting attendance (being on
time, leaving early, missing meetings, etc.)?}

Our team expects all members to show up to every scheduled meeting unless they have an acceptable reason for not being able to attend. The specifics on what constitutes an acceptable excuse will be discussed in the next section.

\subsubsection*{Acceptable Excuse}

\wss{What constitutes an acceptable excuse for missing a meeting or a deadline?}

For in-person meetings, the team understands that some members may have issues commuting and will not be able to physically attend. In this case, members will be allowed to attend online; however, this should not be a regular occurrence, and members should make an effort to attend in person. Moreover, the team believes that there is no acceptable excuse for missing a deadline other than for emergencies. This is further elaborated in the coming sections. 

\wss{What types of excuses will not be considered acceptable?}
\\Excuses that involve prioritizing extracurriculars or recreational activities will not be tolerated.
Excuses that are a result of one's own poor time management or lack thereof will also not be tolerated.
We all understand the busy schedule of an engineering student; however, it is expected that each member manages their time wisely and puts the same amount of effort or more into this capstone as they would into their other courses.



\subsubsection*{In Case of Emergency}

\wss{What process will team members follow if they have an emergency and cannot
attend a team meeting or complete their individual work promised for a team
deliverable?}

Group members may be met with unexpected issues, which will prevent them from attending a meeting or potentially miss a deadline. For meetings, if a member is not able to attend due to an emergency, they are expected to communicate that and follow up with the team on what they missed. They are expected to also review the meeting minutes before reaching out to the team about what they missed. Moreover, if a member is met with an emergency that prevents them from finishing work for a team deliverable, the team will be understanding and assist to the best of their capabilities. Depending on the situation, the member should follow the proper McMaster procedure for these emergencies and reach out to the TA/Instructor to discuss the potential use of an MSAF.

\subsection*{Accountability and Teamwork}

\subsubsection*{Quality} 

\wss{What are your team's expectations regarding the quality
of team members' preparation for team meetings and the quality of the
deliverables that members bring to the team?}

Each team member is expected to produce high-quality work that all other members would approve of. All members understand that the work they produce affects other members and should be considerate of this fact when working on their assigned portion of the deliverables. Additionally, all members agree that meetings should be conducted as smoothly and quickly as possible. To ensure this, each member understands that they are required to properly prepare for the scheduled team meetings.

\subsubsection*{Attitude}

\wss{What are your team's expectations regarding team members' ideas,
interactions with the team, cooperation, attitudes, and anything else regarding
team member contributions?  Do you want to introduce a code of conduct?  Do you
want a conflict resolution plan?  Can adopt existing codes of conduct.}

Each team member is expected to treat all other members with respect and professionalism. Members should feel comfortable sharing ideas and interacting with each other. Members should come into meetings with a positive attitude and a willingness to actively cooperate. Members should also be considerate of each other’s time and should stay on track during meetings and work periods.\\

Additionally, we will be incorporating the already exisitng "Contributor Covenant" as our code of conduct. The exact code of conduct can be found \href{https://www.contributor-covenant.org/version/2/1/code_of_conduct/}{here}. \\

In terms of conflict resolution plan, we decdied to implement a short and decisive process: \\
\begin{enumerate}
  \item Direct Communication
  \begin{itemize}
    \item Members directly involved in the conflict should try to resolve the issue on their own first. 
  \end{itemize}
  \item Team Discussion
  \begin{itemize}
    \item If the issue remains unresolved, the team may hold a brief meeting to resolve said conflict. The meeting will be used to dicuss the concerns and work towards a solution approved by the majority of the group.
  \end{itemize}
  \item Seek Support from Advisor / Instructor
  \begin{itemize}
    \item If a resolution still cannot be reached after the previous steps, the team should schedule a meeting with the TA or Instructor and ask for guidance on how to resolve the conflict. Note, this will only be used as a last resort after exhausting all options.
  \end{itemize}
\end{enumerate}

\subsubsection*{Stay on Track}

\wss{What methods will be used to keep the team on track? How will your team
ensure that members contribute as expected to the team and that the team
performs as expected? How will your team reward members who do well and manage
members whose performance is below expectations?  What are the consequences for
someone not contributing their fair share?}

\wss{You may wish to use the project management metrics collected for the TA and
instructor for this.}

\wss{You can set target metrics for attendance, commits, etc.  What are the
consequences if someone doesn't hit their targets?  Do they need to bring the
coffee to the next team meeting?  Does the team need to make an appointment with
their TA, or the instructor?  Are there incentives for reaching targets early?} \\

Keeping a team on track is very important when working on a long-term project, such as this capstone. To ensure that the team stays on track, we decided to consider implementing methods such as using project management tools, setting up regular update meetings, and defining clear roles and responsibilities at the beginning of each deliverable. The regular update meetings and assigned responsibilities will help ensure that members contribute their fair share and that the team is performing as expected. Each member will hold the others accountable for their share of the deliverables, which will allow our team to perform as expected. \\

It is understandable that some members may not feel motivated or may not be performing as expected, even with all the previous methods in place. To prevent this, we will be incorporating fun consequences, such as having these members bring hot chocolate, cookies, or other food items/treats to the next meeting. If the team member continues to underperform despite the above methods being in place, the other members will discuss potentially meeting with a TA or instructor to resolve this issue. Please note that this will only be used as a last resort after exhausting all other options. \\

Contrarily, if a member does exceptionally well or contributes more than their fair share, we’ve decided to reward these members by having the rest of the team split their cost at the next team event. For instance, the other members could contribute to the cost of that member’s next meal during a team lunch or dinner. \\

Additionally, if members complete their targets early, they won’t be assigned any more work and can focus on their other responsibilities. Incentives for finishing deliverables and tasks are viewed more as a team-building activity. These are discussed in the following section. \\


\subsubsection*{Team Building}

\wss{How will you build team cohesion (fun time, group rituals, etc.)? }

As engineering students, we understand that there are many stressors when working on a major project, such as a capstone. These stresses can have a big effect on productivity and overall work quality. To combat this issue, we’ve decided to partake in a group activity or outing for each major goal/milestone we accomplish. For instance, after a stressful deliverable, we’ll try to wind down by enjoying online games together. If we accomplish a tough goal together, we think it’s a good idea to celebrate with a lunch/dinner to acknowledge our hard work. By doing this, we hope to build team cohesion throughout this project.

\subsubsection*{Decision Making} 

\wss{How will you make decisions in your group? Consensus?  Vote? How will you
handle disagreements? }

We believe that decisions should be made as a group and not by one individual. From this, we decided that we should discuss with all members present when making decisions or handling disagreements. From that discussion, we will hold a vote where the majority rules. We all agree to handle these discussions/meetings professionally and with respect for each other. That way, we can hold a responsible and respected discussion when disagreements arise. 

\end{document}
