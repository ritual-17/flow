\documentclass{article}

\usepackage{booktabs}
\usepackage{tabularx}

\title{Development Plan\\\progname}

\author{\authname}

\date{}

%% Comments

\usepackage{color}

\newif\ifcomments\commentstrue %displays comments
%\newif\ifcomments\commentsfalse %so that comments do not display

\ifcomments
\newcommand{\authornote}[3]{\textcolor{#1}{[#3 ---#2]}}
\newcommand{\todo}[1]{\textcolor{red}{[TODO: #1]}}
\else
\newcommand{\authornote}[3]{}
\newcommand{\todo}[1]{}
\fi

\newcommand{\wss}[1]{\authornote{magenta}{SS}{#1}} 
\newcommand{\plt}[1]{\authornote{cyan}{TPLT}{#1}} %For explanation of the template
\newcommand{\an}[1]{\authornote{cyan}{Author}{#1}}

%% Common Parts

\newcommand{\progname}{SFWRENG 4G06} % PUT YOUR PROGRAM NAME HERE
\newcommand{\authname}{Team 9, Flow
\\ Hussain Muhammed
\\ Chengze Zhao
\\ Jeffrey Doan
\\ Kevin Zhu
\\ Ethan Patterson } % AUTHOR NAMES                   

\usepackage{hyperref}
    \hypersetup{colorlinks=true, linkcolor=blue, citecolor=blue, filecolor=blue,
                urlcolor=blue, unicode=false}
    \urlstyle{same}
                                


\begin{document}

\maketitle

\begin{table}[hp]
\caption{Revision History} \label{TblRevisionHistory}
\begin{tabularx}{\textwidth}{llX}
\toprule
\textbf{Date} & \textbf{Developer(s)} & \textbf{Change}\\
\midrule
Date1 & Name(s) & Description of changes\\
Date2 & Name(s) & Description of changes\\
... & ... & ...\\
\bottomrule
\end{tabularx}
\end{table}

\newpage{}

\wss{Put your introductory blurb here.  Often the blurb is a brief roadmap of
what is contained in the report.}

\wss{Additional information on the development plan can be found in the
\href{https://gitlab.cas.mcmaster.ca/courses/capstone/-/blob/main/Lectures/L02b_POCAndDevPlan/POCAndDevPlan.pdf?ref_type=heads}
{lecture slides}.}

\section{Confidential Information?}

The project does not use any confidential information. 

\wss{State whether your project has confidential information from industry, or
not.  If there is confidential information, point to the agreement you have in
place.}

\wss{For most teams this section will just state that there is no confidential
information to protect.}
\section{IP to Protect}

There is no IP to protect in this project. 

\wss{State whether there is IP to protect.  If there is, point to the agreement.
All students who are working on a project that requires an IP agreement are also
required to sign the ``Intellectual Property Guide Acknowledgement.''}

\section{Copyright License}

The group has agreed that the project should use a MIT License.

\wss{What copyright license is your team adopting.  Point to the license in your
repo.}

\section{Team Meeting Plan}

\wss{How often will you meet? where?}

The team will meet 1-3 times per week with one in person meeting reoccurring at the same time every week. The meeting will be at a previously agreed upon location that will be discussed privately. Alternatively the meeting may be done virtually. We will meet with industry advisers and  other experts on a scheduled basis. These meetings will be announced beforehand. 

\wss{If the meeting is a physical location (not virtual), out of an abundance of
caution for safety reasons you shouldn't put the location online}


\wss{How often will you meet with your industry advisor?  when?  where?}


\wss{Will meetings be virtual?  At least some meetings should likely be
in-person.}


\wss{How will the meetings be structured?  There should be a chair for all meetings.  There should be an agenda for all meetings.}

\section{Team Communication Plan}

\wss{Issues on GitHub should be part of your communication plan.}

The team will use a private discord server along with GitHub for communication. Team members are expected to check both of these daily for updates to meeting times or other milestones.  

\section{Team Member Roles}

We will need a dedicated Team Lead, Keynoter, Team Liaison, Reviewer, Meeting chair.
The team lead is in charge of scheduling meetings and other events that may be needed to complete the project. They Keynoter is in charge of taking notes on any meetings for later use. The Team Liaison is in charge of communicating with the Professor and other external helpers for the project. The reviewer is in charge of reviewing work to ensure it is of high standard. The meeting chair is in charge of managing meeting topics and making sure that the meeting stays productive and relevant. 

\wss{You should identify the types of roles you anticipate, like notetaker,
leader, meeting chair, reviewer.  Assigning specific people to those roles is
not necessary at this stage.  In a student team the role of the individuals will
likely change throughout the year.}

\section{Workflow Plan}

% \begin{itemize}
	
	% \item How will you be using git, including branches, pull request, etc.?
	% \item How will you be managing issues, including template issues, issue
	% classification, etc.?
  % \item Use of CI/CD
  
  The project will be using GitHub for version control and issue tracking. It will be setup with a main branch that is always stable and
  a dev branch shall be used for integration of new features. Each feature or fix will first be developed in a separate branch off of dev. 
  Once the feature is complete and tested, a pull request shall be created to merge the feature branch into dev. The pull request would be 
  reviewed by at least one other team member before being merged. For smaller changes, the dev branch may just be used directly. The dev 
  branch will be periodically merged into main after thorough testing to ensure stability. 
  \\
  \\
  The GitHub Issues will be used to track the various tasks and features that need to be completed, as well as any meetings that are held.
  Templates are to be used for each issue to ensure relevant information is included. Issues will also be labeled based on their type, 
  such as bug, feature, meeting, etc, where applicable. This will help in organizing and prioritizing the work that needs to be done. 
  Assigning of issues will be done where necessary to track who is responsible for what.
  \\
  \\
  GitHub Actions will be used for CI/CD mainly for linting and running tests, as well as updating the pdfs from the tex files. Whenever a 
  pull request is created, the CI/CD pipeline will automatically run the actions which would include linter and any tests that have been 
  written. This will help catch any problems early, making sure any code merged is the correct format. Testing will most likely be setup
  later on as the project progresses and more code is written.

  % \item Work in individual branches for each new issue/change. 
	% \item Merge with pull requests followed by reviews.
	% \item Use issues for work.
	% \item Use automated tests. 

% \end{itemize}

\section{Project Decomposition and Scheduling}

% \begin{itemize}
  % \item How will you be using GitHub projects?
  % \item Include a link to your GitHub project
  
  We will use GitHub projects in the Kanban board format to organize our issue items. To begin, the columns will be "To Do", "In Progress", 
  and "Done". As we work the the project, more columns may be added to better suit our workflow. Each issue will be added to the "To Do" column,
  and when a team member starts working on it, it will be moved to "In Progress". Once the issue is completed and reviewed, it will be moved to
  "Done". This will help us keep track of what needs to be done, what is currently being worked on, and what has been completed. The project 
  board can be found at \url{https://github.com/users/ritual-17/projects/1/views/2}.
  
  % \item GitHub Projects will be used to help organize issues and assign them to developers. 
  % \item Project will be done on GitHub at https://github.com/ritual-17/flow.
  % \item The Project is planned to be finished at the end February 

% \end{itemize}

% \wss{How will the project be scheduled?  This is the big picture schedule, not
% details. You will need to reproduce information that is in the course outline
% for deadlines.}

\section{Proof of Concept Demonstration Plan}

What is the main risk, or risks, for the success of your project?  What will you
demonstrate during your proof of concept demonstration to convince yourself that
you will be able to overcome this risk?

The main risks for the success of the project have to do with factors adding to the time it takes to take a note down, making it longer than
taking notes by other means. As the project involves making and rendering geometry in a GUI, there is a risk that the performance of the 
application may be too slow to be practical. Another risk is that users may also find the syntax of the application too cumbersome or 
unintuitive to use effectively. To mitigate these risks, the proof of concept demonstration will focus on showing that notes can be taken 
down quickly and easily using the application. It will involve having someone technically proficient testing writing different notes using our 
application and comparing the time it takes to write the same notes using handwriting. A successful demonstration will show that notes can be 
taken down in faster or comparable time to handwriting, while still being legible and well formatted.

\section{Expected Technology}

% \wss{What programming language or languages do you expect to use?  What external
% libraries?  What frameworks?  What technologies.  Are there major components of
% the implementation that you expect you will implement, despite the existence of
% libraries that provide the required functionality.  For projects with machine
% learning, will you use pre-trained models, or be training your own model?  }

% \wss{The implementation decisions can, and likely will, change over the course
% of the project.  The initial documentation should be written in an abstract way;
% it should be agnostic of the implementation choices, unless the implementation
% choices are project constraints.  However, recording our initial thoughts on
% implementation helps understand the challenge level and feasibility of a
% project.  It may also help with early identification of areas where project
% members will need to augment their training.}

% Topics to discuss include the following:

% \begin{itemize}
% \item Specific programming language : Electron \& TypeScript (Subject to Change)
% \item Specific libraries : TBD
% \item Pre-trained models : N/A
% \item Specific linter tool (if appropriate) : TBD
% \item Specific unit testing framework : TBD
% \item Investigation of code coverage measuring tools : TBD
% \item Specific plans for Continuous Integration (CI), or an explanation that CI
%   is not being done : Github and GitHub Projects
% \item Specific performance measuring tools (like Valgrind), if
%   appropriate : N/A
% \item Tools you will likely be using? : Discord (to communicate)
% \end{itemize}

% \wss{git, GitHub and GitHub projects should be part of your technology.}

The programming language we expect to use is Electron with TypeScript. These languages are chosen due to their ability to create cross-platform 
desktop applications with graphical user interfaces, which is essential for our note-taking application. Some libraries we may use include React 
for building the user interface components, and others for rendering shapes. GitHub and GitHub Projects will be used for version control and 
project management, respectively. Testing and linting are important aspects of our development process. Jest and ESLint are potential choices 
for unit testing and linting, respectively, but these decisions will be finalized as the project progresses. As previously mentioned, GitHub 
Actions will be used for CI/CD to automate testing and linting processes, as well as updating documentation.

\section{Coding Standard}

% \wss{What coding standard will you adopt?}

The Standard JavaScript Style Guide will be used as the coding standard for this project. This guide provides a simplistic and easy to follow
set of rules for writing clean and consistent code. It covers various aspects of coding style, including naming, variables, functions and more.
The standard style can also be integrated with ESLint, which may be used for linting the code to ensure it adheres to the coding standard.

\newpage{}

\section*{Appendix --- Reflection}

\wss{Not required for CAS 741}

The purpose of reflection questions is to give you a chance to assess your own
learning and that of your group as a whole, and to find ways to improve in the
future. Reflection is an important part of the learning process.  Reflection is
also an essential component of a successful software development process.  

Reflections are most interesting and useful when they're honest, even if the
stories they tell are imperfect. You will be marked based on your depth of
thought and analysis, and not based on the content of the reflections
themselves. Thus, for full marks we encourage you to answer openly and honestly
and to avoid simply writing ``what you think the evaluator wants to hear.''

Please answer the following questions.  Some questions can be answered on the
team level, but where appropriate, each team member should write their own
response:


\begin{enumerate}
    \item Why is it important to create a development plan prior to starting the
    project?
    \item In your opinion, what are the advantages and disadvantages of using
    CI/CD?
    \item What disagreements did your group have in this deliverable, if any,
    and how did you resolve them?
\end{enumerate}

\newpage{}

\section*{Appendix --- Team Charter}

\wss{borrows from
\href{https://engineering.up.edu/industry_partnerships/files/team-charter.pdf}
{University of Portland Team Charter}}

\subsection*{External Goals}

\wss{What are your team's external goals for this project? These are not the
goals related to the functionality or quality fo the project.  These are the
goals on what the team wishes to achieve with the project.  Potential goals are
to win a prize at the Capstone EXPO, or to have something to talk about in
interviews, or to get an A+, etc.}

\subsection*{Attendance}

\subsubsection*{Expectations}

\wss{What are your team's expectations regarding meeting attendance (being on
time, leaving early, missing meetings, etc.)?}
Show up to all meetings unless they have an acceptable excuse or an emergency. 

\subsubsection*{Acceptable Excuse}

\wss{What constitutes an acceptable excuse for missing a meeting or a deadline?
What types of excuses will not be considered acceptable?}
For in person meetings the person must be unable to reach the meeting position and have the excuses vetted by 3/4 of the other members. The excuses should be communicated as soon as possible.  

\subsubsection*{In Case of Emergency}

\wss{What process will team members follow if they have an emergency and cannot
attend a team meeting or complete their individual work promised for a team
deliverable?}

If a team member has a emergency that impairs their ability to do work they should communicate the other team mates as soon as possible to work out an alternative plan.

\subsection*{Accountability and Teamwork}

\subsubsection*{Quality} 

\wss{What are your team's expectations regarding the quality
of team members' preparation for team meetings and the quality of the
deliverables that members bring to the team?}
Team mates should have proper tools for meetings to discuss and work on the project.
Code will be reviewed by other team-mates and code of poor quality will be rewritten  until it is of proper quality.

\subsubsection*{Attitude}

\wss{What are your team's expectations regarding team members' ideas,
interactions with the team, cooperation, attitudes, and anything else regarding
team member contributions?  Do you want to introduce a code of conduct?  Do you
want a conflict resolution plan?  Can adopt existing codes of conduct.}



\subsubsection*{Stay on Track}

\wss{What methods will be used to keep the team on track? How will your team
ensure that members contribute as expected to the team and that the team
performs as expected? How will your team reward members who do well and manage
members whose performance is below expectations?  What are the consequences for
someone not contributing their fair share?}

\wss{You may wish to use the project management metrics collected for the TA and
instructor for this.}

\wss{You can set target metrics for attendance, commits, etc.  What are the
consequences if someone doesn't hit their targets?  Do they need to bring the
coffee to the next team meeting?  Does the team need to make an appointment with
their TA, or the instructor?  Are there incentives for reaching targets early?}

\subsubsection*{Team Building}

\wss{How will you build team cohesion (fun time, group rituals, etc.)? }

\subsubsection*{Decision Making} 

\wss{How will you make decisions in your group? Consensus?  Vote? How will you
handle disagreements? }

\end{document}