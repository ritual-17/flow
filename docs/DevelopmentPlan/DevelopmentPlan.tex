\documentclass{article}

\usepackage{booktabs}
\usepackage{tabularx}

\title{Development Plan\\\progname}

\author{\authname}

\date{}

%% Comments

\usepackage{color}

\newif\ifcomments\commentstrue %displays comments
%\newif\ifcomments\commentsfalse %so that comments do not display

\ifcomments
\newcommand{\authornote}[3]{\textcolor{#1}{[#3 ---#2]}}
\newcommand{\todo}[1]{\textcolor{red}{[TODO: #1]}}
\else
\newcommand{\authornote}[3]{}
\newcommand{\todo}[1]{}
\fi

\newcommand{\wss}[1]{\authornote{magenta}{SS}{#1}} 
\newcommand{\plt}[1]{\authornote{cyan}{TPLT}{#1}} %For explanation of the template
\newcommand{\an}[1]{\authornote{cyan}{Author}{#1}}

%% Common Parts

\newcommand{\progname}{SFWRENG 4G06} % PUT YOUR PROGRAM NAME HERE
\newcommand{\authname}{Team 9, Flow
\\ Hussain Muhammed
\\ Chengze Zhao
\\ Jeffrey Doan
\\ Kevin Zhu
\\ Ethan Patterson } % AUTHOR NAMES                   

\usepackage{hyperref}
    \hypersetup{colorlinks=true, linkcolor=blue, citecolor=blue, filecolor=blue,
                urlcolor=blue, unicode=false}
    \urlstyle{same}
                                


\begin{document}

\maketitle

\begin{table}[hp]
\caption{Revision History} \label{TblRevisionHistory}
\begin{tabularx}{\textwidth}{llX}
\toprule
\textbf{Date} & \textbf{Developer(s)} & \textbf{Change}\\
\midrule
22/09/2025 & Hussain, Oliver, Jeff, Kevin, Ethan & Initial Release\\
\bottomrule
\end{tabularx}
\end{table}

\newpage{}

\section{Confidential Information?}

The project does not use any confidential information. 

\section{IP to Protect}

There is no IP to protect in this project. 


\section{Copyright License}

The group has agreed that the project should use an MIT License.


\section{Team Meeting Plan}


The team will meet 1-3 times per week, with one in-person meeting recurring at 
the same time every week. The meeting will be at a previously agreed-upon 
location that will be discussed privately. Alternatively, the meeting may be 
done virtually. We will meet with industry advisers and  other experts on a 
scheduled basis. These meetings will be announced beforehand. 

\section{Team Communication Plan}

The team will use a private Discord server along with GitHub for communication. 
Team members are expected to check both of these daily for updates to meeting 
times or other milestones.  

\section{Team Member Roles}

We will need a dedicated Team Lead, Keynoter, Team Liaison, Reviewer, and Meeting 
chair.
The team lead is in charge of scheduling meetings and other events that may be 
needed to complete the project. The Keynoter is in charge of taking notes on 
any meetings for later use. The Team Liaison is in charge of communicating with
 the Professor and other external helpers for the project. The reviewer is in
 charge of reviewing work to ensure it is of a high standard. The meeting chair 
 is in charge of managing meeting topics and making sure that the meeting stays
 productive and relevant. 

\section{Workflow Plan}

% \begin{itemize}
	
	% \item How will you be using git, including branches, pull request, etc.?
	% \item How will you be managing issues, including template issues, issue
	% classification, etc.?
  % \item Use of CI/CD
  
  The project will be using GitHub for version control and issue tracking. 
  It will be set up with a main branch that is always stable, and
  a dev branch shall be used for integration of new features. Each feature 
  or fix will first be developed in a separate branch off of dev. 
  Once the feature is complete and tested, a pull request shall be created 
  to merge the feature branch into dev. The pull request would be 
  reviewed by at least one other team member before being merged. 
  For smaller changes, the dev branch may just be used directly. 
  The dev branch will be periodically merged into main after 
  thorough testing to ensure stability. 
  \\
  \\
  The GitHub Issues will be used to track the various tasks and features 
  that need to be completed, as well as any meetings that are held.
  Templates are to be used for each issue to ensure relevant 
  information is included. Issues will also be labelled based on their type, 
  such as bug, feature, meeting, etc, where applicable. This will help in 
  organizing and prioritizing the work that needs to be done. 
  Assigning of issues will be done where necessary to track who is responsible 
  for what.
  \\
  \\
  GitHub Actions will be used for CI/CD, mainly for linting and running tests, 
  as well as updating the PDFs from the TeX files. Whenever a pull request is 
  created, the CI/CD pipeline will automatically run the actions, which would 
  include the linter and any tests that have been written. This will help catch 
  any problems early, making sure any code merged is the correct format. 
  Testing will most likely be set up later on as the project progresses and 
  more code is written.

  % \item Work in individual branches for each new issue/change. 
	% \item Merge with pull requests followed by reviews.
	% \item Use issues for work.
	% \item Use automated tests. 

% \end{itemize}

\section{Project Decomposition and Scheduling}

% \begin{itemize}
  % \item How will you be using GitHub projects?
  % \item Include a link to your GitHub project
  
  We will use GitHub projects in the Kanban board format to organize our 
  issue items. To begin, the columns will be "To Do", "In Progress", 
  and "Done". As we work on the project, more columns may be added to 
  better suit our workflow. Each issue will be added to the "To Do" column,
  and when a team member starts working on it, it will be moved to "In Progress."
  . Once the issue is completed and reviewed, it will be moved to
  "Done". This will help us keep track of what needs to be done, what is 
  currently being worked on, and what has been completed. The project 
  board can be found at 
  \url{https://github.com/users/ritual-17/projects/1/views/2}.
  
  % \item GitHub Projects will be used to 
  % help organize issues and assign them to developers. 
  % \item Project will be done on GitHub at https://github.com/ritual-17/flow.
  % \item The Project is planned to be finished at the end February 

% \end{itemize}

\section{Proof of Concept Demonstration Plan}

% What is the main risk, or risks, for the success of your project?  What will you
% demonstrate during your proof of concept demonstration to convince yourself that
% you will be able to overcome this risk?

The main risks for the success of the project have to do with factors adding 
to the time it takes to take notes, making it longer than
taking notes by other means. As the project involves making and rendering 
geometry in a GUI, there is a risk that the performance of the 
application may be too slow to be practical. Another risk is that users may 
also find the syntax of the application too cumbersome or 
unintuitive to use effectively. To mitigate these risks, the proof of concept 
demonstration will focus on showing that notes can be taken 
down quickly and easily using the application. It will involve having someone
technically proficient test writing different notes using our 
application and comparing the time it takes to write the same notes using 
other digital solutions (e.g. google docs + draw.io). One of the other solutions
will include a tool that the tester is not familiar with (potentially Mermaid, 
PlantUML, etc). This would help measure how intuitive and how straightforward
our application is to learn in comparison to others. Factors such as timing to
learn and user satisfaction can be collected. A successful demonstration will
show that notes can be taken down in a faster or comparable time to other
solutions and in a way that is intuitive and easy to use.

\section{Expected Technology}

% Topics to discuss include the following:

% \begin{itemize}
% \item Specific programming language : Electron \& TypeScript (Subject to Change)
% \item Specific libraries : TBD
% \item Pre-trained models : N/A
% \item Specific linter tool (if appropriate) : TBD
% \item Specific unit testing framework : TBD
% \item Investigation of code coverage measuring tools : TBD
% \item Specific plans for Continuous Integration (CI), or an explanation that CI
%   is not being done : Github and GitHub Projects
% \item Specific performance measuring tools (like Valgrind), if
%   appropriate : N/A
% \item Tools you will likely be using? : Discord (to communicate)
% \end{itemize}

The programming language we expect to use is Electron with TypeScript. 
These languages are chosen due to their ability to create cross-platform 
desktop applications with graphical user interfaces, which are essential 
for our note-taking application. Some libraries we may use include React 
for building the user interface components, and others for rendering shapes.
 GitHub and GitHub Projects will be used for version control and 
project management, respectively. Testing and linting are important aspects
 of our development process. Jest and ESLint are potential choices 
for unit testing and linting, respectively, but these decisions will be 
finalized as the project progresses. As previously mentioned, GitHub 
Actions will be used for CI/CD to automate testing and linting processes,
 as well as updating documentation.

\section{Coding Standard}

The Standard JavaScript Style Guide will be used as the coding standard for
this project. This guide provides a simplistic and easy-to-follow set of 
rules for writing clean and consistent code. It covers various aspects
of coding style, including naming, variables, functions and more. The specific
details of the standard can be found at 
\url{https://www.w3schools.com/js/js_conventions.asp}. standard style can 
also be integrated with ESLint, which may be used for linting the code 
to ensure it adheres to the coding standard.

\newpage{}

\section*{Appendix --- Reflection}

The purpose of reflection questions is to give you a chance to assess your own
learning and that of your group as a whole, and to find ways to improve in the
future. Reflection is an important part of the learning process.  Reflection is
also an essential component of a successful software development process.  

Reflections are most interesting and useful when they're honest, even if the
stories they tell are imperfect. You will be marked based on your depth of
thought and analysis, and not based on the content of the reflections
themselves. Thus, for full marks we encourage you to answer openly and honestly
and to avoid simply writing ``what you think the evaluator wants to hear.''

Please answer the following questions.  Some questions can be answered on the
team level, but where appropriate, each team member should write their own
response:


\begin{enumerate}
    \item Why is it important to create a development plan prior to starting the
    project?
	\begin{itemize}
		\item Jeff: A good development plan is important when working 
    on a large-scale project such as this capstone. 
    This lays out the foundation for the rest of the project.
    From this development plan, we can ensure that all
    members are on the same page for future deliverables.
    \item Ethan: It is important to create a development plan prior to starting
      the project because it establishes clear expectations, which minimizes
      miscommunication. This helps to make sure all team members are on the
      same page, which greatly improves the chance of success and efficiency at
      the implementation stage. It also requires us to outline future risks,
      which helps with planning for mitigation.
		\item Oliver: A development plan is important since it helps organize 
    the project and break it down into manageable pieces. It also helps 
    identify potential risks and challenges early on, allowing the team 
    to plan for them and mitigate them. 
		\item Hussain: Creating a development plan is important because it 
    allows all stakeholders of the project to be on the same page, 
    and to have a clear understanding of the goals and objectives 
    of the project. It also ensures that the resources needed for 
    the project is available and allocated properly. 
    A development plan also highlights any problems that 
    you may run into during the project, allowing you to plan 
    around and mitigate them before they arise.
		\item Kevin: Having a thought-out development plan is important, as the 
        rest of the project's development will follow the plan. If the plan 
        covers all possible issues, the development will be smooth, as all the 
        issues have been accounted for and will be worked around. 
	\end{itemize}
    \item In your opinion, what are the advantages and disadvantages of using
    CI/CD?
	\begin{itemize}
		\item Jeff: The advantages of using CI/CD include: 
    \begin{enumerate}
      \item Faster Feedback and Dev Cycles
      \begin{itemize}
        \item Developers get immediate feedback via automated tests 
        and have issues caught earlier in review.
      \end{itemize}
      \item Improved Code Quality
      \begin{itemize}
        \item Automated testing, linting, and code analysis help
        with coding standards.
      \end{itemize}
      \item Increased Efficiency
      \begin{itemize}
        \item Automated CI bots help reduce repetitive manual tasks.
      \end{itemize}
    \end{enumerate}
    \item Ethan: The advantages of CI/CD are that it continuously checks that
      your codebase is in a good state and can automatically run actions for
      you. This is already the case for the existing LaTeX action, but in the
      future, we will add automatic testing, linting, etc. It can also be used
      to automatically handle deployments, which is convenient. One
      disadvantage is that it is something that you sometimes have to wrestle with
      with. It can add an extra layer of complexity to your development
      environment, and therefore it is another thing that can break and add
      headaches. However, the benefits greatly outweigh the negatives in my
      opinion.
		\item Oliver: The advantages of CI/CD are that it allows for 
    faster and more frequent releases, as well as improved collaboration 
    among team members. It also helps catch bugs and issues early on in the 
    development process, leading to higher quality software. The disadvantages 
    include the initial setup and configuration can be time-consuming and 
    complex, and it may require additional resources and infrastructure to 
    maintain.
		\item Hussain: CI/CD has many advantages. Having an effective CI/CD 
    pipeline set up allows for faster development and deployment of 
    code as it automates many of the tasks that would be done manually. 
    This allows developers to focus on more complex tasks of the project. 
    The disadvantages of CI/CD are that it can be difficult to 
    initially set u,p as developers may not know what is to be 
    automated and what is not. It can also become difficult 
    to fix constantly breaking pipelines if the 
    code is not properly reviewed.
		\item Kevin: The advantages of CI/CD are that if any issues arise with 
        the project, as it is constantly being deployed, these issues can be 
        found and fixed quickly. This prevents said issues from propagating, 
        causing other issues that make the project bad.
		
		The disadvantage is that it can require proper planning and is more 
        costly in terms of time and money. This is because each development must 
        be vetted (requires time and money) before development can continue, 
        slowing down project development. 

	\end{itemize}
    \item What disagreements did your group have in this deliverable, if any,
    and how did you resolve them?
    
   \begin{itemize}
		\item The group did not have any disagreements for this deliverable. 
        The group did their work smoothly and decided in meetings on elements 
        of this document and the Problem Statement.
	\end{itemize}
\end{enumerate}

\newpage{}

\section*{Appendix --- Team Charter}

\subsection*{External Goals}

Our team has many external goals for this capstone. It would go without saying 
that we would like to achieve a high grade for our work in this course; 
however, there are many other factors as to why we are trying to succeed with 
this project. First off, as engineering students, we understand the hardships 
of having to take notes in class and not being able to smoothly incorporate 
diagrams. As we continue our careers and education, we would like to have an 
application that allows us to execute exactly what we wished we had in our 
previous undergraduate years.
Additionally, as some of us are graduating and moving on to new career 
opportunities, it would be a great benefit to us if we had a successful
 project to give us leverage when applying for new graduate positions. 


\subsection*{Attendance}

\subsubsection*{Expectations}

Our team expects all members to show up to every scheduled meeting unless they
 have an acceptable reason for not being able to attend. The specifics on what 
 constitutes an acceptable excuse will be discussed in the next section.

\subsubsection*{Acceptable Excuse}

For in-person meetings, the team understands that some members may have issues
 commuting and will not be able to physically attend. In this case, members 
 will be allowed to attend online; however, this should not be a regular 
 occurrence, and members should make an effort to attend in person. 
 Moreover, the team believes that there is no acceptable excuse for missing
 a deadline other than for emergencies. 
 This is further elaborated in the coming sections. 

Excuses that involve prioritizing extracurriculars or recreational activities will not be tolerated.
Excuses that are a result of one's own poor time management or lack thereof will also not be tolerated.
We all understand the busy schedule of an engineering student; however, it is expected that each member 
manages their time wisely and puts the same amount of effort or 
more into this capstone as they would into their other courses.



\subsubsection*{In Case of Emergency}

Group members may be met with unexpected issues, which will prevent them 
from attending a meeting or potentially miss a deadline. For meetings, 
if a member is not able to attend due to an emergency, they are expected 
to communicate that and follow up with the team on what they missed. 
They are expected to also review the meeting minutes before reaching 
out to the team about what they missed. Moreover, if a member is met 
with an emergency that prevents them from finishing work for a team 
deliverable, the team will be understanding and assist to the best 
of their capabilities. Depending on the situation, the member should 
follow the proper McMaster procedure for these emergencies and reach 
out to the TA/Instructor to discuss the potential use of an MSAF.

\subsection*{Accountability and Teamwork}

\subsubsection*{Quality} 

Team mates should have proper tools for meetings to discuss and work 
on the project.
Code will be reviewed by other team-mates and code of poor quality 
will be rewritten  until it is of proper quality.

\subsubsection*{Attitude}

Each team member is expected to treat all other members with respect and 
professionalism. Members should feel comfortable sharing ideas and 
interacting with each other. Members should come into meetings with a
 positive attitude and a willingness to actively cooperate. 
 Members should also be considerate of each other’s time and 
 should stay on track during meetings and work periods.\\

Additionally, we will be incorporating the already exisitng 
"Contributor Covenant" as our code of conduct. 
The exact code of conduct can be found 
\href{https://www.contributor-covenant.org/version/2/1/code_of_conduct/}
{here}. \\

In terms of conflict resolution plan, we decdied to implement a short and decisive process: \\
\begin{enumerate}
  \item Direct Communication
  \begin{itemize}
    \item Members directly involved in the conflict should 
    try to resolve the issue on their own first. 
  \end{itemize}
  \item Team Discussion
  \begin{itemize}
    \item If the issue remains unresolved, the team may hold a brief
    meeting to resolve said conflict. The meeting will be used to discuss
    the concerns and work towards a solution 
    approved by the majority of the group.
  \end{itemize}
  \item Seek Support from Advisor / Instructor
  \begin{itemize}
    \item If a resolution still cannot be reached after the previous steps,
    the team should schedule a meeting with the TA or Instructor and ask for 
    guidance on how to resolve the conflict. Note, this will only be used as
    a last resort after exhausting all options.
  \end{itemize}
\end{enumerate}

\subsubsection*{Stay on Track}

Keeping a team on track is very important when working on a long-term project,
 such as this capstone. To ensure that the team stays on track, we decided to
 consider implementing methods such as using project management tools, setting
 up regular update meetings, and defining clear roles and responsibilities at
 the beginning of each deliverable. The regular update meetings and assigned
 responsibilities will help ensure that members contribute their fair share 
 and that the team is performing as expected. Each member will hold the others
 accountable for their share of the deliverables, which will allow our team 
 to perform as expected. \\

It is understandable that some members may not feel motivated or may not be
 performing as expected, even with all the previous methods in place.
 To prevent this, we will be incorporating fun consequences,
 such as having these members bring hot chocolate, cookies,
 or other food items/treats to the next meeting.
 If the team member continues to underperform despite the above
 methods are in place, the other members will discuss potentially 
 meeting with a TA or instructor to resolve this issue. 
 Please note that this will only be used as a last
 resort after exhausting all other options. \\

Contrarily, if a member does exceptionally well or contributes more 
than their fair share, we’ve decided to reward these members by having 
the rest of the team split their cost at the next team event. 
For instance, the other members could contribute to the cost of 
that member’s next meal during a team lunch or dinner. \\

Additionally, if members complete their targets early, 
they won’t be assigned any more work and can focus on 
their other responsibilities. 
Incentives for finishing deliverables and tasks are 
viewed more as a team-building activity. 
These are discussed in the following section. \\


\subsubsection*{Team Building}

As engineering students, 
we understand that there are many stressors when working on a major project,
 such as a capstone. 
 These stresses can have a big effect on productivity and overall work quality.
 To combat this issue, we’ve decided to partake in a group activity or outing
 for each major goal/milestone we accomplish. For instance, after a stressful 
 deliverable, we’ll try to wind down by enjoying online games together. 
 If we accomplish a tough goal together, we think it’s a good idea to
 celebrate with a lunch/dinner to acknowledge our hard work. 
 By doing this, we hope to build team cohesion throughout this project.

\subsubsection*{Decision Making} 

We believe that decisions should be made as a group and not by one individual.
 From this, we decided that we should discuss with all members present when 
 making decisions or handling disagreements. From that discussion, 
 we will hold a vote where the majority rules. 
 We all agree to handle these discussions/meetings professionally
 and with respect for each other. That way, we can hold a responsible
 and respected discussion when disagreements arise. 

\end{document}
