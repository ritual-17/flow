\documentclass{article}

\usepackage{booktabs}
\usepackage{tabularx}
\usepackage{hyperref}
\usepackage[paper=portrait,pagesize]{typearea}

\hypersetup{
    colorlinks=true,       % false: boxed links; true: colored links
    linkcolor=red,          % color of internal links (change box color with linkbordercolor)
    citecolor=green,        % color of links to bibliography
    filecolor=magenta,      % color of file links
    urlcolor=cyan           % color of external links
}

\title{Hazard Analysis\\\progname}

\author{\authname}

\date{}

%% Comments

\usepackage{color}

\newif\ifcomments\commentstrue %displays comments
%\newif\ifcomments\commentsfalse %so that comments do not display

\ifcomments
\newcommand{\authornote}[3]{\textcolor{#1}{[#3 ---#2]}}
\newcommand{\todo}[1]{\textcolor{red}{[TODO: #1]}}
\else
\newcommand{\authornote}[3]{}
\newcommand{\todo}[1]{}
\fi

\newcommand{\wss}[1]{\authornote{magenta}{SS}{#1}} 
\newcommand{\plt}[1]{\authornote{cyan}{TPLT}{#1}} %For explanation of the template
\newcommand{\an}[1]{\authornote{cyan}{Author}{#1}}

%% Common Parts

\newcommand{\progname}{SFWRENG 4G06} % PUT YOUR PROGRAM NAME HERE
\newcommand{\authname}{Team 9, Flow
\\ Hussain Muhammed
\\ Chengze Zhao
\\ Jeffrey Doan
\\ Kevin Zhu
\\ Ethan Patterson } % AUTHOR NAMES                   

\usepackage{hyperref}
    \hypersetup{colorlinks=true, linkcolor=blue, citecolor=blue, filecolor=blue,
                urlcolor=blue, unicode=false}
    \urlstyle{same}
                                


\begin{document}

\maketitle
\thispagestyle{empty}

~\newpage

\pagenumbering{roman}

\begin{table}[hp]
\caption{Revision History} \label{TblRevisionHistory}
\begin{tabularx}{\textwidth}{llX}
\toprule
\textbf{Date} & \textbf{Developer(s)} & \textbf{Change}\\
\midrule
Date1 & Name(s) & Description of changes\\
Date2 & Name(s) & Description of changes\\
... & ... & ...\\
\bottomrule
\end{tabularx}
\end{table}

~\newpage

\tableofcontents

~\newpage

\pagenumbering{arabic}

% \wss{You are free to modify this template.}

\section{Introduction}

% \wss{You can include your definition of what a hazard is here.}

Analyzing the hazards associated with a system is an important step in
developing a safe and reliable product. It helps identify and evaluate
potential risks that could impact the safety or usability of the system. By
conducting a hazard analysis before development, we can take proactive measures
to mitigate these risks, and design preventative measures to ensure the system
operates as intended.
\\
\\
For our note-taking system (\progname), this analysis focuses on hazards that 
may lead to data loss, perfomance issues, and security vulnerabilities. As 
\progname aims to procide fast keyword-based editing with real-time diagram
rendering, it introduces potential risks in different areas including rendering
performance, data integrity, and user experience.
\\
\\
The following sections of the document define the scope and objectives of the 
hazard analysis, describe the system boundaries and main components of 
\progname, and outline key assumptions that may influence saftey considerations.
The core of the document presents a detailed Failure Mode and Effect Analysis
(FMEA) table, which identifies potential hazards, their causes and effects.
Finally, the document concludes with a summary of safety and security
requirements, and provides a roadmap for implementing these requirements in
future development phases.

\section{Scope and Purpose of Hazard Analysis}

% \wss{You should say what \textbf{loss} could be incurred because of the hazards.}
The purpose of this hazard analysis is to identify and evaluate the potential 
risks that could impact the safety, security, and usability of \progname. 
Conducting this analysis in the early stages of development allows for us to 
anticipate possible failure points in the system and implement design strategies
to prevent data loss, ensure stable performance, and maintain user confidence.
\\
\\
The text-editing interface, diagram rendering engine, and file storage system all 
make up the core components of \progname. This analysis will focus on both 
software related hazards, such as crashes, performance issues, and 
synchronization errors, as well as user-related risks, including data loss, 
security vulnerabilities, and unintuitive user interactions.
\\
\\
Losses that may be incurred due as a result of these hazards include:
\begin{itemize}
    \item \textbf{Loss of user data} from failed saves, corruption, or unexpected
    application closures
    \item \textbf{Loss of productivity} due to lag, unresponsiveness or
    unintuitive interface or command design
    \item \textbf{Loss of reliability and maintainability} if the application 
    is developed with technical dept or poor coding practices
    \item \textbf{Loss of user trust} due to repeated bugs, confusing features,
    or inconsistent behavior
    \item \textbf{Loss of security or privacy} if data is not properly protected
    or stored securely
\end{itemize}

The hazard analysis focuses on factors in the software domain only. While
hardware failures (e.g., device crashes, power loss), as well as operating
system issues (e.g., file system corruption, OS crashes) can impact the
performance and reliability of \progname, these are outside the scope of this
analysis. \progname will be running in the user space and any potential
problems occuring at the kernal level or below are will therefore not be 
considered. The analysis will assume that the underlying hardware and OS
are functioning correctly, and will focus on hazards that can be directly
attributed to the design and implementation of \progname itself.


\section{System Boundaries and Components}

% \wss{Dividing the system into components will help you brainstorm the hazards.
% You shouldn't do a full design of the components, just get a feel for the major
% ones.  For projects that involve hardware, the components will typically include
% each individual piece of hardware.  If your software will have a database, or an
% important library, these are also potential components.}
\progname is a note-taking application that allows users to create, edit, and
visualize text-based diagrams using only keyboard inputs. The boundaries of
the system include all components directly related to the core functionality of
the application, including taking user input, rendering diagrams, and file 
management. The system boundary does not include external dependencies such as 
the user's operating system, device hardware, or third-party file storage services 
such as cloud storage providers.
\\
\\
The main components of \progname include:
\begin{enumerate}
    \item \textbf{User Interface (UI) Layer}
    \begin{itemize}
        \item Handling user interactions, keyboard inputs, displaying updates
        \item Showing live view of notes and diagrams
        \item Providing feedback on actions (e.g., error messages)
    \end{itemize}
    \item \textbf{Command Processing Engine}
    \begin{itemize}
        \item Parsing and interpreting user commands
        \item Managing application state based on user inputs
        \item Sending structured actions to the rendering and storage layers
    \end{itemize}
    \item \textbf{Diagram Rendering Engine}
    \begin{itemize}
        \item Converting text-based diagram descriptions into visual formats
        \item Handling shape layout, positioning, and connection to the rest of
        the note content
    \end{itemize}
    \item \textbf{File Management and Storage Layer}
    \begin{itemize}
        \item Saving and loading notes and diagrams to/from disk
        \item Managing file formats and data serialization
        \item Ensuring data integrity during read/write operations
    \end{itemize}\
    \item \textbf{Error Handling and State Management Module}
    \begin{itemize}
        \item Detecting, logging, and managing errors across all components
        \item Implementing recovery mechanisms and stability after crashes or
        unexpected shutdowns
    \end{itemize}
    % \item \textbf{Configuration/Customization Module} %potential future 
    % component
    % \begin{itemize}
    %     \item Storing and Managing user preferences and application settings
    % \end{itemize}
\end{enumerate}

In addition to these core components, \progname may also depend on external
libraries for specific functionalities, such as diagram rendering or parsing.
To generate and display diagrams, the application may utilize a graphics 
library to allow for efficient rendering and manipulation of visual elements.
For parsing user commands, a parsing library may be employed to simplify the
interpretation of complex inputs. State management libraries may also be used
to help maintain application state and ensure consistency across different
components.


\section{Critical Assumptions}

% \wss{These assumptions that are made about the software or system.  You should
% minimize the number of assumptions that remove potential hazards.  For instance,
% you could assume a part will never fail, but it is generally better to include
% this potential failure mode.}
Certain key assumptions will be made regarding the user and operating environment
during the development and use of \progname. These assumptions include:
\\
\\
\begin{enumerate}
    \item \textbf{User Competency:} It is assumed that users have a basic level
    of computer literacy and is familiar with computer opertations such as using
    a keyboard and mouse. Users are expected to follow reasonable usage patterns
    (e.g. not shutting down the application abruptly or attempting to
    manipulate internal files directly).
    \item \textbf{File Access and Storage:} It is assumed that the file system
    used for storing notes and diagrams is reliable and free from corruption.
    Users are expected to have sufficient permissions to read/write files in
    the designated storage locations.
    \item \textbf{Operating Environment:} It is assumed that \progname will be
    run on a supported operating system without unexpected interruptions such
    as forced shutdowns, crashes, or power failures. The system is expected to
    have adequate disk space and memory to run the application smoothly.
    \item \textbf{Software Dependencies:} It is assumed that any third-party
    libraries or frameworks used by \progname are stable, compatible, and
    function as intended on the target operating systems. 
\end{enumerate}

% Make new landscape page for FMEA table
\newpage
\KOMAoptions{paper=landscape,DIV=8,pagesize}
\recalctypearea

\section{Failure Mode and Effect Analysis}

% \wss{Include your FMEA table here. This is the most important part of this 
% document.}
% \wss{The safety requirements in the table do not have to have the prefix SR.
% The most important thing is to show traceability to your SRS. You might trace 
% to requirements you have already written, or you might need to add new
% requirements.}
% \wss{If no safety requirement can be devised, other mitigation strategies can 
% be entered in the table, including strategies involving providing additional
% documentation, and/or test cases.}
\begin{table}[htbp]
\centering
\small
\caption{Failure Mode and Effect Analysis (FMEA) for \progname} 
\begin{tabularx}{\textwidth}{X X X X X X X}
\toprule
\textbf{Design Function} & \textbf{Failure Modes} & \textbf{Effect of failure} &
 \textbf{Causes for failure} & \textbf{Detection} & \textbf{Recommended actions} 
 & \textbf{SR}\\
\midrule
Diagram rendering and layout & Overlapping shapes, unreadable layout, slow 
rendering & Reduced usability, user spending more time rearranging rather 
than note taking & Poor algorithm layout, too large graphics & Creation shows
long frame times, UI freezes, user complaints, perfomance test threshold time 
limit exceeded & Have Unit tests performance tests for frame time; Use worker 
threads for rendering; Set rendering performance limits (e.g. < 100ms per 
update); Implement partial rendering (e.g. rendering visual portion first); 
Potentially user to pin diagram positions & SR-RENDER-1 \\
Command parsing & Parser misinterprets or fails to parse user input & 
Unexpected or no diagram output, lost note, user confusion/frustration & 
Unintuitive grammar, improper user input, unhandled exceptions & Parser 
exception and logs, failed rendering, regression tests failing on inputs & 
Define simple, well-documented grammar and implement intuitive parser with
proper error message handling; Provide immediate syntax validation (e.g. 
highlight errors); Fuzz unit tests to handle any and all errors; On parse 
error, preserve raw text and do not delete it & SR-PARSE-1 \\
Load/Open note & Note fails to load, note partially loads, note produces 
wrong rendering & User cannot access previous work, user confusion/frustration,
loss of user trust & Incompatible/invalid file format, deserialization 
error & Load exceptions, visual regression tests, user reports & File format 
version tag with schema validation during load; Save raw text of error log,
tests for loading after saving & SR-LOAD-1 \\
Save note & File not saved, save interrupted & File missing or corrupted, 
Loss of notes or diagrams, decrease in user productivity, loss of user trust 
in application & Power loss mid-save, app crash during write, full file 
storage & Exceptions logged, file hash value (or checksum) mismatch, 
unexpected file size, user reported missing file & Implement autosaving or 
atomic saving (save to temp file and rename it); Verify with checksumes on 
save and load; Inform user of insufficient disk space before save; Unit 
tests for save routines & SR-SAVE-1 \\
Keybinding Commands & Key conflicts, lost keystrokes (commands done
unexpectedly) & Inability to create note, user frustration, potential data
loss from unintended commands done (e.g. Alt+F4) & Global shortcut 
overlapping (e.g. Ctrl+C natively means copy), race conditions in input 
handling & Unit tests for commands, integration tests under different 
environments, user reports & Allow for remappable keybinding (with safe
default set); Expose preview of keybinds; Provide undo command/option &
SR-INPUT-1 \\
Interacting with shapes/diagrams (Edit/Add Text/Move/Delete) & Edit commands 
behave incorrectly, loss of shape selection, text addition is 
inconsistent/undesired & Wrong edits made, user frustration, loss of user trust,
possible data corruption if partial edits made or deletes of inccorect item & 
Text and visual model not in sync, race conditions in input handling, 
undo stack not properly managed & Unit tests for edit commands and sequences
of commands, user reports & Implement robust undo/redo stack; Integration 
testing simulating keyboard macros; Sync checks for text and visual model;
Provide visual feedback on selection and edits; & SR-INTERACT-1 \\
Custom Geometry creation (user-defined shapes) & Custom shapes fail to render, 
Shapes cannot be reusedd reliably & User time wasted recreating shapes, loss of
user trust in customization features & Improper serialization/deserialization,
ambiguous grammar for custom shapes, render limitations & Save/load of custom
shapes fails, unit tests for custom shape rendering and serialization, user 
reports & Define clear grammar for custom shapes; Test creating, saving,
loading, and rendering custom shapes; Initially limit complexity of custom
shapes (e.g. no nested shapes); & SR-CUSTOM-1 \\
Save custom geometry as reusable commands/macros & Macro/command fails to save,
Macro/command working incorrectly & Loss of user productivity, Macro/command
produces wrong output, parse crashes, user frustration & Poor macro 
serialization, grammar/syntax collisions, improper parsing & Macro creation 
errors logged, regression tests for macros application, user reports & Isolate
macro syntax from normal command syntax; Provide clear error messages on
macro creation; Unit tests for macro creation, saving, loading, and execution;
& SR-MACRO-1 \\
Overall performance/responsiveness vs alternatives & Easier and more intuitive 
to take notes on other apps, app feels slower than existing tools (e.g. laggy
rendering) & Users revert to other tools, poor doption of app & Inefficient
algorithms, full rendering on every change, blocking main thread & User 
performance testing, user feedback, collecting performance metrics & 
Incrementally render (diagram first, then text); If document becomes large, 
only render visible portion; Use worker threads for rendering; Performance 
budgets for each operation (e.g. <100ms for rendering update); Measure 
against alternative apps in user tests & SR-PERF-1 \\
\bottomrule
\end{tabularx}
\end{table}

% Revert back to portrait for the rest of the document
\newpage
\KOMAoptions{paper=portrait,pagesize}
\recalctypearea

\section{Safety and Security Requirements}

\wss{Newly discovered requirements.  These should also be added to the SRS.  (A
rationale design process how and why to fake it.)}

\section{Roadmap}

\wss{Which safety requirements will be implemented as part of the capstone timeline?
Which requirements will be implemented in the future?}

\newpage{}

\section*{Appendix --- Reflection}

\wss{Not required for CAS 741}

The purpose of reflection questions is to give you a chance to assess your own
learning and that of your group as a whole, and to find ways to improve in the
future. Reflection is an important part of the learning process.  Reflection is
also an essential component of a successful software development process.  

Reflections are most interesting and useful when they're honest, even if the
stories they tell are imperfect. You will be marked based on your depth of
thought and analysis, and not based on the content of the reflections
themselves. Thus, for full marks we encourage you to answer openly and honestly
and to avoid simply writing ``what you think the evaluator wants to hear.''

Please answer the following questions.  Some questions can be answered on the
team level, but where appropriate, each team member should write their own
response:


\begin{enumerate}
    \item What went well while writing this deliverable? 
    \item What pain points did you experience during this deliverable, and how
    did you resolve them?
    \item Which of your listed risks had your team thought of before this
    deliverable, and which did you think of while doing this deliverable? For
    the latter ones (ones you thought of while doing the Hazard Analysis), how
    did they come about?
    \item Other than the risk of physical harm (some projects may not have any
    appreciable risks of this form), list at least 2 other types of risk in
    software products. Why are they important to consider?
\end{enumerate}

\end{document}