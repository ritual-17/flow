\documentclass{article}

\usepackage{booktabs}
\usepackage{tabularx}
\usepackage{hyperref}

\hypersetup{
    colorlinks=true,       % false: boxed links; true: colored links
    linkcolor=red,          % color of internal links (change box color with linkbordercolor)
    citecolor=green,        % color of links to bibliography
    filecolor=magenta,      % color of file links
    urlcolor=cyan           % color of external links
}

\title{Hazard Analysis\\\progname}

\author{\authname}

\date{}

\input{../Comments}
%% Common Parts

\newcommand{\progname}{Flow} % PUT YOUR PROGRAM NAME HERE
\newcommand{\authname}{Team 9, Team Name
 Student 1 name
 Student 2 name
 Student 3 name
 Student 4 name} % AUTHOR NAMES                  

\usepackage{hyperref}
    \hypersetup{colorlinks=true, linkcolor=blue, citecolor=blue, filecolor=blue,
                urlcolor=blue, unicode=false}
    \urlstyle{same}
                                


\begin{document}

\maketitle
\thispagestyle{empty}

~\newpage

\pagenumbering{roman}

\begin{table}[hp]
\caption{Revision History} \label{TblRevisionHistory}
\begin{tabularx}{\textwidth}{llX}
\toprule
\textbf{Date} & \textbf{Developer(s)} & \textbf{Change}\\
\midrule
Date1 & Name(s) & Description of changes\\
Date2 & Name(s) & Description of changes\\
... & ... & ...\\
\bottomrule
\end{tabularx}
\end{table}

~\newpage

\tableofcontents

~\newpage

\pagenumbering{arabic}

\wss{You are free to modify this template.}

\section{Introduction}

\wss{You can include your definition of what a hazard is here.}

\section{Scope and Purpose of Hazard Analysis}

\wss{You should say what \textbf{loss} could be incurred because of the
hazards.}

\section{System Boundaries and Components}

\wss{Dividing the system into components will help you brainstorm the hazards.
You shouldn't do a full design of the components, just get a feel for the major
ones.  For projects that involve hardware, the components will typically include
each individual piece of hardware.  If your software will have a database, or an
important library, these are also potential components.}

\section{Critical Assumptions}

\wss{These assumptions that are made about the software or system.  You should
minimize the number of assumptions that remove potential hazards.  For instance,
you could assume a part will never fail, but it is generally better to include
this potential failure mode.}

\section{Failure Mode and Effect Analysis}

\wss{Include your FMEA table here. This is the most important part of this document.}
\wss{The safety requirements in the table do not have to have the prefix SR.
The most important thing is to show traceability to your SRS. You might trace to
requirements you have already written, or you might need to add new
requirements.}
\wss{If no safety requirement can be devised, other mitigation strategies can be
entered in the table, including strategies involving providing additional
documentation, and/or test cases.}

\section{Safety and Security Requirements}

\wss{Newly discovered requirements.  These should also be added to the SRS.  (A
rationale design process how and why to fake it.)}

\section{Roadmap}

\wss{Which safety requirements will be implemented as part of the capstone timeline?
Which requirements will be implemented in the future?}

\newpage{}

\section*{Appendix --- Reflection}

\wss{Not required for CAS 741}

\input{../Reflection.tex}

\begin{enumerate}
    \item What went well while writing this deliverable? 
    \begin{itemize}
        \item Chengze - 
        \item Ethan - 
        \item Hussain - 
        \item Jeffrey - 
        \item Kevin - Our group managed to smoothly assign work and a work to review to each member. This was done in group meetings and was helpful as it meant we could fit the work to our schedules, meaning that we did not have to find times we could meet to do work. These times were reserved for asking each other questions and making sure our work lined up with each others. Note that I did not work on the HA document and this is related to my work on the SRS instead.  
    \end{itemize} 
    \item What pain points did you experience during this deliverable, and how
    did you resolve them?
    \begin{itemize}
        \item Chengze - 
        \item Ethan - 
        \item Hussain - 
        \item Jeffrey - 
        \item Kevin - One pain point of this deliverable was the large volume of work. As this SRS was a project that took us the whole semester to finish last year. However, as we are now more experienced and understand better how to write the SRS it took less time as we knew what to write without second guessing ourselves.
    \end{itemize} 
    \item Which of your listed risks had your team thought of before this
    deliverable, and which did you think of while doing this deliverable? For
    the latter ones (ones you thought of while doing the Hazard Analysis), how
    did they come about?
    \begin{itemize}
        \item Chengze - 
        \item Ethan - 
        \item Hussain - 
        \item Jeffrey - 
        \item Kevin - I was not involved in the Writing of the Hazard Analysis. I would assume that since there were two people working on the HA they would ask questions about it to each other before seeking us for more input. 
    \end{itemize} 
    \item Other than the risk of physical harm (some projects may not have any
    appreciable risks of this form), list at least 2 other types of risk in
    software products. Why are they important to consider?
    \begin{itemize}
        \item Chengze - 
        \item Ethan - 
        \item Hussain - 
        \item Jeffrey - 
        \item Kevin - Other risks to consider are financial risk and security risks. Financial risks mean that if the project runs out of money it may be unable to continue or cost more than expected with the extra cost coming out of the companies pocket. These are important as extra costs could make projects bad investments or even unprofitable, eventually bankrupting the company. Security risks can mean sensitive data could leak, allowing bad actors access to do whatever they want. Not only can these bad actors use the data to commit crimes such as fraud, security risks can cause legal trouble and erode trust in the program encouraging users to look elsewhere. 
    \end{itemize} 
\end{enumerate}

\end{document}