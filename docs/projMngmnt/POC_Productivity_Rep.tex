\documentclass{article}

\usepackage{float}
\restylefloat{table}

\usepackage{booktabs}

\title{Team Productivity: POC\\\progname}

\author{\authname}

\date{}

\input{../Comments}
%% Common Parts

\newcommand{\progname}{Flow} % PUT YOUR PROGRAM NAME HERE
\newcommand{\authname}{Team 9, Team Name
 Student 1 name
 Student 2 name
 Student 3 name
 Student 4 name} % AUTHOR NAMES                  

\usepackage{hyperref}
    \hypersetup{colorlinks=true, linkcolor=blue, citecolor=blue, filecolor=blue,
                urlcolor=blue, unicode=false}
    \urlstyle{same}
                                


\begin{document}

\maketitle

This document summarizes the contributions of each team member up to the POC
Demo.  The time period of interest is the time between the beginning of the term
and the POC demo.

\section{Demo Plans}

During the PoC Demonstration our team will do a run through of a basic version of
 our final product. A user will use it to take notes and compare it to other methods
  of note taking (ex docs + draw.io), showing that our product is comparable. This
   shows that our product is viable alternative to the current popular note taking
   methods. 

\section{Team Meeting Attendance}


\begin{table}[H]
\centering
\begin{tabular}{ll}
\toprule
\textbf{Student} & \textbf{Meetings}\\
\midrule
Total & 7\\
Ethan Patterson & 7\\
Hussain Muhammed & 6\\
Jeffrey Doan & 7\\
Kevin Zhu & 7\\
Chengzhe Zhao & 6\\
\bottomrule
\end{tabular}
\end{table}

\wss{If needed, an explanation for the counts can be provided here.}

\section{Supervisor/Stakeholder Meeting Attendance}

\noindent \textbf{Supervisor's Name: } [N/A]

\begin{table}[H]
\centering
\begin{tabular}{ll}
\toprule
\textbf{Student} & \textbf{Meetings}\\
\midrule
Total & 0\\
Ethan & 0\\
Hussain & 0\\
Jeffrey & 0\\
Kevin & 0\\
Chengze & 0\\
\bottomrule
\end{tabular}
\end{table}

We have had discussions within the team when eliciting requirements as you 
would with stakeholders, but not formally structured meetings.

\section{Lecture Attendance}

\begin{table}[H]
\centering
\begin{tabular}{ll}
\toprule
\textbf{Student} & \textbf{Lectures}\\
\midrule
Total & 10\\
Ethan & 10\\
Hussain & 8\\
Jeffrey & 2\\
Kevin & 3\\
Chengze & 6\\
\bottomrule
\end{tabular}
\end{table}


\section{TA Document Discussion Attendance}

\noindent \textbf{TA's Name: Lucas Dutton}

\begin{table}[H]
\centering
\begin{tabular}{ll}
\toprule
\textbf{Student} & \textbf{Lectures}\\
\midrule
Total & 3\\
Ethan & 3\\
Hussain & 3\\
Jeffrey & 2\\
Kevin & 3\\
Chengze & 3\\
\bottomrule
\end{tabular}
\end{table}

\section{Commits}

\begin{table}[H]
\centering
\begin{tabular}{lll}
\toprule
\textbf{Student} & \textbf{Commits} & \textbf{Percent}\\
\midrule
Total & Num & 100\% \\
Ethan & 92 & 33.3\% \\
Hussain & 70 & 25.3\% \\
Jeffrey & 46 & 16.7\% \\
Kevin & 25 & 9.06\% \\
Chengze & 43 & 15.6\% \\
\bottomrule
\end{tabular}
\end{table}

\section{Issue Tracker}

\begin{table}[H]
\centering
\begin{tabular}{lll}
\toprule
\textbf{Student} & \textbf{Authored (O+C)} & \textbf{Assigned (C only)}\\
\midrule
Ethan & 42 & 31 \\
Hussain & 28 & 24 \\
Jeffrey & 15 & 21 \\
Kevin & 8 & 14 \\
Chengze & 15 & 20 \\
\bottomrule
\end{tabular}
\end{table}

\section{CICD}

So far CI/CD has been used for building our Latex doc files. The Latex workflow
was included in the repo template, but was updated by the team to fix a small
bug that would cause pushes to main when running on a PR. Additionally, we used
the SRS-Meyer template, which used ascii doctor for document markup rather than
Latex. We created a new GitHub actions workflow to build the adoc files into
pdfs, similar to the provided Latex one. Going forward, once we have the
testing infrastructure set up, we will be running these tests automatically
with new workflows. We will also create workflows for a linting and formatting
check step.

\section{Team Charter Trigger Items}

Our team has a team charter included in the appendix of the Development Plan. 
However, the charter does not include quantified triggers or measurable performance 
criteria. As such, no violations of triggers were identified, and no adjustments were necessary.
\section{Additional Productivity Metrics}

Our team does not have any additional metrics of
productivity aside from the ones listed in this report.

\end{document}
