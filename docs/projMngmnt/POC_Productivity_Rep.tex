\documentclass{article}

\usepackage{float}
\restylefloat{table}

\usepackage{booktabs}

\title{Team Productivity: POC\\\progname}

\author{\authname}

\date{}

\input{../Comments}
%% Common Parts

\newcommand{\progname}{Flow} % PUT YOUR PROGRAM NAME HERE
\newcommand{\authname}{Team 9, Team Name
 Student 1 name
 Student 2 name
 Student 3 name
 Student 4 name} % AUTHOR NAMES                  

\usepackage{hyperref}
    \hypersetup{colorlinks=true, linkcolor=blue, citecolor=blue, filecolor=blue,
                urlcolor=blue, unicode=false}
    \urlstyle{same}
                                


\begin{document}

\maketitle

This document summarizes the contributions of each team member up to the POC
Demo.  The time period of interest is the time between the beginning of the term
and the POC demo.

\section{Demo Plans}

\wss{What will you be demonstrating}

\section{Team Meeting Attendance}

\wss{For each team member how many team meetings have they attended over the
time period of interest.  This number should be determined from the meeting
issues in the team's repo.  The first entry in the table should be the total
number of team meetings held by the team.}

\begin{table}[H]
\centering
\begin{tabular}{ll}
\toprule
\textbf{Student} & \textbf{Meetings}\\
\midrule
Total & Num\\
Name 1 & Num\\
Name 2 & Num\\
Name 3 & Num\\
Name 4 & Num\\
Name 5 & Num\\
\bottomrule
\end{tabular}
\end{table}

\wss{If needed, an explanation for the counts can be provided here.}

\section{Supervisor/Stakeholder Meeting Attendance}

\wss{For each team member how many supervisor/stakeholder team meetings have
they attended over the time period of interest.  This number should be determined
from the supervisor meeting issues in the team's repo.  The first entry in the
table should be the total number of supervisor and team meetings held by the
team.  If there is no supervisor, there will usually be meetings with
stakeholders (potential users) that can serve a similar purpose.}

\noindent \textbf{Supervisor's Name: } [fill in this information]

\begin{table}[H]
\centering
\begin{tabular}{ll}
\toprule
\textbf{Student} & \textbf{Meetings}\\
\midrule
Total & Num\\
Name 1 & Num\\
Name 2 & Num\\
Name 3 & Num\\
Name 4 & Num\\
Name 5 & Num\\
\bottomrule
\end{tabular}
\end{table}

\wss{If needed, an explanation for the counts can be provided here.}

\section{Lecture Attendance}

\wss{For each team member how many lectures have they attended over the time
period of interest.  This number should be determined from the lecture issues in
the team's repo. You can find the number of lectures in the time period of
interest by looking at the
\href{https://calendar.google.com/calendar/u/0/embed?src=rnboqiaki1k2la7rpu3bn0um58@group.calendar.google.com&ctz=America/Toronto}
{Google calendar} for the capstone course.}

\wss{NOTE: There will be approximately 13 lectures between the start of class
and the POC demos}

\begin{table}[H]
\centering
\begin{tabular}{ll}
\toprule
\textbf{Student} & \textbf{Lectures}\\
\midrule
Total & Num\\
Name 1 & Num\\
Name 2 & Num\\
Name 3 & Num\\
Name 4 & Num\\
Name 5 & Num\\
\bottomrule
\end{tabular}
\end{table}

\wss{If needed, an explanation for the lecture attendance can be provided here.}

\section{TA Document Discussion Attendance}

\noindent \textbf{TA's Name: Lucas Dutton}

\begin{table}[H]
\centering
\begin{tabular}{ll}
\toprule
\textbf{Student} & \textbf{Lectures}\\
\midrule
Total & 3\\
Ethan & 3\\
Hussain & 3\\
Jeffrey & 2\\
Kevin & 3\\
Chengze & 3\\
\bottomrule
\end{tabular}
\end{table}

\section{Commits}

\begin{table}[H]
\centering
\begin{tabular}{lll}
\toprule
\textbf{Student} & \textbf{Commits} & \textbf{Percent}\\
\midrule
Total & Num & 100\% \\
Ethan & 92 & 33.3\% \\
Hussain & 70 & 25.3\% \\
Jeffrey & 46 & 16.7\% \\
Kevin & 25 & 9.06\% \\
Chengze & 43 & 15.6\% \\
\bottomrule
\end{tabular}
\end{table}

\section{Issue Tracker}

\begin{table}[H]
\centering
\begin{tabular}{lll}
\toprule
\textbf{Student} & \textbf{Authored (O+C)} & \textbf{Assigned (C only)}\\
\midrule
Ethan & 42 & 31 \\
Hussain & 28 & 24 \\
Jeffrey & 15 & 21 \\
Kevin & 8 & 14 \\
Chengze & 15 & 20 \\
\bottomrule
\end{tabular}
\end{table}

\section{CICD}

So far CI/CD has been used for building our Latex doc files. The Latex workflow
was included in the repo template, but was updated by the team to fix a small
bug that would cause pushes to main when running on a PR. Additionally, we used
the SRS-Meyer template, which used ascii doctor for document markup rather than
Latex. We created a new GitHub actions workflow to build the adoc files into
pdfs, similar to the provided Latex one. Going forward, once we have the
testing infrastructure set up, we will be running these tests automatically
with new workflows. We will also create workflows for a linting and formatting
check step.

\section{Team Charter Trigger Items}

\wss{Provide a summary of the quantified triggers identified in the team's
charter.}

\wss{Provide a list of any violations of the triggers.  If the team wishes, the
violations can be summarized on aggregate, instead of naming specific team
members.}

\wss{Provide a plan to address the violations.  This could include revising the
triggers, if they are found to be too weak, strong or ambiguous.}

\section{Additional Productivity Metrics}

Our team does not have any additional metrics of
productivity aside from the ones listed in this report.

\end{document}
