\documentclass{article}

\usepackage{float}
\restylefloat{table}

\usepackage{booktabs}

\title{Team Productivity: Rev 0\\\progname}

\author{\authname}

\date{}

\input{../Comments}
%% Common Parts

\newcommand{\progname}{Flow} % PUT YOUR PROGRAM NAME HERE
\newcommand{\authname}{Team 9, Team Name
 Student 1 name
 Student 2 name
 Student 3 name
 Student 4 name} % AUTHOR NAMES                  

\usepackage{hyperref}
    \hypersetup{colorlinks=true, linkcolor=blue, citecolor=blue, filecolor=blue,
                urlcolor=blue, unicode=false}
    \urlstyle{same}
                                


\begin{document}

\maketitle

This document summarizes the contributions of each team member for the Rev 0
Demo.  The time period of interest is the time between the PoC demo and the Rev
0 demo; the contributions prior to the PoC are NOT included.

\wss{Please do not delete sections or modify the table formats.  Standardization
helps the instructors and TAs to review these documents. You may add sections to
the report.}

\wss{Remove the instructor guidelines/comments in the final version.}

\wss{Fill in the requested information, like project url, TA name, supervisor name, etc.}

\section{Demo Plans}

\wss{What will you be demonstrating}
For our Rev0 Demo we plan to showcase taking a note that involves drawing a state machine, writing some text, and quick navigation around the note.
\section{Team Meeting Attendance}

\wss{For each team member how many team meetings have they attended over the
time period of interest.  This number should be determined from the meeting
issues in the team's repo.  The first entry in the table should be the total
number of team meetings held by the team.}

\begin{table}[H]
\centering
\begin{tabular}{ll}
\toprule
\textbf{Student} & \textbf{Meetings}\\
\midrule
Total & 0\\
Ethan & 0\\
Hussain & 0\\
Jeffrey & 0\\
Kevin & 0\\
Chengze & 0\\
\bottomrule
\end{tabular}
\end{table}

\wss{If needed, an explanation for the counts can be provided here.}
Our group did not have any meeting up to the time of this productivity report. Our group used asynchronous communication via discord and github to delegate and complete tasks.

\section{Supervisor/Stakeholder Meeting Attendance}

\wss{For each team member how many supervisor/stakeholder team meetings have
they attended over the time period of interest.  This number should be determined
from the supervisor meeting issues in the team's repo.  The first entry in the
table should be the total number of supervisor and team meetings held by the
team.  If there is no supervisor, there will usually be meetings with
stakeholders (potential users) that can serve a similar purpose.}

\noindent \textbf{Supervisor's Name: } [fill in this information]

\begin{table}[H]
\centering
\begin{tabular}{ll}
\toprule
\textbf{Student} & \textbf{Meetings}\\
\midrule
Total & 0\\
Ethan & 0\\
Hussain & 0\\
Jeffrey & 0\\
Kevin & 0\\
Chengze & 0\\
\bottomrule
\end{tabular}
\end{table}

\wss{If needed, an explanation for the counts can be provided here.}

\section{Lecture Attendance}

\wss{For each team member how many lectures have they attended over the time
period of interest.  This number should be determined from the lecture issues in
the team's repo. You can find the number of lectures in the time period of
interest by looking at the
\href{https://calendar.google.com/calendar/u/0/embed?src=rnboqiaki1k2la7rpu3bn0um58@group.calendar.google.com&ctz=America/Toronto}
{Google calendar} for the capstone course.}

\wss{NOTE: There will be approximately 1 lecture between the POC and Rev0 demos}

\begin{table}[H]
\centering
\begin{tabular}{ll}
\toprule
\textbf{Student} & \textbf{Lectures}\\
\midrule
Total & 11\\
Ethan & 11\\
Hussain & 8\\
Jeffrey & 2\\
Kevin & 3\\
Chengze & 6\\
\bottomrule
\end{tabular}
\end{table}

\wss{If needed, an explanation for the lecture attendance can be provided here.}

\section{TA Document Discussion Attendance}

\wss{For each team member how many of the informal document discussion meetings
with the TA were attended over the time period of interest.}

\noindent \textbf{TA's Name: } Lucas Dutton

\begin{table}[H]
\centering
\begin{tabular}{ll}
\toprule
\textbf{Student} & \textbf{Lectures}\\
\midrule
Total & 5\\
Ethan & 5\\
Hussain & 5\\
Jeffrey & 4\\
Kevin & 5\\
Chengze & 5\\
\bottomrule
\end{tabular}
\end{table}

\wss{If needed, an explanation for the attendance can be provided here.}

\section{Commits}

\wss{There are also columns for lines added and lines deleted to give a sense of
the magnitude of each team member's contributions.  Note that these numbers can
be somewhat misleading, as a team member who does refactoring may have high
numbers in both columns, while a team member who implements a large new feature
may have a high number of lines added, but few lines deleted.  These numbers
will be interpreted with care.}

\begin{table}[H]
\centering
\begin{tabular}{lll}
\toprule
\textbf{Student} & \textbf{Commits} & \textbf{Percent}\\
\midrule
Total & 407 & 100\% \\
Ethan & 146 & 35.87\% \\
Hussain & 96 & 23.59\% \\
Jeffrey & 61 & 14.99\% \\
Kevin & 37 & 9.09\% \\
Chengze & 67 & 16.46\% \\
\bottomrule
\end{tabular}
\end{table}

\begin{table}[H]
\centering
\begin{tabular}{lllll}
\toprule
\textbf{Student} & \textbf{Commits} & \textbf{Percent} & \textbf{Lines Added} &
\textbf{Lines Deleted}\\
\midrule
Total & Num & 100\% & total added & total deleted\\
Name 1 & Num & \% & num & num\\
Name 2 & Num & \% & num & num\\
Name 3 & Num & \% & num & num\\
Name 4 & Num & \% & num & num\\
Name 5 & Num & \% & num & num\\
\bottomrule
\end{tabular}
\end{table}

\section{Issue Tracker}

\begin{table}[H]
\centering
\begin{tabular}{lll}
\toprule
\textbf{Student} & \textbf{Authored (O+C)} & \textbf{Assigned (C only)}\\
\midrule
Ethan & 22  & 4  \\
Hussain & 11  & 8 \\
Jeffrey & 11  & 8  \\
Kevin & 4  & 5  \\
Chengze & 6  & 7 \\
\bottomrule
\end{tabular}
\end{table}

\section{CICD via GitHub Actions}

CI/CD is being used to compile, lint, format, and run tests on our codebase.
The pipeline is run on every PR and needs to pass before it can be merged. It
is also run on every merge to the main branch.

LaTex workflow: \url{https://github.com/ritual-17/flow/blob/main/.github/workflows/buildtex.yml}
Ascii doctor workflow (for SRS): \url{https://github.com/ritual-17/flow/blob/main/.github/workflows/buildadoc.yaml}
CICD workflow: \url{https://github.com/ritual-17/flow/blob/main/.github/workflows/ci.yaml}

\section{Extras}

\wss{What is the plan (as documented in TeamComposition.csv) for the team's
extras?  Should the extras be modified now that the team knows more about the
project?}

\section{Team Charter Trigger Items}

For Revision 0, our team did not use a strict set of quantified trigger items, but
 we followed general expectations such as completing assigned tasks on time, communicating
 regularly, and staying involved in team progress. No major issues or violations
 occurred during this revision, as everyone contributed consistently. In future revisions,
 we plan to define clearer and more measurable triggers to help track productivity and
 address any problems more effectively.

\section{Additional Productivity Metrics}

\wss{If your team has additional metrics of productivity, please feel free to
add them to this report.  If not, please explicitly state that there are no additional
metrics.}

\wss{Additional metrics can include things like code reviews done, pull requests
created, count of joining meetings late, count of number of times contributions
had to be corrected, number of internal deadlines missed, test cases written, etc.}

\wss{We are looking for data on these metrics, not just a list of additional
metrics the team is planning on tracking.  However, if all you have is a plan,
please share it here.}

Our team does not have any additional metrics of productivity aside from
the ones listed in this report.

\end{document}
