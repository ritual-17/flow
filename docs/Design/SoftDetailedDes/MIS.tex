\documentclass[12pt, titlepage]{article}

\usepackage{amsmath, mathtools}

\usepackage[round]{natbib}
\usepackage{amsfonts}
\usepackage{amssymb}
\usepackage{graphicx}
\usepackage{colortbl}
\usepackage{xr}
\usepackage{hyperref}
\usepackage{longtable}
\usepackage{xfrac}
\usepackage{tabularx}
\usepackage{float}
\usepackage{siunitx}
\usepackage{booktabs}
\usepackage{multirow}
\usepackage[section]{placeins}
\usepackage{caption}
\usepackage{fullpage}

\hypersetup{
bookmarks=true,     % show bookmarks bar?
colorlinks=true,       % false: boxed links; true: colored links
linkcolor=red,          % color of internal links (change box color with linkbordercolor)
citecolor=blue,      % color of links to bibliography
filecolor=magenta,  % color of file links
urlcolor=cyan          % color of external links
}

\usepackage{array}

\externaldocument{../../SRS/SRS}

\input{../../Comments}
%% Common Parts

\newcommand{\progname}{Flow} % PUT YOUR PROGRAM NAME HERE
\newcommand{\authname}{Team 9, Team Name
 Student 1 name
 Student 2 name
 Student 3 name
 Student 4 name} % AUTHOR NAMES                  

\usepackage{hyperref}
    \hypersetup{colorlinks=true, linkcolor=blue, citecolor=blue, filecolor=blue,
                urlcolor=blue, unicode=false}
    \urlstyle{same}
                                


\begin{document}

\title{Module Interface Specification for \progname{}}

\author{\authname}

\date{\today}

\maketitle

\pagenumbering{roman}

\section{Revision History}

\begin{tabularx}{\textwidth}{p{3cm}p{2cm}X}
\toprule {\bf Date} & {\bf Version} & {\bf Notes}\\
\midrule
Date 1 & 1.0 & Notes\\
Date 2 & 1.1 & Notes\\
\bottomrule
\end{tabularx}

~\newpage

\section{Symbols, Abbreviations and Acronyms}

See SRS Documentation at \url{https://github.com/ritual-17/flow/tree/main/docs/SRS-Meyer}

%\wss{Also add any additional symbols, abbreviations or acronyms}

\newpage

\tableofcontents

\newpage

\pagenumbering{arabic}

\section{Introduction}

The following document details the Module Interface Specifications for
\progname{}

Complementary documents include the System Requirement Specifications
and Module Guide.  The full documentation and implementation can be
found at \url{https://github.com/ritual-17/flow}.

\section{Notation}

%\wss{You should describe your notation.  You can use what is below as
%  a starting point.}
%
%The structure of the MIS for modules comes from \citet{HoffmanAndStrooper1995},
%with the addition that template modules have been adapted from
%\cite{GhezziEtAl2003}.  The mathematical notation comes from Chapter 3 of
%\citet{HoffmanAndStrooper1995}.  For instance, the symbol := is used for a
%multiple assignment statement and conditional rules follow the form $(c_1
%\Rightarrow r_1 | c_2 \Rightarrow r_2 | ... | c_n \Rightarrow r_n )$.
%
%The following table summarizes the primitive data types used by \progname. 
%
%\begin{center}
%\renewcommand{\arraystretch}{1.2}
%\noindent 
%\begin{tabular}{l l p{7.5cm}} 
%\toprule 
%\textbf{Data Type} & \textbf{Notation} & \textbf{Description}\\ 
%\midrule
%character & char & a single symbol or digit\\
%integer & $\mathbb{Z}$ & a number without a fractional component in (-$\infty$, $\infty$) \\
%natural number & $\mathbb{N}$ & a number without a fractional component in [1, $\infty$) \\
%real & $\mathbb{R}$ & any number in (-$\infty$, $\infty$)\\
%\bottomrule
%\end{tabular} 
%\end{center}

\noindent
The specification of \progname \ uses some derived data types: sequences, strings, and
tuples. Sequences are lists filled with elements of the same data type. Strings
are sequences of characters. Tuples contain a list of values, potentially of
different types. In addition, \progname \ uses functions, which
are defined by the data types of their inputs and outputs. Local functions are
described by giving their type signature followed by their specification.

\section{Module Decomposition}

%The following table is taken directly from the Module Guide document for this project.

\begin{table}[h!]
\centering
\begin{tabular}{p{0.3\textwidth} p{0.6\textwidth}}
\toprule
\textbf{Level 1} & \textbf{Level 2}\\
\midrule

{Hardware-Hiding} & \ref{Display Interface Module} {Display Interface Module}  \\
& \ref{Input Interface Module} {Input Interface Module}\\
& \ref{File Interface Module} {File Interface Module}\\
\midrule

\multirow{7}{0.3\textwidth}{Behaviour-Hiding} & \ref{Geometry State Parser Module} {Geometry State Parser Module}\\
& \ref{Geometry State Converter Module} {Geometry State Converter Module}\\
& \ref{Commands Parser Module} {Commands Parser Module}\\
& \ref{Mode Commands Module} {Mode Commands Module}\\
& \ref{Shape Interface Module} {Shape Interface Module}\\
& \ref{User Preference Module} {User Preference Module}\\
& \ref{Geometry State Parser Module} {Geometry State Parser Module}\\
\midrule

\multirow{3}{0.3\textwidth}{Software Decision} & \ref{Text Buffer Module} {Text Buffer Module}\\
& \ref{Geometry State Module} {Geometry State Module}\\
& \ref{Geometry State Mutator Module} {Geometry State Mutator Module}\\
& \ref{Undo Redo Module} {Undo Redo Module}\\
& \ref{User Persistence Module} {User Persistence Module}\\
\bottomrule

\end{tabular}
\caption{Module Hierarchy}
\label{TblMH}
\end{table}

\newpage
~\newpage

\section{MIS}} \label{Display Interface Module} 
%\wss{Use labels for  cross-referencing}

%\wss{You can reference SRS labels, such as R\ref{R_Inputs}.}

%\wss{It is also possible to use \LaTeX for hypperlinks to external documents.}

\subsection{Module}

%\wss{Short name for the module}
Display for the system 

This is the module that will handle the display for the system.
 This is a module that when called updates the display. 
\subsection{Uses}
\ref{Geometry State Module} {Geometry State Module}

\subsection{Syntax}

\subsubsection{Exported Constants}

N/A

\subsubsection{Exported Access Programs}

\begin{center}
\begin{tabular}{p{2cm} p{4cm} p{4cm} p{2cm}}
\hline
\textbf{Name} & \textbf{In} & \textbf{Out} & \textbf{Exceptions} \\
\hline
display & Shapes & Display to screen & - \\
\hline
\end{tabular}
\end{center}

\subsection{Semantics}

\subsubsection{State Variables}

N/A

\subsubsection{Environment Variables}

\begin{itemize}
	\item System Screen
	\item Rendering API
\end{itemize}
%\wss{This section is not necessary for all modules.  Its purpose is to capture  when the module has external interaction with the environment, such as for   device driver, screen interface, keyboard, file, etc.}

\subsubsection{Assumptions}

%\wss{Try to minimize assumptions and anticipate programmer errors via  exceptions, but for practical purposes assumptions are sometimes appropriate.}

\subsubsection{Access Routine Semantics}

\noindent {display}(Shapes):
\begin{itemize}

\item output: Outputs Shapes visually on the screen 
\end{itemize}

%\wss{A module without environment variables or state variables is unlikely to
%  have a state transition.  In this case a state transition can only occur if
%  the module is changing the state of another module.}

%\wss{Modules rarely have both a transition and an output.  In most cases you
%  will have one or the other.}

\subsubsection{Local Functions}

%\wss{As appropriate} \wss{These functions are for the purpose of specification.
%  They are not necessarily something that is going to be implemented
%  explicitly.  Even if they are implemented, they are not exported; they only
%  have local scope.}
N/A

\newpage
\section{MIS of {Input Interface Module}} \label{Input Interface Module} 
%\wss{Use labels for
%  cross-referencing}
%
%\wss{You can reference SRS labels, such as R\ref{R_Inputs}.}
%
%\wss{It is also possible to use \LaTeX for hypperlinks to external documents.}

\subsection{Module}

Input Module. Deals with User inputs and passes them on to the \ref{Mode Commands Module}{Mode Commands Module} 

\subsection{Uses}
\ref{Mode Commands Module} {Mode Commands Module} 
\subsection{Syntax}

\subsubsection{Exported Constants}



\subsubsection{Exported Access Programs}

\begin{center}
\begin{tabular}{p{2cm} p{4cm} p{4cm} p{2cm}}
\hline
\textbf{Name} & \textbf{In} & \textbf{Out} & \textbf{Exceptions} \\
\hline
 get\_keyboard\_inputs & - & keyboard\_inputs & - \\
\hline
 get\_mouse\_inputs & - & mouse\_inputs & - \\
\hline
\end{tabular}
\end{center}

\subsection{Semantics}

\subsubsection{State Variables}
 
\subsubsection{Environment Variables}

%\wss{This section is not necessary for all modules.  Its purpose is to capture
%  when the module has external interaction with the environment, such as for a
%  device driver, screen interface, keyboard, file, etc.}
\begin{itemize}
\item{Keyboard API}
\item{Mouse API}
\end{itemize}

\subsubsection{Assumptions}

%\wss{Try to minimize assumptions and anticipate programmer errors via
%  exceptions, but for practical purposes assumptions are sometimes appropriate.}

\subsubsection{Access Routine Semantics}

\noindent {get\_keyboard\_inputs}():
\begin{itemize}
\item output: Data Structure containing all current keyboard inputs. All keys will be off if the user is not using a keyboard. 
\end{itemize}

\noindent {get\_mouse\_inputs}():
\begin{itemize}
\item output: Data Structure containing mouse location and inputs. All inputs are off if the user does not have a mouse.
\end{itemize}

%\wss{A module without environment variables or state variables is unlikely to
%  have a state transition.  In this case a state transition can only occur if
%  the module is changing the state of another module.}
%
%\wss{Modules rarely have both a transition and an output.  In most cases you
%  will have one or the other.}

\subsubsection{Local Functions}

%\wss{As appropriate} \wss{These functions are for the purpose of specification.
%  They are not necessarily something that is going to be implemented
%  explicitly.  Even if they are implemented, they are not exported; they only
%  have local scope.}
N/A
\newpage
\section{MIS of {File Interface Module}} \label{File Interface Module} 
%\wss{Use labels for
%  cross-referencing}
%
%\wss{You can reference SRS labels, such as R\ref{R_Inputs}.}
%
%\wss{It is also possible to use \LaTeX for hypperlinks to external documents.}

\subsection{Module}

\ref{Geometry State Parser Module} {Geometry State Parser Module}

\subsection{Uses}

\ref{Mode Commands Module} {Mode Commands Module} 

\subsection{Syntax}

\subsubsection{Exported Constants}
N/A
\subsubsection{Exported Access Programs}

\begin{center}
\begin{tabular}{p{2cm} p{4cm} p{4cm} p{2cm}}
\hline
\textbf{Name} & \textbf{In} & \textbf{Out} & \textbf{Exceptions} \\
\hline
{open} & file\_path & opened\_file & Invalid File \\
\hline
{save} & file\_path, opened\_note & success & - \\
\hline
\end{tabular}
\end{center}

\subsection{Semantics}

\subsubsection{State Variables}

%\wss{Not all modules will have state variables.  State variables give the module
%  a memory.}
N/A

\subsubsection{Environment Variables}

%\wss{This section is not necessary for all modules.  Its purpose is to capture
%  when the module has external interaction with the environment, such as for a
%  device driver, screen interface, keyboard, file, etc.}
\begin{itemize}
\item File System 
\end{itemize}
\subsubsection{Assumptions}
N/A
%\wss{Try to minimize assumptions and anticipate programmer errors via
%  exceptions, but for practical purposes assumptions are sometimes appropriate.}

\subsubsection{Access Routine Semantics}

\noindent {open}(file\_path):
\begin{itemize}
\item output: The opened file 
\item exception: Invalid File if file is missing or of an incorrect type.
\end{itemize}
\noindent {save}(file\_path, opened\_note):

\begin{itemize}
\item output: If the Module was successful in saving opened\_note to file\_path

\end{itemize}
%\wss{A module without environment variables or state variables is unlikely to
%  have a state transition.  In this case a state transition can only occur if
%  the module is changing the state of another module.}
%
%\wss{Modules rarely have both a transition and an output.  In most cases you
%  will have one or the other.}

\subsubsection{Local Functions}
N/A
%\wss{As appropriate} \wss{These functions are for the purpose of specification.
%  They are not necessarily something that is going to be implemented
%  explicitly.  Even if they are implemented, they are not exported; they only
%  have local scope.}

\newpage
\section{MIS of {Geometry State Parser Module}} \label{Geometry State Parser Module} 
%\wss{Use labels for
%  cross-referencing}
%
%\wss{You can reference SRS labels, such as R\ref{R_Inputs}.}
%
%\wss{It is also possible to use \LaTeX for hypperlinks to external documents.}

\subsection{Module}

File parser takes the file input and parses it into data structure containing all shapes.
%\wss{Short name for the module}

\subsection{Uses}

N/A
\subsection{Syntax}

\subsubsection{Exported Constants}
N/A
\subsubsection{Exported Access Programs}

\begin{center}
\begin{tabular}{p{2cm} p{4cm} p{4cm} p{2cm}}
\hline
\textbf{Name} & \textbf{In} & \textbf{Out} & \textbf{Exceptions} \\
\hline
{parse} & opened\_file & shapes & Invalid File \\
\hline
\end{tabular}
\end{center}

\subsection{Semantics}

\subsubsection{State Variables}
N/A

\subsubsection{Environment Variables}
%
%\wss{This section is not necessary for all modules.  Its purpose is to capture
%  when the module has external interaction with the environment, such as for a
%  device driver, screen interface, keyboard, file, etc.}
N/A

\subsubsection{Assumptions}
N/A
%\wss{Try to minimize assumptions and anticipate programmer errors via
%  exceptions, but for practical purposes assumptions are sometimes appropriate.}

\subsubsection{Access Routine Semantics}

\noindent {parse}(opened\_file):
\begin{itemize}
\item output: Data structure containing all shapes from the given file. 
\item exception: Invalid File: if the file given is able to be parsed.
\end{itemize}

%\wss{A module without environment variables or state variables is unlikely to
%  have a state transition.  In this case a state transition can only occur if
%  the module is changing the state of another module.}
%
%\wss{Modules rarely have both a transition and an output.  In most cases you
%  will have one or the other.}

\subsubsection{Local Functions}

%\wss{As appropriate} \wss{These functions are for the purpose of specification.
%  They are not necessarily something that is going to be implemented
%  explicitly.  Even if they are implemented, they are not exported; they only
%  have local scope.}

\newpage
\section{MIS of {Geometry State Converter Module}} \label{Geometry State Converter Module} 
%\wss{Use labels for
%  cross-referencing}
%
%\wss{You can reference SRS labels, such as R\ref{R_Inputs}.}
%
%\wss{It is also possible to use \LaTeX for hypperlinks to external documents.}

\subsection{Module}

Note converter that converts notes to files of different file types.
%\wss{Short name for the module}

\subsection{Uses}
N/A

\subsection{Syntax}

\subsubsection{Exported Constants}
N/A
\subsubsection{Exported Access Programs}

\begin{center}
\begin{tabular}{p{2cm} p{4cm} p{4cm} p{2cm}}
\hline
\textbf{Name} & \textbf{In} & \textbf{Out} & \textbf{Exceptions} \\
\hline
{convert} & opened\_note, file\_type & file & Invalid file\_type \\
\hline
\end{tabular}
\end{center}

\subsection{Semantics}

\subsubsection{State Variables}

%\wss{Not all modules will have state variables.  State variables give the module
%  a memory.}

\subsubsection{Environment Variables}

%\wss{This section is not necessary for all modules.  Its purpose is to capture
%  when the module has external interaction with the environment, such as for a
%  device driver, screen interface, keyboard, file, etc.}

\subsubsection{Assumptions}

%\wss{Try to minimize assumptions and anticipate programmer errors via
%  exceptions, but for practical purposes assumptions are sometimes appropriate.}

\subsubsection{Access Routine Semantics}

\noindent {convert}(opened\_note, file\_type):
\begin{itemize} 
\item output: file that is ready to be saved on the computer.
\item exception: Invalid file\_type if the given file type is not compatable with the ones implemented
\end{itemize}

%\wss{A module without environment variables or state variables is unlikely to
%  have a state transition.  In this case a state transition can only occur if
%  the module is changing the state of another module.}
%
%\wss{Modules rarely have both a transition and an output.  In most cases you
%  will have one or the other.}

\subsubsection{Local Functions}

%\wss{As appropriate} \wss{These functions are for the purpose of specification.
%  They are not necessarily something that is going to be implemented
%  explicitly.  Even if they are implemented, they are not exported; they only
%  have local scope.}
convert\_*(opened\_note) functions that do the converting once the file type has been determined. (ex convert\_pdf)

\newpage
\section{MIS of {Commands Parser Module}} \label{Commands Parser Module} 
%\wss{Use labels for
%  cross-referencing}
%
%\wss{You can reference SRS labels, such as R\ref{R_Inputs}.}
%
%\wss{It is also possible to use \LaTeX for hypperlinks to external documents.}

\subsection{Module}

%\wss{Short name for the module}
Command Parser, converts keyboard inputs into commands and passes them onto the shape mutator. 

\subsection{Uses}
\ref{Geometry State Mutator Module}{Geometry State Mutator Module}
\ref{User Preference Module} {User Preference Module}
\subsection{Syntax}

\subsubsection{Exported Constants}
N/A

\subsubsection{Exported Access Programs}

\begin{center}
\begin{tabular}{p{3cm} p{4cm} p{4cm} p{2cm}}
\hline
\textbf{Name} & \textbf{In} & \textbf{Out} & \textbf{Exceptions} \\
\hline
{parse\_commands} & keyboard\_inputs, mouse\_inputs & commands & - \\
\hline
\end{tabular}
\end{center}

\subsection{Semantics}

\subsubsection{State Variables}
N/A
%\wss{Not all modules will have state variables.  State variables give the module
%  a memory.}

\subsubsection{Environment Variables}
N/A
%\wss{This section is not necessary for all modules.  Its purpose is to capture
%  when the module has external interaction with the environment, such as for a
%  device driver, screen interface, keyboard, file, etc.}

\subsubsection{Assumptions}
Inputs have been already cleaned by the other modules. 
%\wss{Try to minimize assumptions and anticipate programmer errors via
%  exceptions, but for practical purposes assumptions are sometimes appropriate.}

\subsubsection{Access Routine Semantics}

\noindent {parse\_commands}(keyboard\_inputs, mouse\_inputs):
\begin{itemize}
\item output: commands, which would be used by the Geometry State Mutator Module \ref{Geometry State Mutator Module}
\end{itemize}

%\wss{A module without environment variables or state variables is unlikely to
%  have a state transition.  In this case a state transition can only occur if
%  the module is changing the state of another module.}
%
%\wss{Modules rarely have both a transition and an output.  In most cases you
%  will have one or the other.}

\subsubsection{Local Functions}

%\wss{As appropriate} \wss{These functions are for the purpose of specification.
%  They are not necessarily something that is going to be implemented
%  explicitly.  Even if they are implemented, they are not exported; they only
%  have local scope.}
N/A

\newpage
\section{MIS of {Mode Commands Module}} \label{Mode Commands Module} 
%\wss{Use labels for
%  cross-referencing}
%
%\wss{You can reference SRS labels, such as R\ref{R_Inputs}.}
%
%\wss{It is also possible to use \LaTeX for hypperlinks to external documents.}

\subsection{Module}

Main Module contains the commands usable in the current mode. 

\subsection{Uses}
N/A
\subsection{Syntax}

\subsubsection{Exported Constants}
N/A
\subsubsection{Exported Access Programs}

\begin{center}
\begin{tabular}{p{3cm} p{4cm} p{4cm} p{2cm}}
\hline
\textbf{Name} & \textbf{In} & \textbf{Out} & \textbf{Exceptions} \\
\hline
{get\_commands} & mode & commands & - \\
\hline
\end{tabular}
\end{center}

\subsection{Semantics}

\subsubsection{State Variables}

%\wss{Not all modules will have state variables.  State variables give the module
%  a memory.}
N/A
\subsubsection{Environment Variables}
N/A
%\wss{This section is not necessary for all modules.  Its purpose is to capture
%  when the module has external interaction with the environment, such as for a
%  device driver, screen interface, keyboard, file, etc.}

\subsubsection{Assumptions}
Mode given will always be a valid mode.
%\wss{Try to minimize assumptions and anticipate programmer errors via
%  exceptions, but for practical purposes assumptions are sometimes appropriate.}

\subsubsection{Access Routine Semantics}

\noindent {get\_commands}(mode):
\begin{itemize}
\item output: commands that can be run in the given mode 
\end{itemize}
%
%\wss{A module without environment variables or state variables is unlikely to
%  have a state transition.  In this case a state transition can only occur if
%  the module is changing the state of another module.}
%
%\wss{Modules rarely have both a transition and an output.  In most cases you
%  will have one or the other.}

\subsubsection{Local Functions}
N/A
%\wss{As appropriate} \wss{These functions are for the purpose of specification.
%  They are not necessarily something that is going to be implemented
%  explicitly.  Even if they are implemented, they are not exported; they only
%  have local scope.}

\newpage\section{MIS of {Shape Interface Module}} \label{Shape Interface Module} 
%\wss{Use labels for
%  cross-referencing}
%
%\wss{You can reference SRS labels, such as R\ref{R_Inputs}.}
%
%\wss{It is also possible to use \LaTeX for hypperlinks to external documents.}

\subsection{Module}
Module that contains information about shapes.

\subsection{Uses}


\subsection{Syntax}

\subsubsection{Exported Constants}
Shapes: List of shapes

\subsubsection{Exported Access Programs}

\begin{center}
\begin{tabular}{p{2cm} p{4cm} p{4cm} p{2cm}}
\hline
\textbf{Name} & \textbf{In} & \textbf{Out} & \textbf{Exceptions} \\
\hline
{create\_shape} & shape\_type, points & shape & Invalid Points \\
\hline
{move\_shape} & shape, new\_points & - & Invalid Points \\
\hline
\end{tabular}
\end{center}

\subsection{Semantics}

\subsubsection{State Variables}

%\wss{Not all modules will have state variables.  State variables give the module
%  a memory.}
N/A
\subsubsection{Environment Variables}
%
%\wss{This section is not necessary for all modules.  Its purpose is to capture
%  when the module has external interaction with the environment, such as for a
%  device driver, screen interface, keyboard, file, etc.}
N/A
\subsubsection{Assumptions}
%
%\wss{Try to minimize assumptions and anticipate programmer errors via
%  exceptions, but for practical purposes assumptions are sometimes appropriate.}
N/A
\subsubsection{Access Routine Semantics}

\noindent {create\_shape}(shape\_type, points ):
\begin{itemize}
\item output: Shape with type shape\_type using points for location and size. 
\item exception: Invalid points if the points are not possible for the type of shape.  
\end{itemize}

\noindent {move\_shape}():
\begin{itemize}
\item transition: changes the points of the shape to the new points
\item exception: Invalid points if the points are not possible for the type of shape.
\end{itemize}

%\wss{A module without environment variables or state variables is unlikely to
%  have a state transition.  In this case a state transition can only occur if
%  the module is changing the state of another module.}
%
%\wss{Modules rarely have both a transition and an output.  In most cases you
%  will have one or the other.}

\subsubsection{Local Functions}
N/A
%\wss{As appropriate} \wss{These functions are for the purpose of specification.
%  They are not necessarily something that is going to be implemented
%  explicitly.  Even if they are implemented, they are not exported; they only
%  have local scope.}

\newpage
\section{MIS of {User Preference Module}} \label{User Preference Module} 
%\wss{Use labels for
%  cross-referencing}
%
%\wss{You can reference SRS labels, such as R\ref{R_Inputs}.}
%
%\wss{It is also possible to use \LaTeX for hypperlinks to external documents.}

\subsection{Module}
User preferences, provides an interface for creating and manipulating user preferences in the system.
%\wss{Short name for the module}

\subsection{Uses}
\ref{Display Interface Module} {Display Interface Module}
\ref{User Persistence Module} {User Persistence Module}
\subsection{Syntax}

\subsubsection{Exported Constants}
N/A
\subsubsection{Exported Access Programs}

\begin{center}
\begin{tabular}{p{2cm} p{4cm} p{4cm} p{2cm}}
\hline
\textbf{Name} & \textbf{In} & \textbf{Out} & \textbf{Exceptions} \\
\hline
{get\_command\_prefrences} & mode & commands & - \\
\hline
{modify} & - & - & - \\
\hline
{get\_theme\_prefrences} & - & theme & - \\
\hline
\end{tabular}
\end{center}

\subsection{Semantics}

\subsubsection{State Variables}
Command preferences: users defined commands
Theme preferences: user theme prefrences
%\wss{Not all modules will have state variables.  State variables give the module
%  a memory.}

\subsubsection{Environment Variables}
%
%\wss{This section is not necessary for all modules.  Its purpose is to capture
%  when the module has external interaction with the environment, such as for a
%  device driver, screen interface, keyboard, file, etc.}

\subsubsection{Assumptions}
%
%\wss{Try to minimize assumptions and anticipate programmer errors via
%  exceptions, but for practical purposes assumptions are sometimes appropriate.}

\subsubsection{Access Routine Semantics}

\noindent {get\_command\_prefrences}(mode):
\begin{itemize}
\item output: Any user defined preferences for the current mode
\end{itemize}

\noindent {modify}():
\begin{itemize}
\item transition: Has the user input their preferences where the moudles saves it to a file. 
\end{itemize}

\noindent {get\_theme\_prefrences}():
\begin{itemize}
\item output: Theme preferences for the system
\end{itemize}

%\wss{A module without environment variables or state variables is unlikely to
%  have a state transition.  In this case a state transition can only occur if
%  the module is changing the state of another module.}
%
%\wss{Modules rarely have both a transition and an output.  In most cases you
%  will have one or the other.}

\subsubsection{Local Functions}

%\wss{As appropriate} \wss{These functions are for the purpose of specification.
%  They are not necessarily something that is going to be implemented
%  explicitly.  Even if they are implemented, they are not exported; they only
%  have local scope.}

\newpage
\section{MIS of {Text Buffer Module}} \label{Text Buffer Module} 
%\wss{Use labels for
%  cross-referencing}
%
%\wss{You can reference SRS labels, such as R\ref{R_Inputs}.}
%
%\wss{It is also possible to use \LaTeX for hypperlinks to external documents.}

\subsection{Module}
Text, Module for Text representation in notes. 
%\wss{Short name for the module}

\subsection{Uses}
N/A

\subsection{Syntax}

\subsubsection{Exported Constants}

\subsubsection{Exported Access Programs}

\begin{center}
\begin{tabular}{p{2cm} p{6cm} p{4cm} p{2cm}}
\hline
\textbf{Name} & \textbf{In} & \textbf{Out} & \textbf{Exceptions} \\
\hline
{new\_text} & location,text,formatting & text\_box & Invalid Text, Invalid Location \\
\hline
{modify\_text} & text\_box,text,formatting & - & Invalid Text  \\
\hline
{move\_text} & text\_box,location & - & Invalid Location \\
\hline
\end{tabular}
\end{center}

\subsection{Semantics}

\subsubsection{State Variables}

%\wss{Not all modules will have state variables.  State variables give the module
%  a memory.}
N/A
\subsubsection{Environment Variables}
N/A
%\wss{This section is not necessary for all modules.  Its purpose is to capture
%  when the module has external interaction with the environment, such as for a
%  device driver, screen interface, keyboard, file, etc.}

\subsubsection{Assumptions}

%\wss{Try to minimize assumptions and anticipate programmer errors via
%  exceptions, but for practical purposes assumptions are sometimes appropriate.}

\subsubsection{Access Routine Semantics}

\noindent {new\_text}(location,text,formatting):
\begin{itemize}
\item output: Text box at location with text and formatting
\item exception: Invalid Text, Invalid Location
\end{itemize}

\noindent {modify\_text}(text\_box,text,formatting):
\begin{itemize}
\item transition: Text in text\_box is modified to now contain text with formatting
\item exception: Invalid Text
\end{itemize}

\noindent {move\_text}(text\_box,location):
\begin{itemize}
\item transition: text\_box is moved to location. 
\item exception: Invalid Location
\end{itemize}

%\wss{A module without environment variables or state variables is unlikely to
%  have a state transition.  In this case a state transition can only occur if
%  the module is changing the state of another module.}
%
%\wss{Modules rarely have both a transition and an output.  In most cases you
%  will have one or the other.}

\subsubsection{Local Functions}

N/A
%\wss{As appropriate} \wss{These functions are for the purpose of specification.
%  They are not necessarily something that is going to be implemented
%  explicitly.  Even if they are implemented, they are not exported; they only
%  have local scope.}

\newpage
\section{MIS of {Geometry State Module}} \label{Geometry State Module} 
%\wss{Use labels for
%  cross-referencing}
%
%\wss{You can reference SRS labels, such as R\ref{R_Inputs}.}
%
%\wss{It is also possible to use \LaTeX for hypperlinks to external documents.}

\subsection{Module}

Geometry State, contains info on all current geometry 
%\wss{Short name for the module}

\subsection{Uses}
\ref{Shape Interface Module} {Shape Interface Module}
\ref{Text Buffer Module} {Text Buffer Module}

\subsection{Syntax}

\subsubsection{Exported Constants}

\subsubsection{Exported Access Programs}

\begin{center}
\begin{tabular}{p{2cm} p{4cm} p{4cm} p{2cm}}
\hline
\textbf{Name} & \textbf{In} & \textbf{Out} & \textbf{Exceptions} \\
\hline
{get\_shapes} & - & shapes & - \\
\hline
{get\_text} & - & text\_boxes & - \\
\hline
\end{tabular}
\end{center}

\subsection{Semantics}

\subsubsection{State Variables}
Shapes: data structure holding all shapes
Text\_Boxes: data structure holding all Text Boxes
%\wss{Not all modules will have state variables.  State variables give the module
%  a memory.}

\subsubsection{Environment Variables}
%
%\wss{This section is not necessary for all modules.  Its purpose is to capture
%  when the module has external interaction with the environment, such as for a
%  device driver, screen interface, keyboard, file, etc.}
N/A
\subsubsection{Assumptions}
N/A
%\wss{Try to minimize assumptions and anticipate programmer errors via
%  exceptions, but for practical purposes assumptions are sometimes appropriate.}

\subsubsection{Access Routine Semantics}

\noindent {get\_shapes}():
\begin{itemize}
\item output: Data Structure containing all shapes
\end{itemize}

\noindent {get\_text}():
\begin{itemize}
\item output:  Data structure containing all text boxes
\end{itemize}





%\wss{A module without environment variables or state variables is unlikely to
%  have a state transition.  In this case a state transition can only occur if
%  the module is changing the state of another module.}
%
%\wss{Modules rarely have both a transition and an output.  In most cases you
%  will have one or the other.}

\subsubsection{Local Functions}
N/A
%\wss{As appropriate} \wss{These functions are for the purpose of specification.
%  They are not necessarily something that is going to be implemented
%  explicitly.  Even if they are implemented, they are not exported; they only
%  have local scope.}

\newpage
\section{MIS of {Geometry State Mutator Module}} \label{Geometry State Mutator Module} 
%\wss{Use labels for
%  cross-referencing}
%
%\wss{You can reference SRS labels, such as R\ref{R_Inputs}.}
%
%\wss{It is also possible to use \LaTeX for hypperlinks to external documents.}

\subsection{Module}
Geometry State Mutator Module, Provides methods to mutate the geometry state (add, delete, modify shapes).

%\wss{Short name for the module}

\subsection{Uses}
\ref{Commands Parser Module} {Commands Parser Module}
\ref{Shape Interface Module} {Shape Interface Module}
\ref{Text Buffer Module} {Text Buffer Module}
\ref{Geometry State Module} {Geometry State Module}
\subsection{Syntax}

\subsubsection{Exported Constants}
N/A
\subsubsection{Exported Access Programs}

\begin{center}
\begin{tabular}{p{2cm} p{4cm} p{4cm} p{2cm}}
\hline
\textbf{Name} & \textbf{In} & \textbf{Out} & \textbf{Exceptions} \\
\hline
{delete\_text} & text\_box & - & - \\
\hline
{delete\_shapes} & shape & - & - \\
\hline
{add\_text} & text\_box & - & - \\
\hline
{add\_shape} & shape & - & - \\
\hline
{modify\_text} & text\_box,text,formatting & - & Invalid Text  \\
\hline
{move\_shape} & shape, new\_points & - & Invalid Points \\
\hline
\end{tabular}
\end{center}

\subsection{Semantics}

\subsubsection{State Variables}
Geometry\_State\_Module:Stores the info on the shapes and Text boxes
%\wss{Not all modules will have state variables.  State variables give the module
%  a memory.}

\subsubsection{Environment Variables}
N/A
%\wss{This section is not necessary for all modules.  Its purpose is to capture
%  when the module has external interaction with the environment, such as for a
%  device driver, screen interface, keyboard, file, etc.}

\subsubsection{Assumptions}
N/A
%\wss{Try to minimize assumptions and anticipate programmer errors via
%  exceptions, but for practical purposes assumptions are sometimes appropriate.}

\subsubsection{Access Routine Semantics}

\noindent {add\_text}(text\_box):
\begin{itemize}
\item transition: Adds the text\_box to Text\_Boxes
\end{itemize}

\noindent {add\_shape}(shape):
\begin{itemize}
\item transition: Adds the shape to Shapes 
\end{itemize}

\noindent {delete\_text}(text\_box):
\begin{itemize}
\item transition: Removes the text\_box to Text\_Boxes
\end{itemize}


\noindent {delete\_shape}(shape):
\begin{itemize}
\item transition: Removes the shape to Shapes 
\end{itemize}

\noindent {modify\_text}(text\_box,text,formatting):
\begin{itemize}
\item transition: Text in text\_box is modified to now contain text with formatting
\item exception: Invalid Text
\end{itemize}

\noindent {move\_shape}():
\begin{itemize}
\item transition: changes the points of the shape to the new points
\item exception: Invalid points if the points are not possible for the type of shape.
\end{itemize}

%\wss{A module without environment variables or state variables is unlikely to
%  have a state transition.  In this case a state transition can only occur if
%  the module is changing the state of another module.}
%
%\wss{Modules rarely have both a transition and an output.  In most cases you
%  will have one or the other.}

\subsubsection{Local Functions}
N/A
%\wss{As appropriate} \wss{These functions are for the purpose of specification.
%  They are not necessarily something that is going to be implemented
%  explicitly.  Even if they are implemented, they are not exported; they only
%  have local scope.}

\newpage
\section{MIS of {Undo Redo Module}} \label{Undo Redo Module} 
%\wss{Use labels for
%  cross-referencing}
%
%\wss{You can reference SRS labels, such as R\ref{R_Inputs}.}
%
%\wss{It is also possible to use \LaTeX for hypperlinks to external documents.}

\subsection{Module}
Undo/Redo Module: manages undo and redo commands.
%\wss{Short name for the module}

\subsection{Uses}
\ref{Commands Parser Module} {Commands Parser Module}
\ref{Geometry State Mutator Module} {Geometry State Mutator Module}
\ref{File Interface Module} {File Interface Module} 
\subsection{Syntax}

\subsubsection{Exported Constants}
N/A
\subsubsection{Exported Access Programs}

\begin{center}
\begin{tabular}{p{2cm} p{4cm} p{4cm} p{2cm}}
\hline
\textbf{Name} & \textbf{In} & \textbf{Out} & \textbf{Exceptions} \\
\hline
{undo} & - & - & Cant Undo \\
\hline
{redo} & - & - & Cant Redo \\
\hline
{add\_command} & command & - & - \\
\hline
\end{tabular}
\end{center}

\subsection{Semantics}

\subsubsection{State Variables}
Previous\_Commands: previous command that can be used for undo
Redone\_Commands: commands that can be used for redo
%\wss{Not all modules will have state variables.  State variables give the module
%  a memory.}

\subsubsection{Environment Variables}

%\wss{This section is not necessary for all modules.  Its purpose is to capture
%  when the module has external interaction with the environment, such as for a
%  device driver, screen interface, keyboard, file, etc.}
N/A
\subsubsection{Assumptions}
N/A
%\wss{Try to minimize assumptions and anticipate programmer errors via
%  exceptions, but for practical purposes assumptions are sometimes appropriate.}

\subsubsection{Access Routine Semantics}

\noindent {undo}():
\begin{itemize}
\item transition: Uses the Geometry State mutator to undo a previous command.
\end{itemize}

\noindent {redo}():
\begin{itemize}
\item transition: Uses the Geometry State mutator to redo a previous command.
\end{itemize}

\noindent {redo}():
\begin{itemize}
\item transition: Uses the Geometry State mutator to redo a previous command.
\end{itemize}

\noindent {add\_command}(command):
\begin{itemize}
\item transition: Stores the command to use later in case of an undo.
\end{itemize}

%\wss{A module without environment variables or state variables is unlikely to
%  have a state transition.  In this case a state transition can only occur if
%  the module is changing the state of another module.}
%
%\wss{Modules rarely have both a transition and an output.  In most cases you
%  will have one or the other.}

\subsubsection{Local Functions}
N/A
%\wss{As appropriate} \wss{These functions are for the purpose of specification.
%  They are not necessarily something that is going to be implemented
%  explicitly.  Even if they are implemented, they are not exported; they only
%  have local scope.}

\newpage
\section{MIS of {User Persistence Module}} \label{User Persistence Module} 
%\wss{Use labels for
%  cross-referencing}

%\wss{You can reference SRS labels, such as R\ref{R_Inputs}.}
%
%\wss{It is also possible to use \LaTeX for hypperlinks to external documents.}

\subsection{Module}
User Persistence Module: Module for saving and loading user preferences and short-cuts.
%\wss{Short name for the module}

\subsection{Uses}


\subsection{Syntax}

\subsubsection{Exported Constants}
N/A
\subsubsection{Exported Access Programs}

\begin{center}
\begin{tabular}{p{2cm} p{4cm} p{4cm} p{2cm}}
\hline
\textbf{Name} & \textbf{In} & \textbf{Out} & \textbf{Exceptions} \\
\hline
{save} & commands,theme & - & - \\
\hline
{get\_commands} & - & commands & - \\
\hline
{get\_theme} & - & theme & - \\
\hline
\end{tabular}
\end{center}

\subsection{Semantics}

\subsubsection{State Variables}
N/A
%\wss{Not all modules will have state variables.  State variables give the module
%  a memory.}

\subsubsection{Environment Variables}
N/A

%\wss{This section is not necessary for all modules.  Its purpose is to capture
%  when the module has external interaction with the environment, such as for a
%  device driver, screen interface, keyboard, file, etc.}

\subsubsection{Assumptions}
N/A
%\wss{Try to minimize assumptions and anticipate programmer errors via
%  exceptions, but for practical purposes assumptions are sometimes appropriate.}

\subsubsection{Access Routine Semantics}

\noindent {save}(commands,theme):
\begin{itemize}
\item transition: Saves the commands and theme as a file
\end{itemize}

\noindent {get\_commands}():
\begin{itemize}
\item output: User defined commands read from the file
\end{itemize}

\noindent {get\_theme}():
\begin{itemize}
\item output: User preferred themes read from the file
\end{itemize}

%\wss{A module without environment variables or state variables is unlikely to
%  have a state transition.  In this case a state transition can only occur if
%  the module is changing the state of another module.}
%
%\wss{Modules rarely have both a transition and an output.  In most cases you
%  will have one or the other.}

\subsubsection{Local Functions}

%\wss{As appropriate} \wss{These functions are for the purpose of specification.
%  They are not necessarily something that is going to be implemented
%  explicitly.  Even if they are implemented, they are not exported; they only
%  have local scope.}
N/A
\newpage


\bibliographystyle {plainnat}
\bibliography {../../../refs/References}

\newpage

\section{Appendix} \label{Appendix}

%\wss{Extra information if required}

\newpage{}

\section*{Appendix --- Reflection}

\wss{Not required for CAS 741 projects}

The information in this section will be used to evaluate the team members on the
graduate attribute of Problem Analysis and Design.

\input{../../Reflection.tex}

\begin{enumerate}
  \item What went well while writing this deliverable? 
  \item What pain points did you experience during this deliverable, and how
    did you resolve them?
  \item Which of your design decisions stemmed from speaking to your client(s)
  or a proxy (e.g. your peers, stakeholders, potential users)? For those that
  were not, why, and where did they come from?
  \item While creating the design doc, what parts of your other documents (e.g.
  requirements, hazard analysis, etc), it any, needed to be changed, and why?
  \item What are the limitations of your solution?  Put another way, given
  unlimited resources, what could you do to make the project better? (LO\_ProbSolutions)
  \item Give a brief overview of other design solutions you considered.  What
  are the benefits and tradeoffs of those other designs compared with the chosen
  design?  From all the potential options, why did you select the documented design?
  (LO\_Explores)
\end{enumerate}


\end{document}