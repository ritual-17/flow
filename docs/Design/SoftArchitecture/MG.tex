\documentclass[12pt, titlepage]{article}

\usepackage{fullpage}
\usepackage[round]{natbib}
\usepackage{multirow}
\usepackage{booktabs}
\usepackage{tabularx}
\usepackage{graphicx}
\usepackage{float}
\usepackage{hyperref}
\hypersetup{
    colorlinks,
    citecolor=blue,
    filecolor=black,
    linkcolor=red,
    urlcolor=blue
}

\input{../../Comments}
%% Common Parts

\newcommand{\progname}{Flow} % PUT YOUR PROGRAM NAME HERE
\newcommand{\authname}{Team 9, Team Name
 Student 1 name
 Student 2 name
 Student 3 name
 Student 4 name} % AUTHOR NAMES                  

\usepackage{hyperref}
    \hypersetup{colorlinks=true, linkcolor=blue, citecolor=blue, filecolor=blue,
                urlcolor=blue, unicode=false}
    \urlstyle{same}
                                


\newcounter{acnum}
\newcommand{\actheacnum}{AC\theacnum}
\newcommand{\acref}[1]{AC\ref{#1}}

\newcounter{ucnum}
\newcommand{\uctheucnum}{UC\theucnum}
\newcommand{\uref}[1]{UC\ref{#1}}

\newcounter{mnum}
\newcommand{\mthemnum}{M\themnum}
\newcommand{\mref}[1]{M\ref{#1}}

\begin{document}

\title{Module Guide for \progname{}}
\author{\authname}
\date{\today}

\maketitle

\pagenumbering{roman}

\section{Revision History}

\begin{tabularx}{\textwidth}{p{3cm}p{2cm}X}
\toprule {\bf Date} & {\bf Version} & {\bf Notes}\\
\midrule
Date 1 & 1.0 & Notes\\
Date 2 & 1.1 & Notes\\
\bottomrule
\end{tabularx}

\newpage

\section{Reference Material}

This section records information for easy reference.

\subsection{Abbreviations and Acronyms}

\renewcommand{\arraystretch}{1.2}
\begin{tabular}{l l}
  \toprule
  \textbf{symbol} & \textbf{description}\\
  \midrule
  AC & Anticipated Change\\
  DAG & Directed Acyclic Graph \\
  M & Module \\
  MG & Module Guide \\
  OS & Operating System \\
  R & Requirement\\
  SC & Scientific Computing \\
  SRS & Software Requirements Specification\\
  \progname & Explanation of program name\\
  UC & Unlikely Change \\
  \wss{etc.} & \wss{...}\\
  \bottomrule
\end{tabular}\\

\newpage

\tableofcontents

\listoftables

\listoffigures

\newpage

\pagenumbering{arabic}

\section{Introduction}

Decomposing a system into modules is a commonly accepted approach to developing
software.  A module is a work assignment for a programmer or programming
team~\citep{ParnasEtAl1984}.  We advocate a decomposition
based on the principle of information hiding~\citep{Parnas1972a}.  This
principle supports design for change, because the ``secrets'' that each module
hides represent likely future changes.  Design for change is valuable in SC,
where modifications are frequent, especially during initial development as the
solution space is explored.

Our design follows the rules layed out by \citet{ParnasEtAl1984}, as follows:
\begin{itemize}
\item System details that are likely to change independently should be the
  secrets of separate modules.
\item Each data structure is implemented in only one module.
\item Any other program that requires information stored in a module's data
  structures must obtain it by calling access programs belonging to that module.
\end{itemize}

After completing the first stage of the design, the Software Requirements
Specification (SRS), the Module Guide (MG) is developed~\citep{ParnasEtAl1984}. The MG
specifies the modular structure of the system and is intended to allow both
designers and maintainers to easily identify the parts of the software.  The
potential readers of this document are as follows:

\begin{itemize}
\item New project members: This document can be a guide for a new project member
  to easily understand the overall structure and quickly find the
  relevant modules they are searching for.
\item Maintainers: The hierarchical structure of the module guide improves the
  maintainers' understanding when they need to make changes to the system. It is
  important for a maintainer to update the relevant sections of the document
  after changes have been made.
\item Designers: Once the module guide has been written, it can be used to
  check for consistency, feasibility, and flexibility. Designers can verify the
  system in various ways, such as consistency among modules, feasibility of the
  decomposition, and flexibility of the design.
\end{itemize}

The rest of the document is organized as follows. Section
\ref{SecChange} lists the anticipated and unlikely changes of the software
requirements. Section \ref{SecMH} summarizes the module decomposition that
was constructed according to the likely changes. Section \ref{SecConnection}
specifies the connections between the software requirements and the
modules. Section \ref{SecMD} gives a detailed description of the
modules. Section \ref{SecTM} includes two traceability matrices. One checks
the completeness of the design against the requirements provided in the SRS. The
other shows the relation between anticipated changes and the modules. Section
\ref{SecUse} describes the use relation between modules.

\section{Anticipated and Unlikely Changes} \label{SecChange}

This section lists possible changes to the system. According to the likeliness
of the change, the possible changes are classified into two
categories. Anticipated changes are listed in Section \ref{SecAchange}, and
unlikely changes are listed in Section \ref{SecUchange}.

\subsection{Anticipated Changes} \label{SecAchange}

Anticipated changes are the source of the information that is to be hidden
inside the modules. Ideally, changing one of the anticipated changes will only
require changing the one module that hides the associated decision. The approach
adapted here is called design for
change.

\begin{description}
\item[\refstepcounter{acnum} \actheacnum \label{acOS}:] The specific operating
  system environment on which the software is running.
\item[\refstepcounter{acnum} \actheacnum \label{asFramwork}:] The UI framework
  or technology used for the display of \progname{}.
\item[\refstepcounter{acnum} \actheacnum \label{acFileFormat}:] The format used for
  saving and loading the note files and diagrams (e.g., XML, JSON, binary).
\item [\refstepcounter{acnum} \actheacnum \label{acSyntax}:] The syntax used for
  editing and navigating the note and diagrams.
\item [\refstepcounter{acnum} \actheacnum \label{{acGeometry}:}] The internal
  representation of geometric shapes and diagrams.
\item [\refstepcounter{acnum} \actheacnum \label{acShapeManipulation}:] The
  rules as well as algorithms used for manipulating shapes (resizing, rotating,
  grouping, etc).
\item [\refstepcounter{acnum} \actheacnum \label{acPreferences}:] The user
  preferences and configuration options available for the user to customize.
\item [\refstepcounter{acnum} \actheacnum \label{acTextFormatting}:] The text
  formatting options available (e.g., font size, color, style).
\item [\refstepcounter{acnum} \actheacnum \label{acImportExport}:] The formats
  and options available for importing or exporting notes (e.g., PDF, image formats).
\item [\refstepcounter{acnum} \actheacnum \label{acUndoRedo}:] The implementation
  of the undo/redo functionality.
\item [\refstepcounter{acnum} \actheacnum \label{acPersistence}:] The method used
  for saving user data and session state.
\item [\refstepcounter{acnum} \actheacnum \label{acCanvasState}:] The implementation
  of how the state of the canvas will be captured and restored.
\end{description}

% \wss{Anticipated changes relate to changes that would be made in requirements,
% design or implementation choices.  They are not related to changes that are made
% at run-time, like the values of parameters.}

\subsection{Unlikely Changes} \label{SecUchange}

The module design should be as general as possible. However, a general system is
more complex. Sometimes this complexity is not necessary. Fixing some design
decisions at the system architecture stage can simplify the software design. If
these decision should later need to be changed, then many parts of the design
will potentially need to be modified. Hence, it is not intended that these
decisions will be changed.

\begin{description}
\item[\refstepcounter{ucnum} \uctheucnum \label{ucIO}:] Input/Output devices
  (Input: File and/or keyboard, Output: Screen display and file storage).
\item [\refstepcounter{ucnum} \uctheucnum \label{ucPlatform}:] The target platform
  will be desktop computers (Windows, MacOS, Linux).
\item [\refstepcounter{ucnum} \uctheucnum \label{ucStructure}:] The note structure
  will remain a combination of text and geometric shapes.
\end{description}

\section{Module Hierarchy} \label{SecMH}

This section provides an overview of the module design. Modules are summarized
in a hierarchy decomposed by secrets in Table \ref{TblMH}. The modules listed
below, which are leaves in the hierarchy tree, are the modules that will
actually be implemented.

\begin{description}
\item [\refstepcounter{mnum} \mthemnum \label{mDI}:] Display Interface Module
\item [\refstepcounter{mnum} \mthemnum \label{mII}:] Input Interface Module
\item [\refstepcounter{mnum} \mthemnum \label{mFI}:] File Interface Module
\item [\refstepcounter{mnum} \mthemnum \label{mGSP}:] Geometry State Parser Module
\item [\refstepcounter{mnum} \mthemnum \label{mGSC}:] Geometry State Converter Module
\item [\refstepcounter{mnum} \mthemnum \label{mCP}:] Commands Parser Module
\item [\refstepcounter{mnum} \mthemnum \label{mMC}:] Mode Commands Module
\item [\refstepcounter{mnum} \mthemnum \label{mSI}:] Shape Interface Module
\item [\refstepcounter{mnum} \mthemnum \label{mUPref}:] User Preference Module
\item [\refstepcounter{mnum} \mthemnum \label{mTB}:] Text Buffer Module
\item [\refstepcounter{mnum} \mthemnum \label{mGS}:] Geometry State Module
\item [\refstepcounter{mnum} \mthemnum \label{mGSM}:] Geometry State Mutator Module
\item [\refstepcounter{mnum} \mthemnum \label{mUR}:] Undo Redo Module
\item [\refstepcounter{mnum} \mthemnum \label{mUPer}:] User Persistence Module
\end{description}


\begin{table}[h!]
\centering
\begin{tabular}{p{0.3\textwidth} p{0.6\textwidth}}
\toprule
\textbf{Level 1} & \textbf{Level 2}\\
\midrule

  \multirow{3}{0.3\textwidth}{Hardware-Hiding Module} & {Display Interface Module (\mref{mDI})}\\
  & {Input Interface Module (\mref{mII})}\\
  & {File Interface Module (\mref{mFI})}\\
\midrule

  \multirow{6}{0.3\textwidth}{Behaviour-Hiding Module} & {Geometry State Parser Module (\mref{mGSP})}\\
  & {Geometry State Converter Module (\mref{mGSC})}\\
& {Command Parser Module  (\mref{mCP})}\\
& {Mode Commands Module (\mref{mMC})}\\
& {Shape Interface Module (\mref{mSI})}\\
& {User Preference Module (\mref{mUPref})}\\
\midrule

\multirow{5}{0.3\textwidth}{Software Decision Module} & {Text Buffer Module (\mref{mTB})}\\
& {Geometry State Module (\mref{mGS})}\\
& {Geometry State Mutator Module (\mref{mGSM})}\\
  & {Undo Redo Module (\mref{mUR})}\\
  & {User Persistence Module (\mref{mUPer})}\\
\bottomrule

\end{tabular}
\caption{Module Hierarchy}
\label{TblMH}
\end{table}

\section{Connection Between Requirements and Design} \label{SecConnection}

The design of the system is intended to satisfy the requirements developed in
the SRS. In this stage, the system is decomposed into modules. The connection
between requirements and modules is listed in Table~\ref{TblRT}.

\wss{The intention of this section is to document decisions that are made
  ``between'' the requirements and the design.  To satisfy some requirements,
  design decisions need to be made.  Rather than make these decisions implicit,
  they are explicitly recorded here.  For instance, if a program has security
  requirements, a specific design decision may be made to satisfy those
  requirements with a password.}

\section{Module Decomposition} \label{SecMD}

Modules are decomposed according to the principle of ``information hiding''
proposed by \citet{ParnasEtAl1984}. The \emph{Secrets} field in a module
decomposition is a brief statement of the design decision hidden by the
module. The \emph{Services} field specifies \emph{what} the module will do
without documenting \emph{how} to do it. For each module, a suggestion for the
implementing software is given under the \emph{Implemented By} title. If the
entry is \emph{OS}, this means that the module is provided by the operating
system or by standard programming language libraries.  \emph{\progname{}} means the
module will be implemented by the \progname{} software.

Only the leaf modules in the hierarchy have to be implemented. If a dash
(\emph{--}) is shown, this means that the module is not a leaf and will not have
to be implemented.

\subsection{Hardware Hiding Modules}

\subsubsection{Display Interface Module (\mref{mDI})}

\begin{description}
\item[Secrets:] Implementation of rendering to the screen.
\item[Services:] Provides API to render graphics and text to the
  screen.
\item[Implemented By:] [OS, Electron, React]
\end{description}

\subsubsection{Input Interface Module (\mref{mII})}

\begin{description}
\item[Secrets:] User input event handling.
\item[Services:] Provides API to capture user input events (i.e. keyboard input).
\item[Implemented By:] [OS, Electron]
\end{description}

\subsubsection{File Interface Module (\mref{mFI})}

\begin{description}
\item[Secrets:] File system interaction.
\item[Services:] Provides API to read and write files (open, save).
\item[Implemented By:] [OS]
\end{description}

\subsection{Behaviour-Hiding Module}

\subsubsection{Geometry State Parser Module (\mref{mGSP})}

\begin{description}
\item[Secrets:]Logic for parsing the input file format into internal geometry representation state.
\item[Services:]Parses input files and converts them into the internal data structures used by the system.
\item[Implemented By:] [\progname{}]
\item[Type of Module:] [Abstract Object]
\end{description}

\subsubsection{Geometry State Converter Module (\mref{mGSC})}

\begin{description}
\item[Secrets:]Logic for converting internal geometry representation state into output file format.
\item[Services:] Converts internal data structures into output files (e.g. xml for local saves, pdf for exporting).
\item[Implemented By:] [\progname{}]
\item[Type of Module:] [Abstract Object]
\end{description}

\subsubsection{Command Parser Module  (\mref{mCP})}

\begin{description}
\item[Secrets:]Mapping between raw input events to editor commands.
\item[Services:]Converts keyboard input into internal commands.
\item[Implemented By:] [\progname{}]
\item[Type of Module:] [Abstract Object]
\end{description}

\subsubsection{Mode Commands Module (\mref{mMC})}

\begin{description}
\item[Secrets:]Mapping between active mode and available commands.
\item[Services:]Provides the set of available commands based on the current mode.
\item[Implemented By:] [\progname{}]
\item[Type of Module:] [Record]
\end{description}

\subsubsection{Shape Interface Module (\mref{mSI})}

\begin{description}
  \item[Secrets:]Data structure containing shape properties and methods (e.g. size, position).
\item[Services:] Provides an interface for creating and manipulating a shape in the system.
\item[Implemented By:] [\progname{}]
\item[Type of Module:] [Abstract Data Type]
\end{description}

\subsubsection{User Preference Module (\mref{mUPref})}

\begin{description}
  \item[Secrets:]Data structure containing user preferences (e.g. theme, shortcuts) and actions for creating them.
\item[Services:] Provides an interface for creating and manipulating user preferences in the system.
\item[Implemented By:] [\progname{}]
\item[Type of Module:] [Abstract Object]
\end{description}

\subsection{Software Decision Module}

\subsubsection{Text Buffer Module (\mref{mTB})}

\begin{description}
\item[Secrets:]Text data structure.
\item[Services:]Provides editing functionality for text data.
\item[Implemented By:] [\progname{}]
\item[Type of Module:] [Abstract Data Type]
\end{description}

\subsubsection{Geometry State Module (\mref{mGS})}

\begin{description}
\item[Secrets:]Geometry state data structure.
\item[Services:]Holds the current state of the geometry.
\item[Implemented By:] [\progname{}]
\item[Type of Module:] [Abstract Data Type]
\end{description}

\subsubsection{Geometry State Mutator Module (\mref{mGSM})}

\begin{description}
\item[Secrets:]Algorithms for updating geometry state.
\item[Services:]Provides methods to mutate the geometry state (add, delete, modify shapes).
\item[Implemented By:] [\progname{}]
\item[Type of Module:] [Abstract Object]
\end{description}

\subsubsection{Undo Redo Module (\mref{mUR})}

\begin{description}
\item[Secrets:]Algorithms for managing undo and redo stacks.
\item[Services:]Provides functionality to undo and redo actions performed on the geometry state.
\item[Implemented By:] [\progname{}]
\item[Type of Module:] [Abstract Object]
\end{description}


\subsubsection{User Persistence Module (\mref{mUPer})}

\begin{description}
\item[Secrets:]Mechanism for saving and loading user preferences and shortcuts.
\item[Services:]Provides functionality for persisting user preferences and shortcuts.
\item[Implemented By:] [\progname{}]
\item[Type of Module:] [Abstract Object]
\end{description}

\section{Traceability Matrix} \label{SecTM}

This section shows two traceability matrices: between the modules and the
requirements and between the modules and the anticipated changes.

% the table should use mref, the requirements should be named, use something
% like fref
\begin{table}[H]
\centering
\begin{tabular}{p{0.2\textwidth} p{0.6\textwidth}}
\toprule
\textbf{Req.} & \textbf{Modules}\\
\midrule
R1 & \mref{mHH}, \mref{mInput}, \mref{mParams}, \mref{mControl}\\
R2 & \mref{mInput}, \mref{mParams}\\
R3 & \mref{mVerify}\\
R4 & \mref{mOutput}, \mref{mControl}\\
R5 & \mref{mOutput}, \mref{mODEs}, \mref{mControl}, \mref{mSeqDS}, \mref{mSolver}, \mref{mPlot}\\
R6 & \mref{mOutput}, \mref{mODEs}, \mref{mControl}, \mref{mSeqDS}, \mref{mSolver}, \mref{mPlot}\\
R7 & \mref{mOutput}, \mref{mEnergy}, \mref{mControl}, \mref{mSeqDS}, \mref{mPlot}\\
R8 & \mref{mOutput}, \mref{mEnergy}, \mref{mControl}, \mref{mSeqDS}, \mref{mPlot}\\
R9 & \mref{mVerifyOut}\\
R10 & \mref{mOutput}, \mref{mODEs}, \mref{mControl}\\
R11 & \mref{mOutput}, \mref{mODEs}, \mref{mEnergy}, \mref{mControl}\\
\bottomrule
\end{tabular}
\caption{Trace Between Requirements and Modules}
\label{TblRT}
\end{table}

\begin{table}[H]
\centering
\begin{tabular}{p{0.2\textwidth} p{0.6\textwidth}}
\toprule
\textbf{AC} & \textbf{Modules}\\
\midrule
\acref{acHardware} & \mref{mHH}\\
\acref{acInput} & \mref{mInput}\\
\acref{acParams} & \mref{mParams}\\
\acref{acVerify} & \mref{mVerify}\\
\acref{acOutput} & \mref{mOutput}\\
\acref{acVerifyOut} & \mref{mVerifyOut}\\
\acref{acODEs} & \mref{mODEs}\\
\acref{acEnergy} & \mref{mEnergy}\\
\acref{acControl} & \mref{mControl}\\
\acref{acSeqDS} & \mref{mSeqDS}\\
\acref{acSolver} & \mref{mSolver}\\
\acref{acPlot} & \mref{mPlot}\\
\bottomrule
\end{tabular}
\caption{Trace Between Anticipated Changes and Modules}
\label{TblACT}
\end{table}

\section{Use Hierarchy Between Modules} \label{SecUse}

In this section, the uses hierarchy between modules is
provided. \citet{Parnas1978} said of two programs A and B that A {\em uses} B if
correct execution of B may be necessary for A to complete the task described in
its specification. That is, A {\em uses} B if there exist situations in which
the correct functioning of A depends upon the availability of a correct
implementation of B.  Figure \ref{FigUH} illustrates the use relation between
the modules. It can be seen that the graph is a directed acyclic graph
(DAG). Each level of the hierarchy offers a testable and usable subset of the
system, and modules in the higher level of the hierarchy are essentially simpler
because they use modules from the lower levels.

The uses hierarchy for our system organizes the modules into a clear, layered structure 
that minimizes coupling and simplifies testing. At the top of the hierarchy are the user-facing 
modules—Display Interface (M1) and Input Interface (M2)—which collect user input and present 
output but perform no processing themselves. These modules rely on the Mode Commands Module (M7), 
which acts as the central coordinator that interprets high-level interactions and dispatches tasks 
to more specialized behavioural modules. M7 uses the Command Parser (M6), Shape Interface (M8), 
and Text Buffer (M10) modules to interpret commands, manipulate shapes, and edit textual content. 
None of these behavioural modules modify the document state directly; instead, they depend on the 
Geometry State Mutator (M12), which centralizes all updates to the application’s data. M12 writes 
changes to both the Geometry State Module (M11), which stores the current in-memory representation 
of the document, and to the User Persistence Module (M14), which handles permanent storage tasks 
such as saving, loading, and managing autosave or history information. User preferences (M9) are 
also stored through this persistence layer. The Geometry State Module (M11) is then used by 
lower-level utility modules—Geometry State Parser (M4), Geometry State Converter (M5), File Interface 
(M3), and Undo Redo Module (M13)—each of which derives additional representations or services from 
the core state. This structure forms a directed acyclic graph with a predominantly downward flow of 
dependencies, ensuring that higher-level modules rely on simpler, stable, and easily testable 
lower-level modules.

\begin{figure}[H]
\centering
\includegraphics[width=0.7\textwidth]{../DesignAssets/Module_Hierarchy_Figure.png}
\caption{Use hierarchy among modules}
\label{FigUH}
\end{figure}

%\section*{References}

\section{User Interfaces}

A rough drawing / sketch of our basic functionaility of creating elements / diagrams:

\begin{figure}[H]
\centering
\includegraphics[width=0.9\textwidth]{../DesignAssets/prototype_sketch_1.jpg}
\caption{Sketch of UI focusing on creation boxes and legends}
\label{fig:UI}
\end{figure}


\section{Design of Communication Protocols}

This section was not deemed appropriate for our project.

\section{Timeline}

\begin{table}[H]
\centering
\begin{tabularx}{\textwidth}{p{0.15\textwidth} p{0.18\textwidth} X p{0.22\textwidth} p{0.15\textwidth}}
\toprule
\textbf{Phase} & \textbf{Dates} & \textbf{Focus} & \textbf{Modules} & \textbf{Member(s)} \\
\midrule

1 - Foundations & Oct 1 -- Nov 10 & Implement file I/O and basic note text storage & File Interface, Text Buffer & Ethan Hussain \\

2 - Core Editing & Nov 10 -- Dec 15 & Adding editing operations, undo/redo stack & Undo Redo, Text Buffer & Ethan Hussain \\

3 - UI Layer & Dec 1 -- Jan 20 & Create an editor window and input capture & Display Interface, Input Interface & Jeffrey Chengze \\

4 - Command and Mode Logic & Jan 10 -- Feb 15 & Shortcuts, editing modes, keyboard mappings & Command Parser, Mode Commands & Jeffrey Chengze \\

5 - Preferences & Feb 1 -- Mar 10 & Store and load user settings & User Preferences, User Persistence & Kevin \\

6 - Optional Geometry and Drawing Tools & Feb 15 -- Apr 10 & Add diagram/sketch layer support & Geometry State, Mutator, Parser, Converter, Shape Interface & Chengze, Kevin, Ethan \\

7 - Integration & Apr 1 -- Apr 25 & Full app integration, testing, and report write-ups & All Modules & All \\

\bottomrule
\end{tabularx}
\caption{Development Timeline and Module Responsibilities}
\label{tab:timeline}
\end{table}

\bibliographystyle {plainnat}
\bibliography{../../../refs/References}

\newpage{}

\end{document}
