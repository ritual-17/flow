\begin{enumerate}
  \item What went well while writing this deliverable?
  \begin{itemize}
        \item Chengze - I was responsible for the systems section, and 
        I think the structure and logic of the document came together 
        well. Once I clearly understood how each subsystem interacted, 
        writing the descriptions became much easier. Collaborating with 
        teammates also helped ensure that my section aligned with the 
        overall SRS structure.
        \item Ethan - Something that went well while working on the SRS was the
          elicitation of requirements that are maybe not necessarily new or
          specific to our product, but are necessary for it to be successful
          and usable. In our first deliverable problem statement we focused
          mostly on selling points and novel functionality, but in this
          document we fleshed out more practical requirements for what a note
          taking app should have. This included requirements such as the need
          to be able to save files. This is an obvious requirement, but
          sometimes it is easy to skip over documenting them because they are
          almost implicit.
        \item Hussain - For part of this deliverable, I was assigned to work on
         the risk mitigation section. In the previous deliverable, I was assign
         proof of concept. I think something that went well while working on
         this deliverable was that I was able to build on the knowledge I
         gained from the previous deliverable. I was able to add to my previous
         knowledge and expand on feedback to make this section better.
        \item Jeffrey - During this deliverable, I found that distrbuting tasks
        went extremely well. For the SRS docuement, there were 4 main topic areas:
        Goals, Environment, Project, and Systems. We were able to assign a 
        section for each member to be an "expert" in, which made this 
        deliverable proceed smoothly. Note, each member was involved in the 
        other sections; however, the members each had a priority section assigned 
        to them. 
        \item Kevin - 
  \end{itemize} 
  \item What pain points did you experience during this deliverable, and how did
  you resolve them?
  \begin{itemize}
        \item Chengze - It was challenging to work on my section(Systems) without 
        constant communication since the system design depended on finalized 
        functional requirements from other parts of the SRS. I resolved this 
        by checking the latest commits regularly and syncing with the team when 
        major updates were made to ensure consistency across sections.
        \item Ethan - One part of this deliverable that was challenging was
          splitting up the work and working in parallel with my team members.
          The sections in this document are very inter-related, so it required
          lots of discussion to make sure we were all on the same page. To
          overcome this, we made sure to ask questions as they arose and
          discuss. Another challenge was organization and tracking of work
          tasks. There were many more parts to work on here than our previous
          deliverable, so it was that much more important to track big tasks
          (e.g. deliverable work) and small tasks (e.g. document formatting and
          folder clean up chores) to make sure everything gets done. Overall,
          we came up with a system that works well and allows us to view
          relevant tasks in our kanban board sorted by milestone.
        \item Hussain - One pain point i experienced was dividing up the work
            equally. I think this was a pain point because some sections were
            larger than others, and it was hard to make sure everyone had a fair
            amount of work. To resolve this, we tried our best to split up the
            work equally, and we also helped each other out when someone was
            struggling with their section.
        \item Jeffrey - While working on this deliverable, a pain point / issue
        I came across was adding a new section S7 to the SRS document. This
        section was added as a result of the Hazard Analysis document requesting
        safety requirements be included here. The issue was not knowing
        where we should include this and how to explain the new section.
        To resolve this issue, I discussed with my team members on how we all
        think would be best to proceed. Since a team member was working more 
        closely with the systems portion, they recommended adding a S7 with my
        portion of the Hazard Analysis in it. To explain the new section,
        another member decided to create an introduction document where the
        new section is mentioned. 
        \item Kevin - 
  \end{itemize} 
  \item How many of your requirements were inspired by speaking to your
  client(s) or their proxies (e.g. your peers, stakeholders, potential users)?
  \begin{itemize}
        \item Most of the requirements were based on internal design discussions 
        rather than external feedback. As we are students ourselves, many of the 
        requirements would have naturally came from our own ideas. Reviewing 
        similar note-taking tools and discussing usability with peers helped 
        identify which system functions would feel most natural to users, such 
        as the canvas-based note interface.
  \end{itemize} 
  \item Which of the courses you have taken, or are currently taking, will help
  your team to be successful with your capstone project.
  \begin{itemize}
        \item Chengze - My experience in Software Design and Software Requirements 
        and Specifications has been especially useful. These courses taught me how 
        to break down systems into components, identify functional dependencies, and 
        document them clearly — all of which directly applied to writing the systems section.
        \item Ethan - Many courses have had relevant material to help us in
          this project. From a technical perspective, some that come to mind
          include 2AA4 and 3A04. These two courses heavily focused on the
          design of software systems and collaborating in a team while doing
          so, just on a smaller scale than what we are doing in capstone. From
          a requirements perspective, 3RA3 and 4HC3 are extremely relevant, as
          we learned about requirements, why they are important, and how to use
          them to deliver a good product to users.
        \item Hussain - I have previously taken courses such as 3RA3 and a web
        development course, COMP 466. 3RA3 being a requirements course greatly 
        helped me understand how to write good requirements and the importance
        of them. COMP 466 helped me practice using certain tools building out
        web applications, which will be relevant to our project as we are 
        building a web application.
        \item Jeffrey - A course that I've taken which helped with this deliverable
        specifically is the Software Engineering Requirements course 3RA3. This course
        helped me with formatting the SRS and writing requirements in a software space.
        Additionally, I believe that the engineering design courses 1P13, 2PX3, 
        and 3PX3, will be a great asset to the success of this capstone project 
        as it teaches engineering design principals and gives students experience 
        with working on a technical team. 
        \item Kevin - 
  \end{itemize} 
  \item What knowledge and skills will the team collectively need to acquire to
  successfully complete this capstone project?  Examples of possible knowledge
  to acquire include domain specific knowledge from the domain of your
  application, or software engineering knowledge, mechatronics knowledge or
  computer science knowledge.  Skills may be related to technology, or writing,
  or presentation, or team management, etc.  You should look to identify at
  least one item for each team member.
  \begin{itemize}
        \item Chengze - From a systems perspective, we’ll need a stronger understanding 
        of client–server communication and data handling. Personally, I also want to 
        improve in designing scalable architectures and integrating front-end and back-end 
        components effectively.
        \item Ethan - Personally, I will have to learn more about how the
          backend layer for native apps work. I have done a lot of web dev
          work, but the idioms and quirks related to native app development
          will be new for me. Although I will have to do more work to learn
          what I don't know, one thing I know I need to learn is how a long
          running local app process stores global state.
        \item Hussain - To complete this project successfully, I believe our
        team will require deep knowledge in app development. This includes
        understanding frameworks, libraries, and best practices for building
        robust and user-friendly applications. Personally, I will need to work
        my architecture and system design skills. This includes learning how to
        break down complex systems into manageable components, and how to
        design systems that are scalable and maintainable. I also think 
        web-development skills will be very important. This is because we are
        building a web application, so understanding front-end and back-end
        development, as well as how to integrate the two, will be crucial.
        \item Jeffrey - The technical knowledge and skills that are required to
        complete this capstone project are application development skills (frontend and
        backend), git and GitHub knowledge, proper coding review skills, and 
        UI/UX design skills. These skills are important and are a good base
        to obtain when attempting to develop a notetaking application.
        Additionally, I believe a decent amount of 
        interpersonal skills such as communication and organization
        will benefit the success of this project. It is important to 
        be able to communicate with team members and organize data clearly
        when working on a capstone project with other members. 
        Personally, I believe I need to work on acquiring better application
        development skills. This is an very important skill to have when
        creating a notetaking application and I feel I do not have many
        past experiences in application development.
        \item Kevin - 
  \end{itemize} 
  \item For each of the knowledge areas and skills identified in the previous
  question, what are at least two approaches to acquiring the knowledge or
  mastering the skill?  Of the identified approaches, which will each team
  member pursue, and why did they make this choice?
  \begin{itemize}
        \item Chengze - To strengthen my system design and integration skills, 
        I plan to study open-source architecture examples and practice by setting 
        up small demo projects that connect APIs and front-end interfaces. I’ll also 
        review documentation on frameworks our team uses to understand their structure 
        better. Hands-on learning and reviewing teammates’ implementations will help 
        me apply theory to our project quickly.
        \item Ethan - Two approaches to acquiring this knowledge are reading
          documentation and watching video tutorials. Personally, I will be
          using both approaches. Video tutorials work well for my learning,
          especially starting out with big concepts. Once I get more
          comfortable and my knowledge gaps become more fine grained, I find
          documentation to be a lot more helpful for answering my questions.
        \item Hussain - One knowledge area I identified was web development. Two
        approaches to acquiring this knowledge are taking online courses and
        building small projects. Thankfully, I have taken a web development
        course before, so I have a good foundation to build on. This makes me a
        good candidate to take the lead on web development tasks. Another way
        to acquire this knowledge is by building small projects. Thankfully, I 
        also help run the MSA website, so I have experience building and
        maintaining a website. 
        \item Jeffrey - For my section specifically, I outlined that I should
        work on acquiring more application development skills. Some approaches
        in acquiring these skills would include: working through a code
        bootcamp over the winter break and review consistantly. Moreover,
        in order to keep up with my peers, I should be constantly asking
        questions to those members more experienced in this skill. Both
        of these approaches will benefit me in being a successful 
        contributing member of this capstone project. Moreover, with 
        the use of AI tools such as Co-Pilot, I am able to ask generic
        application development questions in order to obtain knowledge
        in this field.
        \item Kevin - 
  \end{itemize} 
\end{enumerate}
