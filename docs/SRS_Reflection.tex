\begin{enumerate}
  \item What went well while writing this deliverable?
  \begin{itemize}
        \item Chengze - I was responsible for the systems section, and 
        I think the structure and logic of the document came together 
        well. Once I clearly understood how each subsystem interacted, 
        writing the descriptions became much easier. Collaborating with 
        teammates also helped ensure that my section aligned with the 
        overall SRS structure.
        \item Ethan - 
        \item Hussain - 
        \item Jeffrey - During this deliverable, I found that distrbuting tasks
        went extremely well. For the SRS docuement, there were 4 main topic areas:
        Goals, Environment, Project, and Systems. We were able to assign a 
        section for each member to be an "expert" in, which made this 
        deliverable proceed smoothly. Note, each member was involved in the 
        other sections; however, the members each had a priority section assigned 
        to them. 
        \item Kevin - 
  \end{itemize} 
  \item What pain points did you experience during this deliverable, and how did
  you resolve them?
  \begin{itemize}
        \item Chengze - It was challenging to work on my section(Systems) without 
        constant communication since the system design depended on finalized 
        functional requirements from other parts of the SRS. I resolved this 
        by checking the latest commits regularly and syncing with the team when 
        major updates were made to ensure consistency across sections.
        \item Ethan - 
        \item Hussain - 
        \item Jeffrey - While working on this deliverable, a pain point / issue
        I came across was adding a new section S7 to the SRS document. This
        section was added as a result of the Hazard Analysis document requesting
        safety requirements be included here. The issue was not knowing
        where we should include this and how to explain the new section.
        To resolve this issue, I discussed with my team members on how we all
        think would be best to proceed. Since a team member was working more 
        closely with the systems portion, they recommended adding a S7 with my
        portion of the Hazard Analysis in it. To explain the new section,
        another member decided to create an introduction document where the
        new section is mentioned. 
        \item Kevin - 
  \end{itemize} 
  \item How many of your requirements were inspired by speaking to your
  client(s) or their proxies (e.g. your peers, stakeholders, potential users)?
  \begin{itemize}
        \item Chengze - Most of my work was based on internal design discussions 
        rather than external feedback. However, reviewing similar note-taking tools 
        and discussing usability with peers helped me identify which system functions 
        would feel most natural to users, such as the canvas-based note interface.
        \item Ethan - 
        \item Hussain - 
        \item Jeffrey - 
        \item Kevin - 
  \end{itemize} 
  \item Which of the courses you have taken, or are currently taking, will help
  your team to be successful with your capstone project.
  \begin{itemize}
        \item Chengze - My experience in Software Design and Software Requirements 
        and Specifications has been especially useful. These courses taught me how 
        to break down systems into components, identify functional dependencies, and 
        document them clearly — all of which directly applied to writing the systems section.
        \item Ethan - 
        \item Hussain - 
        \item Jeffrey - A course that I've taken which helped with this deliverable
        specifically is the Software Engineering Requirements course 3RA3. This course
        helped me with formatting the SRS and writing requirements in a software space.
        Additionally, I believe that the engineering design courses 1P13, 2PX3, 
        and 3PX3, will be a great asset to the success of this capstone project 
        as it teaches engineering design principals and gives students experience 
        with working on a technical team. 
        \item Kevin - 
  \end{itemize} 
  \item What knowledge and skills will the team collectively need to acquire to
  successfully complete this capstone project?  Examples of possible knowledge
  to acquire include domain specific knowledge from the domain of your
  application, or software engineering knowledge, mechatronics knowledge or
  computer science knowledge.  Skills may be related to technology, or writing,
  or presentation, or team management, etc.  You should look to identify at
  least one item for each team member.
  \begin{itemize}
        \item Chengze - From a systems perspective, we’ll need a stronger understanding 
        of client–server communication and data handling. Personally, I also want to 
        improve in designing scalable architectures and integrating front-end and back-end 
        components effectively.
        \item Ethan - 
        \item Hussain - 
        \item Jeffrey - The technical knowledge and skills that are required to
        complete this capstone project are application development skills (frontend and
        backend), git and GitHub knowledge, proper coding review skills, and 
        UI/UX design skills. These skills are important and are a good base
        to obtain when attempting to develop a notetaking application.
        Additionally, I believe a decent amount of 
        interpersonal skills such as communication and organization
        will benefit the success of this project. It is important to 
        be able to communicate with team members and organize data clearly
        when working on a capstone project with other members. 
        Personally, I believe I need to work on acquiring better application
        development skills. This is an very important skill to have when
        creating a notetaking application and I feel I do not have many
        past experiences in application development.
        \item Kevin - 
  \end{itemize} 
  \item For each of the knowledge areas and skills identified in the previous
  question, what are at least two approaches to acquiring the knowledge or
  mastering the skill?  Of the identified approaches, which will each team
  member pursue, and why did they make this choice?
  \begin{itemize}
        \item Chengze - To strengthen my system design and integration skills, 
        I plan to study open-source architecture examples and practice by setting 
        up small demo projects that connect APIs and front-end interfaces. I’ll also 
        review documentation on frameworks our team uses to understand their structure 
        better. Hands-on learning and reviewing teammates’ implementations will help 
        me apply theory to our project quickly.
        \item Ethan - 
        \item Hussain - 
        \item Jeffrey - For my section specifically, I outlined that I should
        work on acquiring more application development skills. Some approaches
        in acquiring these skills would include: working through a code
        bootcamp over the winter break and review consistantly. Moreover,
        in order to keep up with my peers, I should be constantly asking
        questions to those members more experienced in this skill. Both
        of these approaches will benefit me in being a successful 
        contributing member of this capstone project. Moreover, with 
        the use of AI tools such as Co-Pilot, I am able to ask generic
        application development questions in order to obtain knowledge
        in this field.
        \item Kevin - 
  \end{itemize} 
\end{enumerate}
