\begin{enumerate}
  \item What went well while writing this deliverable?
  \begin{itemize}
        \item Chengze - 
        \item Ethan - 
        \item Hussain - 
        \item Jeffrey - 
        \item Kevin - During this deliverable our group managed to effectively communicate and quickly assign work and work to review to each other. 
  \end{itemize} 
  \item What pain points did you experience during this deliverable, and how did
  you resolve them?
  \begin{itemize}
        \item Chengze - 
        \item Ethan - 
        \item Hussain - 
        \item Jeffrey - 
        \item Kevin - One pain point was getting our group to finish the work and have time for review and git merging. To resolve this we set earlier fake deadlines where our group would have the work done by, giving time for reviews. 
  \end{itemize} 
  \item How many of your requirements were inspired by speaking to your
  client(s) or their proxies (e.g. your peers, stakeholders, potential users)?
  \begin{itemize}
        \item Chengze - 
        \item Ethan - 
        \item Hussain - 
        \item Jeffrey - 
        \item Kevin - As we are students and a potential stakeholder, many of our requirements would have naturally been inspired by our thoughts. However, this meant that we did not seek as much outside influence as we could have. 
  \end{itemize} 
  \item Which of the courses you have taken, or are currently taking, will help
  your team to be successful with your capstone project.
  \begin{itemize}
        \item Chengze - 
        \item Ethan - 
        \item Hussain - 
        \item Jeffrey - 
        \item Kevin - Courses that would help me do this capstone project are project management courses like 2RA3, 3A04 and 3RA3. These courses help as they give me experience for working in a group and using git. Also, courses like 4HC3 help me by specifying how to design interfaces, which our project will be using.  
  \end{itemize} 
  \item What knowledge and skills will the team collectively need to acquire to
  successfully complete this capstone project?  Examples of possible knowledge
  to acquire include domain specific knowledge from the domain of your
  application, or software engineering knowledge, mechatronics knowledge or
  computer science knowledge.  Skills may be related to technology, or writing,
  or presentation, or team management, etc.  You should look to identify at
  least one item for each team member.w
  \begin{itemize}
        \item Chengze - 
        \item Ethan - 
        \item Hussain - 
        \item Jeffrey - 
        \item Kevin - I felt that I would need to acquire info on how to properly set up keyboard inputs for programs. Most programs that I have created have used either text or file input and minimised use of keyboard except for typing text. 
  \end{itemize} 
  \item For each of the knowledge areas and skills identified in the previous
  question, what are at least two approaches to acquiring the knowledge or
  mastering the skill?  Of the identified approaches, which will each team
  member pursue, and why did they make this choice?
  \begin{itemize}
        \item Chengze - 
        \item Ethan - 
        \item Hussain - 
        \item Jeffrey - 
        \item Kevin - Two approaches to acquire the info I need to implement this sort of system would be to use libraries that implement it, or to alternately study how it is done by other programs. Using provided libraries would be easier and faster and help me learn how it was implemented. However, I would learn much less than if I decided to properly learn and implement my own system. Dues to this reason I will most likely instead opt to study and implement my own system  if time allows. This is not without problems as I many introduce bias from the systems that I looked over.
  \end{itemize} 
\end{enumerate}
